% !TEX root
% \documentclass[oneside]{book}
\documentclass[a5paper,10pt,oneside]{report}

\usepackage{stat-methods}

\begin{document}

\title{Обработка результатов учебного эксперимента}
\author{П.В. Попов, А.А. Нозик}

\maketitle
\tableofcontents

\listoftodos

\chapter*{Предисловие}
    Данное пособие содержит краткое изложение основных понятий и методов,
    необходимых для корректной обработки результатов экспериментов в учебной
    лаборатории, а также указания по представлению результатов лабораторных
    работ, с тем чтобы оформленные студентами отчёты в целом соответствовали
    сложившимся современным стандартам оформления научных публикаций.

    Умение работать с погрешностями, или \textquote{ошибками},
    является важной частью любого научного эксперимента на всех его этапах.
    Так, при подготовке и проведении эксперимента необходимо знать точность
    используемых приборов, уметь находить пути возможного уменьшения ошибок,
    разумно организовать сами измерения и правильно оценивать точность
    полученных значений. На этапе обработки возникает необходимость пересчитывать
    возможную ошибку в конечных результатах по известным оценкам погрешностей
    в исходных данных. А на самом важном этапе --- интерпретации
    результатов эксперимента --- без знания точности проведённых
    измерений и без корректной статистической обработки невозможно делать
    обоснованные выводы в пользу той или иной физической модели, той или
    иной гипотезы.

    Пособие может быть рекомендовано студентам первого курса для первичного
    ознакомления с предметом, либо в качестве \textquote{шпаргалки}
    по основным формулам. Более подробное изложение материала, снабжённое
    большим количеством практических примеров, можно найти в пособиях
    \cite{taylor,squires,zaidel}. Студентам старших курсов, знакомым
    с основами теории вероятностей и математической статистики, и желающим
    на более глубоком уровне познакомиться с современными методиками обработки
    данных, можно порекомендовать начать с книг \cite{hudson,idie}.

\chapter{Измерения и погрешности}

% \disclaimer{
%     Этот раздел предназначен для поверхностного ознакомления с измерением
% физических величин. Для более глубокого понимания рекомендуется сначала
% ознакомиться с основами теории вероятности, приведенными в главе \ref{ch:prob}.
% }

Свойства физического объекта (явления, процесса) определяются набором
количественных характеристик~--- \emph{физических величин}.
Результатом измерения может быть количество отсчётов некоторого события, логическое утверждение (да/нет) или даже качественная оценка (сильно/слабо/умеренно). 
Мы ограничимся наиболее типичным для физики случаем, когда результат отдельного измерения представляет собой \emph{число}, задающее отношение измеряемой величины к некоторому \emph{эталону}.
Сравнение с эталоном может быть как \emph{прямым} (проводится непосредственно экспериментатором), так и \emph{косвенным} (проводится с помощью некоторого прибора, которому экспериментатор доверяет).
%Полученные таким образом величины имеют \emph{размерность}, определяемую выбором эталона.

Взаимосвязь между различными физическими величинами описывается
\emph{физическими законами}, представляющими собой идеализированную
\emph{модель} действительности. Конечной целью любого физического
эксперимента является проверка адекватности и
уточнение параметров таких моделей.

\section{Результат измерения}

Рассмотрим простейший пример: измерение длины стержня
с помощью линейки. Линейка проградуирована производителем с помощью
некоторого стандартного эталона длины~--- таким образом, сравнивая длину
стержня со шкалой линейки, мы выполняем косвенное сравнение с
эталоном.

% Важно понимать, что результат измерения не отражает точную (абсолютную)
% характеристику объекта, но в зависимости от самого измерение является числом
% \emph{приблизительно} равным этому абсолютному значению.

Приложим линейку к стержню и получим по её шкале некоторый результат~$x=x_{изм}$. Можно ли утверждать, что $x_{изм}$~--- это и есть \emph{истинная} длина стержня? Ответ не столь очевиден, как может показаться.

Во-первых, значение~$x$ \emph{не может быть задано точно}. Хотя бы
потому, что оно обязательно \emph{округлено} до некоторой значащей
цифры: если линейка \textquote{обычная}, то у неё
есть \emph{цена деления}; а если \textquote{линейка}, к примеру, лазерная
--- у неё высвечивается конечное \emph{число значащих цифр}
на дисплее.

Во-вторых, истинная длина вполне может отличаться от измеренной больше,
чем на ошибку округления. Действительно,
мы могли приложить линейку не вполне ровно; сама линейка всегда
изготовлена не вполне точно; стержень не является идеальным цилиндром и т.\,п.

И, наконец, если пытаться хотя бы гипотетически переходить к бесконечной
точности измерения, теряет смысл само понятие \textquote{истинной} длины стержня. Ведь на масштабах атомов у стержня нет чётких границ, а значит говорить о его
геометрических размерах в таком случае крайне затруднительно! В общем случае 
под <<истинным значением>> физической величины всегда подразумевается 
некоторая \emph{идеализация}.

Итак, из нашего примера видно, что никакое физическое измерение не может быть
произведено абсолютно точно, то есть
\textbf{у любого измерения есть \emph{погрешность}}.%
\note{Также используют эквивалентный термин \emph{ошибка измерения}
(от \emph{англ.} error). Подчеркнём, что смысл этого термина отличается от
общеупотребительного бытового: если физик говорит \textquote{в измерении есть ошибка},
--- это не означает, что оно неправильно и его надо переделать.
Имеется ввиду лишь, что это измерение \emph{неточно}, то есть имеет
\emph{погрешность}.}

Определим количественно погрешность как разность между измеренным и \textquote{истинным} значением физической величины:
\[
    \delta x = x_{\text{изм}} - x_{\text{ист}}.
\]
Однако использование такого определения затруднено~---
величину~$\delta x$ невозможно непосредственно вычислить или измерить. Во-первых, из-за неизбежного наличия погрешностей мы можем знать только~$x_{изм}$,
но не~$x_{ист}$, и во-вторых, само значение~$x_{ист}$ может отличаться в разных измерениях (например, стержень
неровный или изогнутый, его торцы дрожат из-за тепловых флуктуаций и т.\,д.).
Поэтому на практике можно говорить лишь об \emph{оценке} величины погрешности.

Об измеренной величине также часто говорят как об \emph{оценке}, подчёркивая,
что эта величина не точна и зависит не только от физических свойств
исследуемого объекта, но и от процедуры измерения.

%\note{
%    Термин \emph{оценка} имеет и более формальное значение. Оценкой называют результат процедуры получения значения параметров физической модели, описывающей
%    изучаемое явление. Теория оценок является подразделом математической статистики. Некоторые ее положения изложены в главе \ref{ch:estimate}.
%%     , но для более серьезного понимания следует обратиться к \cite{idie}.
%}

Для оценки значения физической величины корректно использовать
не просто некоторое фиксированное число~$x_{\text{изм}}$, а \emph{интервал} (или
\emph{диапазон}) значений, в пределах которого может лежать её
\textquote{истинное} значение. В~простейшем случае этот интервал
может быть записан как
\[
    x = x_{\text{изм}} \pm \delta x,
\]
где $\delta x$~--- \emph{абсолютная} величина погрешности.
Эта запись означает, что исследуемая величина лежит в интервале
$x \in (x_{\text{изм}} - \delta x;\, x_{\text{изм}} + \delta x)$
с некоторой достаточно большой долей вероятности (более подробно о
вероятностном содержании интервалов см. п.~\ref{sec:normal}).
Для наглядной оценки точности измерения удобно также использовать
\emph{относительную} величину погрешности:
\[
    \varepsilon_{x} = \frac{\delta x}{x_{\text{изм}}}.
\]
Она показывает, насколько погрешность мала по сравнению с
самой измеряемой величиной (её также можно выразить в процентах:
$\varepsilon = \frac{\delta x}{x} \cdot 100\%$).

\example{Штангенциркуль ---
устройство для измерения длин с ценой деления $0{,}1\;\text{мм}$. Пусть
диаметр некоторой проволоки равен $0{,}37$~мм. Считая, что абсолютная
ошибка составляет половину цены деления, результат измерения
можно будет записать как $d=0{,}40\pm0{,}05\;\text{мм}$ (или
$d=(40\pm5)\cdot10^{-5}\;\text{м}$).
Относительная погрешность составляет $\varepsilon\approx13\%$, то
есть точность измерения весьма посредственная~--- это обусловлено 
близостью размера объекта к пределу точности прибора.}

\paragraph{О необходимости оценки погрешностей.}

Измерим длины двух стержней~$x_{1}$ и~$x_{2}$ и сравним результаты.
Можно ли сказать, что стержни одинаковы или различны?

Казалось бы,
достаточно проверить, справедливо ли $x_{1}=x_{2}$. Но \emph{никакие}
два результата измерения не равны друг другу с абсолютной точностью! Таким
образом, без указания погрешности измерения ответ на этот вопрос дать
\emph{невозможно}.

С другой стороны, если погрешность $\delta x$ известна, то можно
утверждать, что измеренные длины одинаковы
\emph{в пределах погрешности опыта}, если $|x_{2}-x_{1}|<\delta x$
(и различны в противоположном случае).

Итак, без знания погрешностей невозможно сравнить между собой никакие
два измерения, и, следовательно, невозможно сделать \emph{никаких}
значимых выводов по результатам эксперимента: ни о наличии зависимостей
между величинами, ни о практической применимости какой-либо теории,
и т.\,п. В связи с этим задача правильной оценки погрешностей является крайне
важной. Существенное занижение или завышение значения погрешности
(по сравнению с реальной точностью измерений) ведёт к \emph{неправильным выводам}.

В физическом эксперименте оценка погрешностей должна проводиться \emph{всегда}
(даже когда составители \textquote{задания} не упомянули об этом!).

\section{Многократные измерения}

Проведём серию из~$n$ \emph{одинаковых} (однотипных) измерений одной
и той же физической величины (например, многократно приложим линейку к стержню) и получим
ряд значений
\[
    \mathbf{x} = \left\{ x_{1},\,x_{2},\,\ldots\,,x_{n}\right\} .
\]
Что можно сказать о длине стержня по данному набору измерений?
И влияет ли число проведённых измерений на конечный результат?

Если цена деления самой линейки достаточно мала, то как нетрудно убедиться
на практике, величины $\left\{ x_{i}\right\}$ почти наверняка окажутся
\emph{различными}. Причиной тому могут быть
самые разные обстоятельства, например: у нас недостаточно остроты
зрения и точности рук, чтобы каждый раз прикладывать линейку одинаково;
стенки стержня могут быть слегка неровными; у стержня может и не быть
определённой длины, например, если в нём возбуждены звуковые волны,
из-за чего его торцы колеблются, и т.\,д.

В такой ситуации результат измерения является \emph{случайной величиной}, 
которую можно описать некоторым \emph{вероятностным} законом~---
\emph{распределением} (подробнее см. гл.~\ref{ch:prob}). 
Вычислим среднее арифметическое по набору результатов $\mathbf{x}$,
обозначив его угловыми скобками:
\begin{equation}
    \average{x} = \frac{x_{1}+x_{2}+\ldots+x_{n}}{n} \equiv 
    \frac{1}{n} \sum\limits_{i=1}^{n}x_{i}.
    \label{eq:average}
\end{equation}
Это значение, вычисленное по результатам \emph{конечного} числа~$n$ измерений,
принято называть \emph{выборочным} средним.

% Как правило предполагают, что распределение, характеризующее изучаемую величину
% при заданном способе измерения (на примере с линейкой легко убедиться, что
% результат зависит не только от изучаемого объекта но и от методики его
% исследования), устроено таким образом, что его среднее значение (математическое
% ожидание) соответствует истинной величине.
% \todo[author=ppv,inline]{А почему так предполагают? И что такое "истинная
% величина"? Надо переформулировать.}
% В этом случае разумно в качестве оценки значения по серии измерений использовать
% среднее арифметическое:

% Для количественной оценки разброса
% используют понятие \emph{дисперсии}. Понятие дисперсии и его формальное
% определение описаны в разделе \ref{ch:prob}. В этой главе ограничимся тем, что
% введем понятие среднеквадратичного отклонения (среднеквадратичное отклонение по
% определению равно корню из дисперсии), и его определение по серии измерений.

Кроме среднего представляет интерес и то, насколько сильно варьируются
результаты от опыта к опыту. Определим отклонение каждого измерения от среднего как
\[
    \Delta x_{i} = x_{i} - \average{x},\qquad i=1\ldots n.
\]
Разброс совокупности данных~$\{x_i\}$ относительно среднего принято характеризовать
\emph{среднеквадратичным} (или \emph{стандартным}) \emph{отклонением}:
\begin{equation}
    s = \sqrt{\frac{\Delta x_{1}^{2} + \Delta x_{2}^{2} +
              \ldots + \Delta x_{n}^{2}}{n}}
        \equiv \sqrt{\frac{1}{n} \sum\limits_{i=1}^{n} \Delta x_{i}^{2}}
\label{eq:sigma}
\end{equation}
или кратко
\begin{equation}
    s = \sqrt{\left<\Delta x^{2}\right>} \equiv 
        \sqrt{\left<(x - \left<x\right>)^{2}\right>}.
    \label{eq:sigma_s}
\end{equation}
Средний квадрат отклонения~$s^{2}$ называют \emph{выборочной дисперсией}.

% \note{Разброс значений можно характеризовать по-другому, например: средним
%     или максимальным модулем отклонения. Из дальнейшего рассмотрения будет видно,
%     почему использование среднеквадратичного значения математически наиболее удобно.
%
%     Среднее арифметическое является далеко не единственным способом получения оценки
% по выборке (набору данных), в некоторых случаях к примеру используют наиболее
% вероятное значение, но такие случаи лежат за пределами данного пособия.
% }

Будем увеличивать число измерений~$n$ ($n\to \infty$). Если объект измерения и методика
достаточно стабильны, то можно ожидать что отклонения от среднего~$\Delta x_i$ будут, во-первых,
относительно малы, а во-вторых, положительные и отрицательные отклонения будут
встречаться примерно одинаково часто. Тогда при вычислении \eqref{eq:average}
почти все отклонения $\Delta x_i$ компенсируются и можно ожидать,
что выборочное среднее при $n\gg 1$ будет стремиться к некоторому пределу:
\[
    \limaverage{x} = \lim_{n\to\infty}\frac{1}{n} \sum_{i=1}^{n}x_{i}.
\]
Тогда предельное значение $\limaverage{x}$, называемое в теории вероятностей \emph{математическим ожиданием}, 
можно отождествить с \textquote{истинным} средним для исследуемой величины.

Предельную величину среднеквадратичного отклонения обозначим как
\[
    \sigma = \lim\limits_{n\to\infty} \sqrt{\frac{1}{n} \sum\limits_{i=1}^{n} \Delta x_{i}^{2}}.
\]

\note{Вообще говоря указанные пределы могут и не существовать. Например, если величина меняется во времени или в результате самого измерения, либо испытывает слишком большие
случайные скачки и т.\,п. Такие ситуации требуют особого рассмотрения и мы на них не останавливаемся.}

Итак, если результаты небольшого числа
измерений имеют не слишком большой разброс, то величина~$\average{x}$
может быть использована как приближенное значение (\emph{оценка}) истинного значения
$\average{x} \approx \limaverage{x}$,
а увеличение числа измерений позволит уточнить результат. 
При этом выборочное среднее значение~$\average{x}$ оказывается с большей
вероятностью ближе к истинному значению~$\limaverage{x}$, чем каждое из 
измерений~$\left\{x_i\right\}$ в отдельности. При этом, как будет показано в
п.~\ref{sec:average}, погрешность вычисления среднего убывает пропорционально корню 
из числа измерений~$\sqrt{n}$.


%Многие случайные величины подчиняются так называемому \emph{нормальному закону}
%распределения (подробнее см. Главу~\ref{ch:prob}). Для таких величин
%могут быть строго доказаны следующие свойства:
%\begin{itemize}
%    \item при многократном повторении эксперимента б\'{о}льшая часть измерений
%    ($\sim$68\%) попадает в интервал $\limaverage{x}-\sigma<x<
%    \limaverage{x}+\sigma$
%    (см. п.~\ref{sec:normal}).
%    \item выборочное среднее значение $\average{x}$ оказывается с большей
%    вероятностью ближе к истинному значению $\limaverage{x}$, чем каждое из измерений
%    $\left\{x_i\right\}$ в отдельности. При этом ошибка вычисления среднего
%    убывает пропорционально корню из числа опытов $\sqrt{n}$
%    (см. п.~\ref{sec:average}).
%\end{itemize}

% В теории вероятности показывается (см. раздел \ref{ch:prob}), что среднее
% арифметическое значение величин с одинаковым распределением имеет значительно
% меньший разброс, чем единичное измерение.
%
% В математической статистике показывается, что если некоторая величина,
% распределенная по нормальному закону (см. \ref{sec:normal}) имеет среднее
% значение $\average{x}$ и стандартное отклонение $\sigma_x$, то при многократном
% повторении эксперимента приблизительно в 68\% случаев результат $x$ будет
% попадать в интервал $\average{x}-\sigma_{x}<x<\average{x}+\sigma_{x}.$

\enlargethispage{1em}

\note{Если $n$ мало ($n\lesssim 10$), для среднеквадратичного отклонения
    следует вместо \eqref{eq:sigma} использовать так называемую \emph{несмещенную} оценку:
    \begin{equation}
    s_{n-1}^2=\frac{1}{n-1}\sum\limits _{i=1}^{n}
    \Delta x_{i}^{2}\label{eq:sigma_straight},
    \end{equation}
    где произведена замена $n\to n-1$. Подробнее см.~\cite{idie}, Гл.~7.
}

\exercise{Показать, что
    \begin{equation}
    % s_{x}^{2}=\average{\left(x-\average{x}\right)^{2}}=\average{x^{2}-2x\average{x}
    % +\average{x}^{2}}=\average{x^{2}}-\average{x}^{2}.
        s^{2} = \average{x^{2}} - \average{x}^{2}.
        \label{eq:sigma_x2}
    \end{equation}
    то есть дисперсия равна разности среднего значения квадрата
    $\average{x^{2}} = \frac{1}{n} \sum\limits_{i=1}^{n} x_{i}^{2}$
    и квадрата среднего 
    $\average{x}^{2} = \left(\frac{1}{n}\sum\limits_{i=1}^{n}x_{i}\right)^{2}$.
}

\section{Классификация погрешностей}

Чтобы лучше разобраться в том, сколько раз нужно повторять измерения,
и в каком случае это позволит улучшить результаты опыта,
проанализируем источники и виды погрешностей.

В первую очередь, многократные измерения позволяют проверить
\emph{воспроизводимость} результатов: повторные измерения в \emph{одинаковых}
условиях, должны давать близкие результаты. В противном случае
исследование будет существенно затруднено, если вообще возможно.
Таким образом, многократные измерения \emph{необходимы} для того,
чтобы убедиться как в надёжности методики, так и в стабильности измеряемой
величины как таковой.
% \note{Невоспроизводимые явления исследовать, очевидно, невозможно. Например,
% мы почти ничего не знаем об устройстве \textquote{шаровой молнии},
% поскольку ещё никому не удалось создать стабильные условия для её
% получения, хотя свидетельства существования этого редкого атмосферного
% явления довольно многочисленны и надёжны.}

При любых измерениях возможны грубые ошибки~--- \emph{промахи}
(\emph{англ.} miss). Это \textquote{ошибки} в стандартном
понимании этого слова~--- возникающие по вине экспериментатора
или в силу других непредвиденных обстоятельств (например, из-за сбоя
аппаратуры). Промахов, конечно, нужно избегать, а результаты таких
измерений должны быть по возможности исключены из рассмотрения.

%\note{Порой и \emph{весь эксперимент} может оказаться одним большим
%\textquote{промахом}. Из недавних примеров можно вспомнить сверхсветовые нейтрино,
%якобы обнаруженные в 2011~г. на установке OPERA --- там причиной \textquote{открытия}
%оказался плохой контакт кабеля. Самый радикальный способ проверки достоверности
%результатов --- многократная независимая проверка на других
%приборах, другими методами, и желательно, другими экспериментаторами.}

Как понять, является ли \textquote{аномальный} результат промахом? Вопрос этот весьма
непрост. В литературе существуют статистические
критерии отбора промахов, которыми мы, однако, настоятельно \emph{не рекомендуем}
пользоваться (по крайней мере, без серьёзного понимания последствий
такого отбора). Отбрасывание аномальных данных может, во-первых, привести
к тенденциозному искажению результата исследований, а во-вторых, так
можно упустить открытие неизвестного эффекта. Поэтому при научных
исследованиях необходимо максимально тщательно проанализировать причину
каждого промаха, в частности, многократно повторив эксперимент. Только если факт и причина промаха установлены вполне достоверно,
соответствующий результат можно отбросить.

\note{Часто причины аномальных отклонений невозможно установить на этапе
обработки данных, поскольку часть информации о проведении измерений к этому моменту
утеряна. Единственным способ борьбы с этим~--- это максимально подробное описание всего
процесса измерений в \emph{лабораторном журнале}. Подробнее об этом
см. п.~\ref{sec:journal}.}

При многократном повторении измерений одной и той же физической величины
погрешности могут иметь \emph{систематический} либо \emph{случайный}
характер. Назовём погрешность \emph{систематической}, если она повторяется
от опыта к опыту, сохраняя свой знак и величину, либо \emph{закономерно}
меняется в процессе измерений. \emph{Случайные} (или \emph{статистические})
погрешности меняются хаотично при повторении измерений как по величине,
так и по знаку, и в изменениях не прослеживается какой-либо закономерности.

Кроме того, удобно разделять погрешности по их происхождению. Можно
выделить
\begin{itemize}\small
    \item \emph{инструментальные} (или \emph{приборные}) \emph{погрешности},
связанные с несовершенством конструкции (неточности, допущенные при
изготовлении или вследствие воздействия окружающей среды), ошибками калибровки или ненормативными условиями эксплуатации измерительных приборов;
    \item \emph{методические} \emph{погрешности}, связанные с несовершенством
теоретической модели явления (использование приближенных формул и
моделей явления) или с несовершенством методики измерения (например,
влиянием взаимодействия прибора и объекта измерения на результат измерения);
    \item \emph{естественные} \emph{погрешности}, связанные со случайным
характером измеряемой физической величины~--- они являются не столько
\textquote{ошибками} измерения, сколько характеризуют
природу изучаемого объекта или явления.
\end{itemize}

\note{Разделение погрешностей на систематические и случайные
не является однозначным и зависит от постановки опыта. Например, производя
измерения не одним, а несколькими однотипными приборами, мы переводим
систематическую приборную ошибку, связанную с неточностью шкалы и
калибровки, в случайную. Разделение по происхождению также условно,
поскольку любой прибор подвержен воздействию \textquote{естественных}
случайных и систематических ошибок (шумы и наводки, тряска, атмосферные
условия и т.\,п.), а в основе работы прибора всегда лежит некоторое
физическое явление, описываемое не вполне совершенной теорией.}

\subsection{Случайные погрешности}

Случайный характер присущ большому количеству физических
явлений, и в той или иной степени проявляется в работе всех без исключения
приборов. Случайные погрешности обнаруживаются просто при многократном
повторении опыта~--- в виде хаотичных изменений (\emph{флуктуаций})
значений~$\left\{x_{i}\right\}$.

Если случайные отклонения от среднего в большую или меньшую стороны
примерно равновероятны, можно рассчитывать, что при вычислении среднего
арифметического~\eqref{eq:average} эти отклонения компенсируются,
и погрешность результирующего значения $\average{x}$ будем меньше,
чем погрешность отдельного измерения.

Случайные погрешности бывают связаны, например,
\begin{itemize}\small
    \item \emph{с особенностями используемых приборов}: техническими
недостатками
(люфт в механических приспособлениях, сухое трение в креплении стрелки
прибора), с естественными (тепловой и дробовой шумы в электрических
цепях, тепловые флуктуации и колебания измерительных устройств из-за
хаотического движения молекул, космическое излучение) или техногенными
факторами (тряска, электромагнитные помехи и наводки);

    \item \emph{с особенностями и несовершенством методики измерения} (ошибка
при отсчёте по шкале, ошибка времени реакции при измерениях с секундомером);

    \item \emph{с несовершенством объекта измерений} (неровная поверхность,
неоднородность состава);

    \item \emph{со случайным характером исследуемого явления} (радиоактивный
распад, броуновское движение).
\end{itemize}

Остановимся несколько подробнее на двух последних случаях. Они отличаются
тем, что случайный разброс данных в них порождён непосредственно объектом
измерения. Если при этом приборные погрешности малы, то \textquote{ошибка}
эксперимента возникает лишь в тот момент, когда мы \emph{по своей
воле} совершаем замену ряда измеренных значений на некоторое среднее
$\left\{ x_{i}\right\} \to\average{x}$. Разброс данных при этом
характеризует не точность измерения, а \emph{сам исследуемый объект} или
явление. Однако с \emph{математической} точки зрения приборные и
\textquote{естественные}
погрешности \emph{неразличимы}~--- глядя на одни только
экспериментальные данные невозможно выяснить, что именно явилось причиной
их флуктуаций: объект исследования или внешние причины.
Таким образом, для исследования естественных случайных процессов необходимо
сперва отдельно исследовать и оценить случайные инструментальные погрешности
и убедиться, что они достаточно малы.

% \note{Зависимость случайной погрешности среднего от погрешности отдельного
% измерения как правило имеет вид $\sigma_N = \sigma_1 / \sqrt{N}$ и выводится в
% разделе \ref{sec:average}.
% }

\subsection{Систематические погрешности}

Систематические погрешности, в отличие от случайных, невозможно обнаружить,
исключить или уменьшить просто многократным повторением измерений.
Они могут быть обусловлены, во-первых, неправильной работой приборов
(\emph{инструментальная погрешность}), например, сдвигом нуля отсчёта
по шкале, деформацией шкалы, неправильной калибровкой, искажениями
из-за не нормативных условий эксплуатации, искажениями из-за износа
или деформации деталей прибора, изменением параметров прибора во времени
из-за нагрева и т.\,п. Во-вторых, их причиной может быть ошибка в интерпретации
результатов (\emph{методическая погрешность}), например, из-за использования
слишком идеализированной физической модели явления, которая не учитывает
некоторые значимые факторы (так, при взвешивании тел малой плотности
в атмосфере необходимо учитывать силу Архимеда; при измерениях в электрических
цепях может быть необходим учёт неидеальности амперметров и вольтметров
и т.\,д.).

Систематические погрешности условно можно разделить на следующие категории.
\begin{itemize}\small
    \item Известные погрешности, которые могут быть достаточно точно вычислены
или измерены. При необходимости они могут быть учтены непосредственно:
внесением поправок в расчётные формулы или в результаты измерений.
Если они малы, их можно отбросить, чтобы упростить вычисления.

    \item Погрешности известной природы, конкретная величина которых неизвестна,
но максимальное значение вносимой ошибки может быть оценено теоретически
или экспериментально. Такие погрешности неизбежно присутствуют в любом
опыте, и задача экспериментатора~--- свести их к минимуму,
совершенствуя методики измерения и выбирая более совершенные приборы.

    Чтобы оценить величину систематических погрешностей опыта, необходимо
учесть паспортную точность приборов (производитель, как правило, гарантирует,
что погрешность прибора не превосходит некоторой величины), проанализировать
особенности методики измерения, и по возможности, провести контрольные
опыты.

    \item Погрешности известной природы, оценка величины которых по каким-либо
причинам затруднена (например, сопротивление контактов при подключении
электронных приборов). Такие погрешности должны быть обязательно исключены
посредством модификации методики измерения или замены приборов.

    \item Наконец, нельзя забывать о возможности существования ошибок, о
которых мы не подозреваем, но которые могут существенно искажать результаты
измерений. Такие погрешности самые опасные, а исключить их можно только
многократной \emph{независимой} проверкой измерений, разными методами
и в разных условиях.
\end{itemize}

В учебном практикуме учёт систематических погрешностей ограничивается,
как правило, теоретическими поправками
к упрощённой модели исследуемого явления и паспортными погрешностями приборов.
Стоит однако подчеркнуть, что приборная погрешность вовсе не обязательно является 
исключительно систематической, и для большинства приборов имеет также 
и случайную составляющую.

В общем случае корректный учёт систематической погрешности возможен только 
при анализе специфики конкретного эксперимента. Особенное внимание следует 
обратить на возможную зависимость (корреляцию) систематических смещений различных величин 
при многократных измерениях. 
%Одна и та же погрешность в разных случаях может быть интерпретирована и как случайная, и как систематическая. // это уже сказано выше в замечании

% // Пример громоздкий, возможно, стоит отправить его в п. 4.4.

%\example{
%    Калибровка электромагнита производится при помощи внесения в него датчика Холла или другого измерителя магнитного потока. При последовательных измерениях с разными токами (и соотственно полями в зазоре) калибровку можно учитыать двумя различными способами:
%    \begin{itemize}
%        \item Измерить значение поля для разных токов, построить линейную калибровочную кривую и потом использовать значения, восстановленные по этой кривой для вычисления поля по току, используемому в измерениях.
%        \item Для каждого измерения проводить допольнительное измерения поля и вообще не испльзовать значения тока.
%    \end{itemize}
%
%    В первом случае погрешность полученного значения будет меньше, поскльку при проведении прямой, отдельные отклонения усреднятся. При этом погрешность измерения поля будет носить систематический харрактер и при обработке данных ее надо будет учитывать в последний момент. Во втором случае погрешность будет носить статистический (случайный) харрактер и ее надо будет добавить к погрешности каждой измеряемой точки. При этом сама погрешность будет больше. Выбор той или иной методики зависит от конретной ситуации. При большом количестве измерений, второй способ более надежный, поскольку статистическая ошибка при усреднении уменьшается пропорционально корню из количества измерений. Кроме того, такой способ повзоляет избежать методической ошибки, связанной с тем, что зависимость поля от тока не является линейной.
%}

\example{Рассмотрим измерение напряжения по стрелочному вольтметру. 
В~показаниях прибора будет присутствовать, как минимум, три вида погрешности:
1)~случайная погрешность, связанная с дрожанием стрелки и ошибкой визуального 
наблюдения, примерно равная половине цены деления шкалы;
2)~cистематическая погрешность, связанная с неправильной установкой нуля;
3)~систематическая погрешность, связанная с неправильным коэффициентом 
пропорциональности между напряжением и отклонением стрелки. Как правило, приборы 
сконструированы так, чтобы максимальное значение этой погрешности было также равно 
половине цены деления (хотя на практике это и не гарантируется).}


\chapter{Элементы теории ошибок}
\label{ch:prob}

% Не претендуя ни в коей мере на полноту или строгость, мы изложим наиболее
% базовые понятия и результаты теории вероятностей, без которых нельзя обойтись
% при описании погрешностей.

% Для работы со случайными величинами необходимо понимать, что они не являются
% полностью произвольными, а подчиняются некоторым законам. Эти законы называются
% распределениями.

Результат любого измерения не определён однозначно и имеет случайную составляющую.
Поэтому адекватным языком для описания погрешностей является язык вероятностей.
Тот факт, что значение некоторой величины \textquote{случайно}, не означает, что
она может принимать совершенно произвольные значения. Ясно, что частоты, с которыми
возникает те или иные значения, различны. Вероятностные законы, которым
подчиняются случайные величины, называют \emph{распределениями}.

\section{Случайная величина}

% 1)Это наглядное определение вероятности иногда называют «частотным» или «фи-
% зическим». Современная аксиоматическая теория вероятностей использует другое, бо-
% лее абстрактное определение, к которому однако нет смысла прибегать в рамках курса
% общей физики. То же касается и вводимых ниже определений среднего, случайной ве-
% личины и т. п. Наша цель — наглядное описание физики явлений, а не построение
% строгой формальной теории.
% 2)Иногда w (ξ) называют просто «функцией распределения». Мы будем избегать этой
% терминологии, поскольку в теории вероятностей «функцией распределения» называют
% другую характеристику случайной величины — вероятность того, что она не превосхо-
% дит некоторое значение. В англоязычной литературе различают соответственно partial
% distribution function (pdf) и cumulative distribution function (cdf).

% Определение понятия вероятности является сложной задачей без однозначно
% правильного решения. В зависимости от области применения, могут применяться
% различные, во многих случаях взаимоисключающие определения. При этом большинство
% определений приводят к одним и тем же практическим выводам, поэтому с прикладной
% точки зрения не так важно, какое из них выбрать.

% \disclaimer{
%     Конкретное определение вероятности не является существенным для практических
% применений, поэтому всем, кому не интересна идейная составляющая вопроса
% рекомендуется пропустить этот раздел и пользоваться "наивным" определением,
% основанном на "здравом смысле".
% }

% Как это ни парадоксально, формальная теория вероятности и математическая
% статистика никак не определяют понятие вероятности. Все теоретические выводы
% основываются на том, что изучаемые величины (вероятности) удовлетворяют
% некоторым формальным правилам (например системе аксиом Колмогорова
% \todo{ссылка}), но никаким образом не определяют как эти величины получить. Со
% времен создания теории вероятности существует два лагеря статистиков, каждый из
% которых использует свое определение. Обсуждение этих формальных моментов лежит
% далеко за пределами данного пособия, так что для упрощения понимания материала а
% также сокращения многих доказательств, в данном пособии мы будем придерживаться
% следующих определений:

Назовём \emph{вероятностью} $P_A$ некоторого события $A$
(например, результата измерения) \emph{долю случаев}, в которых реализуется
данный результат в пределе большого числа измерений $n$:
\[
P_A = \lim_{n\to \infty} \frac{n_A}{n},
\]
где $n_A$ --- количество наблюдений результата $A$.

\emph{Случайной} будем называть величину, значение которой не может быть
\emph{достоверно} определено экспериментатором с абсолютной точностью или
значение которой меняется в каждом последующем опыте неконтролируемым образом.
Согласно такому определению, \emph{любой} результат любых измерений является
случайным, поскольку осведомленность экспериментатора всегда чем-то ограничена.

\note{Хотя понятия вероятности и случайной величины являются основополагающими,
    в литературе нет единства в их определении. Обсуждение формальных тонкостей
    или построение строгой теории лежит за пределами данного пособия.
    Наша цель --- дать упрощенное, но наглядное представление о предмете.
Заинтересованным читателям рекомендуем обратиться к специальной литературе: \cite{idie}.}

% \subsection{Распределения}
% \emph{Распределением случайной величины} $x$ или \emph{функцией плотности
% вероятности} называют такую функцию $f(x)$, что вероятность для нее находиться в
% диапазоне от $x$ до $x+dx$ равна $p(x \in (x_0, x_0+dx)) = f(x_0) dx$. Очевидно
% \todo{не очевидно}, что такое определение годится только для численным величин и
% имеет ряд других ограничений, но в данном пособии мы ограничимся им. Более
% строгие определения можно почерпнуть из специальной литературы (\todo{ссылка}).
% \todo[inline,author=ppv]{Отослать один раз в начале главы к более строгим
% определениям в специальной литературе. Мы заведомо не претендуем на строгость
% или полноту.}

Рассмотрим случайную физическую величину $x$, которая при измерениях может
принимать \emph{непрерывный} набор значений. Пусть
$P_{[x_0,\,x_0+\delta x]}$~--- вероятность того, что результат окажется вблизи
некоторой точки $x_0$ в пределах интервала $\delta x$: $x\in [x_0,\,x_0+\delta x]$.
Устремим интервал
$\delta x$ к нулю. Нетрудно понять, что вероятность попасть в этот интервал
также будет стремиться к нулю. Однако отношение
$w(x_0) = \frac{P_{[x_0,\,x_0+\delta x]}}{\delta x}$ будет оставаться конечным.
Функцию $w(x)$ называют \emph{плотностью распределения вероятности} или кратко
\emph{распределением} непрерывной случайной величины $x$.

% Важно отметить, что в работах по статистике, написанных математиками,
% распределением часто называют не саму функцию $f$, а ее интеграл $F(x_0) =
% \int{f(x)dx}$, который можно определить, не используя бесконечно малые. Такую
% функцию физики называют \emph{интегральным распределением}.

\note{В математической литературе распределением часто называют не функцию
    $w(x)$, а её интеграл $W(x)=\int w(x)\,dx$. Такую функцию в физике принято
    называть \emph{интегральным} или \emph{кумулятивным} распределением. В англоязычной литературе
    для этих функций принято использовать сокращения:
    \emph{pdf} (\emph{probability distribution function}) и
    \emph{cdf} (\emph{cumulative distribution function})
    соответственно.}

\paragraph{Гистограммы.}
Проиллюстрируем наглядно понятие плотности распределение. Результат
большого числа измерений случайной величины удобно представить с помощью
специального типа графика --- \emph{гистограммы}.
Для этого область значений $x$, размещённую на оси абсцисс, разобьём на
равные малые интервалы --- \textquote{корзины>> или <<бины} (\emph{англ.} bins)
некоторого размера $h$. По оси ординат будет откладывать долю измерений $w$,
результаты которых попадают в соответствующую корзину. А именно,
пусть $k$ --- номер корзины; $n_k$ --- число измерений, попавших
в диапазон $x\in [kh,\,(k+1)h]$. Тогда  на графике изобразим \textquote{столбик}
шириной $h$ и высотой $w_{k}=n_{k}/n$.
В результате получим картину, подобную изображённой на рис.~\ref{fig:normhist}.

\begin{figure}[h!]
    \centering
    \includegraphics[width=7cm]{images/normhist.pdf}
    \caption{Пример гистограммы для нормального распределения ($\limaverage{x}=10$,
$\sigma=1{,}0$, $h=0{,}1$, $n=10^{4}$)}\label{fig:normhist}
\end{figure}

Высоты построенных столбиков будут приближённо соответствовать значению
плотности распределения $w(x)$ вблизи соответствующей точки $x$.
Если устремить число измерений к бесконечности ($n\to \infty$), а ширину корзин
к нулю ($h\to0$), то огибающая гистограммы будет стремиться к некоторой
непрерывной функции $w(x)$.

Самые высокие столбики гистограммы будут группироваться вблизи максимума
функции $w(x)$ --- это \emph{наиболее вероятное} значение случайной величины.
Если отклонения в положительную и отрицательную стороны равновероятны,
то гистограмма будет симметрична --- в таком случае среднее значение $\average{x}$
также будет лежать вблизи этого максимума. Ширина гистограммы будет характеризовать разброс
значений случайной величины --- по порядку величины
она, как правило, близка к среднеквадратичному отклонению $\sigma_x$.

\paragraph{Свойства распределений.}

Из определения функции $w(x)$ следует, что вероятность получить в результате
эксперимента величину $x$ в диапазоне от $a$ до $b$
% или вероятностное содержание интервала $(a,b)$
можно найти, вычислив интеграл:
\begin{equation}
    P_{x\in [a, b]}=\int\limits _{a}^{b}w\!\left(x\right)dx.\label{eq:P}
\end{equation}

Согласно определению вероятности, сумма вероятностей для всех возможных случаев
всегда равна единице. Поэтому интеграл распределения $w(x)$ по всей области
значений $x$ (то есть суммарная площадь под графиком $w(x)$) равен единице:
\[
\int\limits_{-\infty}^{+\infty} w(x)\,dx=1.
\]
Это соотношение называют \emph{условием нормировки}.

\paragraph{Среднее и дисперсия.}

Вычислим среднее по построенной гистограмме. Если размер корзин
$h$ достаточно мал, все измерения в пределах одной корзины можно считать примерно
одинаковыми. Тогда среднее арифметическое всех результатов можно вычислить как
\[
\average{x} \approx \frac{1}{n}\sum_i n_i x_i = \sum_i w_i x_i.
\]
Переходя к пределу, получим следующее определение среднего значения
случайной величины:
\begin{equation}
    \limaverage{x} = \int{x w\,dx},
\end{equation}
где интегрирование ведётся по всей области значений $x$.
В теории вероятностей \limaverage{x} также называют \emph{математическим ожиданием}
распределения.
Величину
\begin{equation}
    \sigma^2 = \limaverage{(x-\limaverage{x})^2}= \int{(x - \limaverage{x})^2 w\,dx}
\end{equation}
называют \emph{дисперсией} распределения. Значение $\sigma$ есть
срекднеквадратичное отклонение в пределе $n\to \infty$. Оно имеет ту
же размерность, что и сама величина $x$ и характеризует разброс распределения.
Именно эту величину, как правило, приводят как характеристику погрешности
измерения $x$.

\paragraph{Доверительный интервал.}
Обозначим как $P_{\left|\Delta x\right|<\delta}$ вероятность
того, что отклонение от среднего $\Delta x=x-\limaverage{x}$ составит величину,
не превосходящую по модулю значение $\delta$:
\begin{equation}\label{eq:confidenceP}
P_{\left|\Delta x\right|<\delta}=\int\limits
_{\limaverage{x}-\delta}^{\limaverage{x}+\delta}w\!\left(x\right)dx.
\end{equation}
Эту величину называют \emph{доверительной вероятностью} для
\emph{доверительного интервала} $\left|x-\limaverage{x}\right|\le\delta$.

% \todo{Что иллюстрирует пример?}
% \example{Пусть есть набор резисторов из одной серии с одной и той же маркировкой,
% соответствующей одному и тому же номиналу их сопротивления. В силу неидеальности
% процесса изготовления, точное значение резистора является случайной величиной,
% средней значение которое должно совпадать с заводским номиналом, но при этом
% будет наблюдаться также некоторый разброс значений. Рассмотрим два измерения: в
% первом будем просто брать все резисторы из одной серии, замерять их
% сопротивления и строить гистограмму результата измерений. В этом случае мы
% увидим картину похожую на рис.~\ref{fig:normhist}.\par
%     Теперь предположим, что перед тем как мы начали проводить свои измерения,
% кто-то взял и отобрал из изучаемой партии все резисторы с сопротивлением
% максимально приближенным к номинальному значению. В этом случае окажется, что в
% нашем измерении вероятность получить сопротивление, близкое к номиналу, будет
% мала. Как следствие, в результатем измерения будет получено \textquote{двух-горбое}
% распределение с провалом посередине. Среднее значение будет таким же, как и в
% первом случае, но разброс будет больше и наиболее вероятное значение (точнее два
% наиболее вероятных значения) не будут совпадать со средним.}


% Предположим, что систематические погрешности малы и займёмся отдельно
% изучением случайных погрешностей. Пусть по результатам многократных
% измерений получен набор значений $\left\{ x_{i}\right\} $, вычислено
% их среднее (\ref{eq:average}) $\average{x}$ и среднеквадратичное
% отклонение (\ref{eq:sigma}) $\sigma_{x}\approx s{}_{x}$. Можно надеяться,
% что измеряемая величина лежит в диапазоне
% \[
% x\in\left(\average{x}-\sigma_{x};\,\average{x}+\sigma_{x}\right).
% \]
% Какова вероятность $P$ того, что результат действительно находится
% в указанном интервале?

% Для ответа на этот вопрос необходимо знать \emph{вероятностный закон},
% которому подчиняется исследуемая величина. Казалось бы, для каждой
% случайной физической величины должен существовать свой особенный закон
% и общую теорию здесь построить невозможно. Это отчасти верно, но оказывается,
% что существует вполне \emph{универсальный} вероятностный закон, называемый
% \emph{нормальным}, которому подчиняются многие случайные величины.
% Рассмотрим его подробнее.

\section{Нормальное распределение}\label{sec:normal}

Одним из наиболее примечательных результатов теории вероятностей является
так называемая \emph{центральная предельная теорема}. Она утверждает,
что сумма большого количества независимых случайных слагаемых, каждое
из которых вносит в эту сумму относительно малый вклад, подчиняется
универсальному закону, не зависимо от того, каким вероятностным законам
подчиняются её составляющие, --- так называемому \emph{нормальному
распределению} (или \emph{распределению Гаусса}).

Доказательство теоремы довольно громоздко и мы его не приводим (его можно найти
в любом учебнике по теории вероятностей). Остановимся
кратко на том, что такое нормальное распределение и его основных свойствах.

Плотность нормального распределения выражается следующей формулой:
\begin{equation}
    \label{eq:normal}
    \boxed{
w_{\mathcal{N}}\!\left(x\right)=\frac{1}{\sqrt{2\pi}\sigma}e^{-\tfrac{(x-\limaverage{x})^
{2}}{2\sigma^{2}}}
    }.
\end{equation}
Здесь $\limaverage{x}$ и $\sigma$
--- параметры нормального распределения: $\limaverage{x}$ равно
среднему значению $x$, a $\sigma$ ---
среднеквадратичному отклонению, вычисленным в пределе $n\to\infty$.

Как видно из рис.~\ref{fig:normhist}, распределение представляет собой
симметричный
\textquote{колокол}, положение вершины которого
соответствует $\bar{x}$ (ввиду симметрии оно же
совпадает с наиболее вероятным значением --- максимумом
функции $w_{\mathcal{N}}(x)$).

При значительном отклонении $x$ от среднего величина
$w_{\mathcal{N}}\!\left(x\right)$
очень быстро убывает. Это означает, что вероятность встретить отклонения,
существенно большие, чем $\sigma$, оказывается \emph{пренебрежимо
мала}. Ширина \textquote{колокола} по порядку величины
равна $\sigma$ --- она характеризует \textquote{разброс}
экспериментальных данных относительно среднего значения.

\note{Точки $x=\bar{x}\pm\sigma$ являются точками
    перегиба графика $w\left(x\right)$ (в них вторая производная по $x$
    обращается в нуль, $w''=0$), а их положение по высоте составляет
    $w\!\left(\bar{x}\pm\sigma\right)/w(\bar{x})=e^{_{-1/2}}\approx0{,}61$
    от высоты вершины.}

Универсальный характер центральной предельной теоремы позволяет широко
применять на практике нормальное (гауссово) распределение для обработки
результатов измерений, поскольку часто случайные погрешности складываются из
множества случайных \emph{независимых} факторов. Заметим, что на практике
для \emph{приближённой оценки} параметров нормального распределения
случайной величины используются \emph{выборочные} значения среднего
и дисперсии: $\limaverage{x}\approx\average{x}$, $s_{x}\approx\sigma_{x}$.

\begin{figure}[h]
    \centering
    \input{images/gauss.pdf_t}
    \caption{Плотность нормального распределения}
\end{figure}

\paragraph{Доверительные вероятности.}
Вычислим некоторые доверительные вероятности \eqref{eq:confidenceP} для нормально
распределённых случайных величин.
\note{Значение интеграла вида $\int e^{-x^{2}/2}dx$
(его называют \emph{интегралом ошибок}) в элементарных функциях не выражается,
но легко находится численно.}
Вероятность того, что результат отдельного измерения $x$ окажется
в пределах $\limaverage{x}\pm\sigma$ оказывается равна
\[
P_{\left|\Delta x\right|<\sigma} =
\int\limits_{\limaverage{x}-\sigma}^{\limaverage{x}+\sigma}
w_{\mathcal{N}} dx \approx0{,}68.
\]
Вероятность отклонения в пределах $\limaverage{x}\pm2\sigma$:
\[
P_{\left|\Delta x\right|<2\sigma}\approx0{,}95,
\]
а в пределах $\limaverage{x}\pm3\sigma$:
\[
P_{\left|\Delta x\right|<3\sigma}\approx0{,}9973.
\]
Иными словами, при большом числе измерений нормально распределённой
величины можно ожидать, что лишь треть измерений выпадут за пределы интервала
$\left[\bar{x}-\sigma,\,\bar{x}+\sigma\right]$. При этом около 5\%
измерений выпадут за пределы $\left[\bar{x}-2\sigma;\bar{x}+2\sigma\right]$,
и лишь 0,27\% окажутся за пределами
$\left[\bar{x}-3\sigma;\bar{x}+3\sigma\right]$.

\example{В сообщениях об открытии бозона Хиггса на Большом адронном коллайдере
говорилось о том, что исследователи ждали подтверждение результатов
с точностью \textquote{5 сигма}. Используя нормальное распределение \eqref{eq:normal}
нетрудно посчитать, что они использовали доверительную вероятность
$P\approx1-5{,}7\cdot10^{-7}=0{,}99999943$. Такую точность можно назвать фантастической.}

Полученные значения доверительных вероятностей используются при
\emph{стандартной записи результатов измерений}. В физических измерениях
(в частности, в учебной лаборатории), как правило, используется $P=0{,}68$,
то есть, запись
\[
x=\bar{x}\pm\delta x
\]
означает, что измеренное значение лежит в диапазоне (доверительном
интервале) $x\in\left[\bar{x}-\delta x;\bar{x}+\delta x\right]$ с
вероятностью 68\%. Таким образом погрешность $\pm\delta x$ считается
равной одному среднеквадратичному отклонению: $\delta x=\sigma$.
В \emph{технических} измерениях чаще используется $P=0{,}95$, то есть под
абсолютной погрешностью имеется в виду удвоенное среднеквадратичное
отклонение, $\delta x=2\sigma$. Во избежание разночтений доверительную
вероятность следует указывать отдельно.

\note{Хотя нормальный закон распределения встречается на практике довольно
    часто, стоит помнить, что он реализуется \emph{далеко не всегда}.
    Полученные выше соотношения для вероятностей попадания значений в
    доверительные интервалы можно использовать в качестве простейшего
    признака нормальности распределения: в частности, если количество попадающих
    в интервал $\pm \sigma$ результатов существенно отличается от 2/3 --- это повод
    для более детального исследования закона распределения ошибок.}


\paragraph{Сравнение результатов измерений.}
Теперь мы можем дать количественный критерий для сравнения двух измеренных
величин или двух результатов измерения одной и той же величины.

Пусть $x_{1}$ и $x_{2}$ ($x_{1}\ne x_{2}$) измерены с
погрешностями $\sigma_{1}$ и $\sigma_{2}$ соответственно.
Ясно, что если различие результатов $|x_2-x_1|$ невелико,
его можно объяснить просто случайными отклонениями.
Если же теория предсказывает, что вероятность обнаружить такое отклонение
слишком мала, различие результатов следует признать \emph{значимым}.
Предварительно необходимо договориться о соответствующем граничном значении
вероятности. Универсального значения здесь быть не может,
поэтому приходится полагаться на субъективный выбор исследователя. Часто
в качестве \textquote{разумной} границы выбирают вероятность 5\%,
что, как видно из изложенного выше, для нормального распределения
соответствует отклонению более, чем на $2\sigma$.

Допустим, одна из величин известна с существенно большей точностью:
$\sigma_{2}\ll\sigma_{1}$ (например, $x_{1}$ --- результат, полученный
студентом в лаборатории, $x_{2}$ --- справочное значение).
Поскольку $\sigma_{2}$ мало, $x_{2}$ можно принять за \textquote{истинное}:
$x_{2}\approx \limaverage{x}$. Предполагая, что погрешность измерения
$x_{1}$ подчиняется нормальному закону с и дисперсией $\sigma_{1}^{2}$,
можно утверждать, что
% можно с помощью функции (\ref{eq:normal}) вычислить вероятность
% того, что отклонение $\left|x_{1}-x_{2}\right|$ возникло исключительно
% в силу случайных причин.
% То есть
различие считают будет значимы, если
\[
\left|x_{1}-x_{2}\right|>2\sigma_{1}.
\]

Пусть погрешности измерений сравнимы по порядку величины:
$\sigma_{1}\sim\sigma_{2}$. В теории вероятностей показывается, что
линейная комбинация нормально распределённых величин также имеет нормальное
распределение с дисперсией $\sigma^{2}=\sigma_{1}^{2}+\sigma_{2}^{2}$
(см. также правила сложения погрешностей (\ref{eq:sigma_sum})). Тогда
для проверки гипотезы о том, что $x_{1}$ и $x_{2}$ являются измерениями
одной и той же величины, нужно вычислить, является ли значимым отклонение
$\left|x_{1}-x_{2}\right|$ от нуля при $\sigma=\sqrt{\sigma_{1}^{2}+\sigma_{2}^{2}}$.

\example{Два студента получили следующие значения для теплоты испарения
    некоторой жидкости: $x_{1}=40{,}3\pm0{,}2$~кДж/моль и
    $x_{2}=41{,}0\pm0{,}3$~кДж/моль, где погрешность соответствует
    одному стандартному отклонению. Можно ли утверждать, что они исследовали
    одну и ту же жидкость?\par
Имеем наблюдаемую разность $\left|x_{1}-x_{2}\right|=0{,}7$~кДж/моль,
среднеквадратичное отклонение для разности
$\sigma=\sqrt{0{,}2^{2}+0{,}3^{2}}=0{,}36$~кДж/моль.
Их отношение $\frac{\left|x_{2}-x_{1}\right|}{\sigma}\approx2$. Из
свойств нормального распределения находим вероятность того, что измерялась
одна и та же величина, а различия в ответах возникли из-за случайных
ошибок: $P\approx5\%$. Ответ на вопрос, \textquote{достаточно}
ли мала или велика эта вероятность, остаётся на усмотрение исследователя.}

\note{Изложенные здесь соображения применимы, только если $\limaverage{x}$ и
его стандартное отклонение $\sigma$ получены на основании достаточно
большой выборки $n\gg1$ (или заданы точно). При небольшом числе измерений
($n\lesssim10$) выборочные средние $\average{x}$ и среднеквадратичное отклонение
$s_x$ сами имеют довольно большую ошибку, а
их распределение будет описываться не нормальным законом, а так
называемым $t$-распределением Стъюдента. В частности, в зависимости от
значения $n$ интервал $\average{x}\pm s_{x}$ будет соответствовать несколько
меньшей доверительной вероятности, чем $P=0{,}68$. Особенно резко различия
проявляются при высоких уровнях доверительных вероятностей $P\to1$.
Подробнее об этом см. {[}?{]}.}

% \section{Распределение Пуассона}
% \disclaimer{
%     Распределение Пуассона применяется в случаях, когда имеет место измерения
% количества событий, произошедших за определенный интервал времени или в
% определенном объеме. Понимание этого распределения необходимо для студентов 5
% семестра, изучающих основы физички частиц. Остальные студенты могут пропустить
% этот раздел.
% }
%
% \todo[inline, color = red]{TODO Пуассон}

\section{Независимые величины. Корреляция}
\todo[inline, author=Nozik]{Эта глава вообще зачем нужна?}

Рассмотрим две физические величины $x$ и $y$. Величины называют
\emph{независимыми} если результат измерения одной из них никак не
влияет на результат измерения другой.

Обозначим отклонения от средних как $\Delta x=x-\average{x}$ и $\Delta
y=y-\average{y}$.
Средние значения отклонений равны, очевидно, нулю: $\average{\Delta
x}=x-\average{x}=0$,
$\average{\Delta y}=0$. Из независимости величин $x$ и $y$ следует,
что среднее значение от произведения $\average{\Delta x\cdot\Delta y}$
равно произведению средних $\average{\Delta x}\cdot\average{\Delta y}$
и, следовательно, равно нулю:
\begin{equation}
\average{\Delta x\cdot\Delta y}=\average{\Delta x}\cdot\average{\Delta
y}=0.\label{eq:indep}
\end{equation}

Если $x$ и $y$ не являются независимыми, среднее значение произведения
их отклонений может быть использовано как количественная мера их зависимости.
Наиболее употребительной мерой зависимости двух случайных величин
является \emph{коэффициент линейной корреляции}:
\begin{equation}
r_{xy}=\frac{\average{\Delta x\cdot\Delta
y}}{\sigma_{x}\cdot\sigma_{y}}.\label{eq:pearson}
\end{equation}
Нетрудно проверить (с помощью неравенства Коши\textendash Буняковского),
что $-1\le r\le1$. В частности, для полностью независимых величин
коэффициент корреляции равен нулю, $r=0$, а для линейно зависимых
$y=kx+b$ нетрудно получить $r=1$ при $k>0$ и $r=-1$ при $k<0$.
Примеры промежуточных случаев представлены на рис. TODO.

Если коэффициент $r_{xy}$ близок к единице, говорят, что величины
\emph{коррелируют} между собой (от \emph{англ.} correlate ---
находиться в связи).

\paragraph{Отсутствие корреляции $\protect\not\Rightarrow$ независимость.}

Отметим, что (\ref{eq:indep}) --- необходимое,
но не достаточное условие независимости величин. На рис. TODO приведён
пример очевидно зависимых $x$ и $y$, для которых $r\approx0$.

\paragraph{Корреляция $\protect\not\Rightarrow$ причинность.}

Ещё одна типичная ошибка --- исходя из большого
коэффициента корреляции ($r\to1$) между двумя величинами сделать
вывод о функциональной (причинной) связи между $x$ и $y$. Рассмотрим
конкретный пример. Между током и напряжением на некотором резисторе
имеет место линейная зависимость $U=IR$, и коэффициент корреляции
$r_{UI}$ действительно равен единице. Однако \emph{обратное
в общем случае неверно}. Например, ток в резисторе коррелирует
с его температурой $T$, $r_{IT}\to1$ (больше ток --- больше
тепловыделение по закону Джоуля\textendash Ленца), однако ясно, что
нагрев резистора извне не приведёт к повышению тока в нём (скорее
наоборот, так как сопротивление металлов с температурой растёт). Ошибка
отождествления корреляции и причинности особенно характерна при исследовании
сложных многофакторных систем, например, в медицине, социологии и
т.п.

\section{Дисперсия суммы}\label{sec:sum2}

Пусть измеряемая величина $z=x+y$ складывается из двух \emph{независимы}х
случайных слагаемых $x$ и $y$, для которых известны средние значения
$\average{x}$ и $\average{y}$, и их среднеквадратичные погрешности
$\sigma_{x}$ и $\sigma_{y}$. Непосредственно из определения (\ref{eq:average})
следует вполне очевидный результат, что среднее суммы равно сумме
средних:
\[
    \average{z}=\average{x}+\average{y}.
\]

Найдём дисперсию $\sigma_{z}^{2}$. В силу независимости имеем
\[
    \average{\Delta z^{2}}=\average{\Delta x^{2}}+\average{\Delta
    y^{2}}+2\average{\Delta x\cdot\Delta y}=\average{\Delta x^{2}}+\average{\Delta
    y^{2}},
\]
то есть:
\begin{equation}
    \label{eq:sigma_sum}
    \boxed{{\sigma_{x+y}=\sqrt{\sigma_{x}^{2}+\sigma_{y}^{2}}}}.
\end{equation}
Таким образом, при сложении \emph{независимых }величин их погрешности
складываются среднеквадратичным образом.

Подчеркнём, что для справедливости соотношения (\ref{eq:sigma_sum})
величины $x$ и $y$ не обязаны быть нормально распределёнными ---
достаточно существования конечных значений их дисперсий (однако можно
показать, что если $x$ и $y$ распределены нормально, нормальным
будет и распределение их суммы).

\note{Требование независимости
слагаемых является принципиальным. Например, положим $y=x$. Тогда
$z=2x$. Здесь $y$ и $x$, очевидно, зависят друг от друга. Используя
(\ref{eq:sigma_sum}), находим $\sigma_{2x}=\sqrt{2}\sigma_{x}$,
что, конечно, неверно --- непосредственно из определения
следует, что $\sigma_{2x}=2\sigma_{x}$.}

Отдельно стоит обсудить математическую структуру формулы (\ref{eq:sigma_sum}).
Если если одна из погрешностей много больше другой, например,
$\sigma_{x}\gg\sigma_{y}$,
то меньшей погрешностью можно пренебречь, $\sigma_{x+y}\approx\sigma_{x}$.
С другой стороны, если два источника погрешностей имеют один порядок
$\sigma_{x}\sim\sigma_{y}$, то и $\sigma_{x+y}\sim\sigma_{x}\sim\sigma_{y}$.

Эти обстоятельства важны при планирования эксперимента: как правило,
величина, измеренная наименее точно, вносит наибольший вклад в погрешность
конечного результата. При этом, пока не устранены наиболее существенные
ошибок, бессмысленно гнаться за повышением точности измерения остальных
величин.

\example{Пусть $\sigma_{y}=\sigma_{x}/3$,
тогда $\sigma_{z}=\sigma_{x}\sqrt{1+\frac{1}{9}}\approx1{,}05\sigma_{x}$,
то есть при различии двух погрешностей более, чем в 3 раза, поправка
к погрешности составляет менее 5\%, и уже нет особого смысла в учёте
меньшей погрешности: $\sigma_{z}\approx\sigma_{x}$. Это утверждение
касается сложения любых независимых источников погрешностей в эксперименте.}

\section{Погрешность среднего}\label{sec:average}

Выборочное среднее арифметическое значение $\average{x}$, найденное
по результатам $n$ измерений, само является случайной величиной.
Действительно, если поставить серию одинаковых опытов по $n$ измерений,
то в каждом опыте получится своё среднее значение, отличающееся от
предельного среднего $\limaverage{x}$.

Вычислим среднеквадратичную погрешность среднего арифметического
$\sigma_{\average{x}}$.
Рассмотрим вспомогательную сумму $n$ слагаемых
\[
    Z=x_{1}+x_{2}+\ldots+x_{n}.
\]
Если $\left\{ x_{i}\right\} $ есть набор \emph{независимых} измерений
\emph{одной и той же} физической величины, то мы можем, применяя результат
(\ref{eq:sigma_sum}) предыдущего параграфа, записать
\[
    \sigma_{Z}=\sqrt{\sigma_{x_{1}}^{2}+\sigma_{x_{2}}^{2}+\ldots+\sigma_{x_{n}}^{2}
    }=\sqrt{n}\sigma_{x},
\]
поскольку под корнем находится $n$ одинаковых слагаемых. Отсюда с
учётом $\average{x}=Z/n$ получаем
\begin{equation}
\boxed{{\sigma_{\average{x}}=\frac{\sigma_{x}}{\sqrt{n}}}}.\label{eq:sigma_avg}
\end{equation}

Таким образом, \emph{погрешность среднего значения $x$ по результатам
$n$ независимых измерений оказывается в $\sqrt{n}$ раз меньше погрешности
отдельного измерения}. Это один из важнейших результатов, позволяющий
уменьшать случайные погрешности эксперимента за счёт многократного
повторения измерений.

Подчеркнём отличия между $\sigma_{x}$ и $\sigma_{\average{x}}$:

величина $\sigma_{x}$ --- \emph{погрешность отдельного
измерения} --- является характеристикой разброса значений
в совокупности измерений $\left\{ x_{i}\right\} $, $i=1..n$. При
нормальном законе распределения примерно 68\% измерений попадают в
интервал $\average{x}\pm\sigma_{x}$;

величина $\sigma_{\average{x}}$ --- \emph{погрешность
среднего} --- характеризует точность, с которой определено
среднее значение измеряемой физической величины $\average{x}$ относительно
предельного (\textquote{истинного}) среднего $\limaverage{x}$;
при этом с доверительной вероятностью $P=68\%$ искомая величина $\limaverage{x}$
лежит в интервале
$\average{x}-\sigma_{\average{x}}<\limaverage{x}<\average{x}+\sigma_{\average{x}}$.

% \section{Погрешность погрешности}
% \todo[inline, author = Nozik]{Вот это материал повышенной сложности. Не
% уверен, что он вообще нужен и уж точно не в середине главы}
% \todo[inline,author = ppv]{Сложности тут особой не вижу, но материал
% действительно побочный. Но полезный -- ведь это обоснование того, как нужно
% округлять погрешность.}

% С какой точностью можно вычислить величину $\sigma$ по ограниченному
% количеству $n$ измерений? Оценим среднеквадратичное отклонения от
% своего среднего значения для величины $s$, вычисляемой по формуле
% (\ref{eq:sigma_straight}). Её квадрат $s^{2}$, как нетрудно видеть,
% состоит из $n$ примерно одинаковых слагаемых (обозначим их как
% $\xi_{i}=\left(x_{i}-\average{x}\right)^{2}$):
% \[
% s^{2}=\frac{1}{n-1}\sum_{i}\xi_{i}.
% \]
% Тогда, повторяя рассуждения п. \ref{subsec:average}, приходим к выводу,
% что погрешность вычисления $s^{2}$ пропорциональна корню из числа
% входящих в неё слагаемых (точнее, нужно использовать число \emph{независимых}
% слагаемых, равное $n-1$, как показано выше):
% \[
% \sigma_{s^{2}}\approx\sqrt{n}\cdot\frac{\sigma_{\xi}}{n-1}\approx
% \frac{\sigma_{\xi}}{\sqrt{n-1}}.
% \]
% С учётом того, что $\average{\xi}\approx s^{2}\approx\sigma_{x}^{2}$,
% величину $\sigma_{\xi}=\sqrt{\average{\left(\xi-\average{\xi}\right)^{2}}}$
% можно по порядку величины оценить как
% $\sigma_{\xi}\sim\average{\xi}\sim\sigma_{x}^{2}$;
% точный расчёт с использованием распределения Гаусса (\ref{eq:normal})
% даёт $\sigma_{\xi}=\sqrt{2}\sigma_{x}\approx\sqrt{2}s.$ Наконец,
% из соотношения $\sigma_{s^{2}}=2s\sigma_{s}$ (см. формулу (\ref{eq:sxy})
% ниже), окончательно получаем
% \begin{equation}
% \sigma_{s}=\frac{s}{\sqrt{2\left(n-1\right)}}.\label{eq:sigma_sigma}
% \end{equation}
% Более подробный и аккуратный вывод можно найти, например в {[}?{]}.

% Главный вывод, который можно сделать на основании результата
% (\ref{eq:sigma_sigma})
% --- ошибка вычисления стандартного отклонения, как правило,
% довольно велика. Например, при $n=6$ её относительная величина составляет
% $\approx$30\%, и даже при $n=50$ она уменьшается лишь до $10\%$.
% По этой причине величину погрешности имеет смысл \emph{округлять до
% 1\textendash 2 значащих цифр} (см. также п.~\ref{subsec:round}).

\section{Результирующая погрешность опыта}

Пусть для некоторого результата измерения известна оценка его максимальной
систематической погрешности $\Delta_{\text{сист}}$ и случайная
среднеквадратичная
погрешность $\sigma_{\text{случ}}$. Какова \textquote{полная}
погрешность измерения?

Предположим для простоты, что измеряемая величина \emph{в принципе}
может быть определена сколь угодно точно, так что можно говорить о
некотором её \textquote{истинном} значении $x_{\text{ист}}$
(иными словами, погрешность результата связана в основном именно с
процессом измерения). Назовём \emph{полной погрешностью} измерения
среднеквадратичное значения отклонения от результата измерения от
\textquote{истинного}:
\[
\sigma_{\text{полн}}^{2}=\average{\left(x-x_{\text{ист}}\right)^{2}}.
\]
Отклонение $x-x_{\text{ист}}$ можно представить как сумму случайного
отклонения от среднего $\delta x_{\text{случ}}=x-\limaverage{x}$
и постоянной (но, вообще говоря, неизвестной) систематической составляющей
$\delta x_{\text{сист}}= \limaverage{x} - x_{\text{ист}} = \mathrm{const}$:
\[
x-x_{\text{ист}}=\delta x_{\text{сист}}+\delta x_{\text{случ}}.
\]
Причём случайную составляющую можно считать независимой от систематической.
В таком случае из \eqref{eq:sigma_sum} находим:
\begin{equation}
\sigma_{\text{полн}}^{2}=\average{\delta x_{\text{сист}}^{2}}+\average{\delta
x_{\text{случ}}^{2}}\le\Delta_{\text{сист}}^{2}+\sigma_{\text{случ}}^{2}.
\label{eq:syst_full}
\end{equation}
Таким образом, для получения \emph{максимального} значения полной
погрешности некоторого измерения нужно квадратично сложить максимальную
систематическую и случайную погрешности.

Если измерения проводятся многократно, то согласно (\ref{eq:sigma_avg})
случайная составляющая погрешности может быть уменьшена, а систематическая
составляющая при этом остаётся неизменной:
\[
\sigma_{\text{полн}}^{2}\le\Delta_{\text{сист}}^{2}+\frac{\sigma_{x}^{2}}{n}.
\]

Отсюда следует важное практическое правило
(см. также обсуждение в п.~\ref{sec:sum2}): если случайная погрешность измерений
в 2--3 раза меньше предполагаемой систематической, то
\emph{нет смысла проводить многократные измерения} в попытке уменьшить погрешность
всего эксперимента. В такой ситуации измерения достаточно повторить
2--3 раза --- чтобы убедиться в повторяемости результата, исключить промахи
и проверить, что случайная ошибка действительно мала.
В противном случае повторение измерений может иметь смысл до
тех пор, пока погрешность среднего
$\sigma_{\average{x}}=\frac{\sigma_{x}}{\sqrt{n}}$
не станет меньше систематической.

\note{Поскольку конкретная
величина систематической погрешности, как правило, не известна, её
можно в некотором смысле рассматривать наравне со случайной ---
предположить, что её величина была определена по некоторому случайному
закону перед началом измерений (например, при изготовлении линейки
на заводе произошло некоторое случайное искажение шкалы). При такой
трактовке формулу \eqref{eq:syst_full} можно рассматривать просто
как частный случай формулы сложения погрешностей независимых величин
\eqref{eq:sigma_sum}.\par
Подчеркнем, что вероятностный закон, которому подчиняется
систематическая ошибка, зачастую неизвестен. Поэтому неизвестно и
распределение итогового результата. Из этого, в частности, следует,
что мы не может приписать интервалу $x\pm\Delta_{\text{сист}}$ какую-либо
определённую доверительную вероятность --- она равна 0,68
только если систематическая ошибка имеет нормальное распределение.
Можно, конечно, \emph{предположить},
--- и так часто делают --- что, к примеру, ошибки
при изготовлении линеек на заводе имеют гауссов характер. Также часто
предполагают, что систематическая ошибка имеет \emph{равномерное}
распределение (то есть \textquote{истинное} значение может с равной вероятностью
принять любое значение в пределах интервала $\pm\Delta_{\text{сист}}$).
Строго говоря, для этих предположений нет достаточных оснований.}

\example{В результате измерения диаметра проволоки микрометрическим винтом,
имеющим цену деления $h=0,01$ мм, получен следующий набор из $n=8$ значений:\par
{\footnotesize
\begin{tabular}{|c|c|c|c|c|c|c|c|c|}
\hline
$d$, мм & 0,39 & 0,38 & 0,39 & 0,37 & 0,40 & 0,39 & 0,38 & 0,39 \\ \hline
\end{tabular}\par}
\smallskip
Вычисляем среднее значение: $\average{d}\approx386{,}3$~мкм.
Среднеквадратичное отклонение:
% вычисляем по формуле (\ref{eq:sigma_straight}):
$\sigma_{d}\approx9{,}2$~мкм. Случайная погрешность среднего согласно
(\ref{eq:sigma_avg}):
$\sigma_{\average{d}}=\frac{\sigma_{d}}{\sqrt{8}}\approx3{,}2$
мкм. Все результаты лежат в пределах $\pm2\sigma_{d}$, поэтому нет
причин сомневаться в нормальности распределения. Максимальную погрешность
микрометра оценим как половину цены деления, $\Delta=\frac{h}{2}=5$~мкм.
Результирующая полная погрешность
$\sigma\le\sqrt{\Delta^{2}+\frac{\sigma_{d}^{2}}{8}}\approx6{,}0$~мкм.
Видно, что $\sigma_{\text{случ}}\approx\Delta_{\text{сист}}$ и проводить дополнительные измерения
особого смысла нет. Окончательно результат измерений может быть представлен
в виде (см. также \emph{правила округления}
результатов измерений в п.~\ref{subsec:round})
\[
d=386\pm6\;\text{мкм},\qquad\varepsilon_{d}=1{,}5\%.
\]

Заметим, что поскольку случайная погрешность и погрешность
прибора здесь имеют один порядок величины, наблюдаемый случайный разброс
данных может быть связан как с неоднородностью сечения проволоки,
так и с дефектами микрометра (например, с неровностями зажимов, люфтом
винта, сухим трением, деформацией проволоки под действием микрометра
и т.\,п.). Для ответа на вопрос, что именно вызвало разброс, требуются
дополнительные исследования, желательно с использование более точных
приборов.\par
}%\footnotesize

\example{Измерение скорости
полёта пули было осуществлено с погрешностью $\delta v=\pm1$ м/c.
Результаты измерений для $n=6$ выстрелов представлены в таблице:\par
{\footnotesize
\begin{tabular}{|c|c|c|c|c|c|c|}
\hline
$v$, м/с & 146 & 170 & 160 & 181 & 147 & 168 \\ \hline
\end{tabular}\par}
\smallskip
Усреднённый результат $\average{v}=162{,}0\;\text{м/с}$,
среднеквадратичное отклонение $\sigma_{v}=13{,}8\;\text{м/c}$, случайная
ошибка для средней скорости
$\sigma_{\bar{v}}=\sigma_{v}/\sqrt{6}=5{,}6\;\text{м/с}$.
Поскольку разброс экспериментальных данных существенно превышает погрешность
каждого измерения, $\sigma_{v}\gg\delta v$, он почти наверняка связан
с реальным различием скоростей пули в разных выстрелах, а не с ошибками
измерений. В качестве результата эксперимента представляют интерес
как среднее значение скоростей $\average{v}=162\pm6\;\text{м/с}$
($\varepsilon\approx4\%$), так и значение $\sigma_{v}\approx14\;\text{м/с}$,
характеризующее разброс значений скоростей от выстрела к выстрелу.
Малая инструментальная погрешность в принципе позволяет более точно
измерить среднее и дисперсию, и исследовать закон распределения выстрелов
по скоростям более детально --- для этого требуется набрать
б\'{о}льшую статистику по выстрелам.\par
}%\footnotesize

\example{Измерение скорости
полёта пули было осуществлено с погрешностью $\delta v=10$ м/c. Результаты
измерений для $n=6$ выстрелов представлены в таблице:\par
{\footnotesize
\begin{tabular}{|c|c|c|c|c|c|c|}
\hline
$v$, м/с & 150 & 170 & 160 & 180 & 150 & 170 \\ \hline
\end{tabular}\par}
\smallskip
Усреднённый результат $\average{v}=163{,}3\;\text{м/с}$,
$\sigma_{v}=12{,}1\;\text{м/c}$, $\sigma_{\average{v}}=5\;\text{м/с}$,
$\sigma_{\text{полн}}\approx11{,}2\;\text{м/с}$. Инструментальная
погрешность каждого измерения превышает разброс данных, поэтому в
этом опыте затруднительно сделать вывод о различии скоростей от выстрела
к выстрелу. Результат измерений скорости пули:
$\average{v}=163\pm11\;\text{м/с}$,
$\varepsilon\approx7\%$. Проводить дополнительные выстрелы при такой
большой инструментальной погрешности особого смысла нет ---
лучше поработать над точностью приборов и методикой измерений.\par
}%\footnotesize

\section{Обработка косвенных измерений\label{sec:kosv}}

\emph{Косвенными} называют измерения, полученные в результате расчётов,
использующих результаты \emph{прямых} (то есть \textquote{непосредственных})
измерений физических величин. Сформулируем основные правила пересчёта
погрешностей при косвенных измерениях.

\subsection{Случай одной переменной}

Пусть в эксперименте измеряется величина $x$, а её \textquote{наилучшее}
(в некотором смысле) значение равно $x^{\star}$ и оно известно с
погрешностью $\sigma_{x}$. После чего с помощью известной функции
вычисляется величина $y=f\!\left(x\right)$.

В качестве \textquote{наилучшего} приближения для $y$ используем значение функции
при \textquote{наилучшем} $x$:
\[
y^{\star}=f\!\left(x^{\star}\right).
\]

Найдём величину погрешности $\sigma_{y}$. Обозначая отклонение измеряемой
величины как $\Delta x=x-x^{\star}$, и пользуясь определением производной,
при условии, что функция $y\left(x\right)$ --- гладкая
вблизи $x\approx x^{\star}$, запишем
\[
\Delta y\equiv y\left(x\right)-y\left(x^{\star}\right)\approx f'\cdot\Delta x,
\]
где $f'\equiv\frac{dy}{dx}$ --- производная фукнции $f(x)$, взятая в точке
$x^{\star}$. Возведём полученное в квадрат, проведём усреднение
($\sigma_{y}^{2}=\average{\Delta y^{2}}$,
$\sigma_{x}^{2}=\average{\Delta x^{2}}$), и затем снова извлечём
корень. В результате получим
\begin{equation}
\boxed{{\sigma_{y}=\left|\frac{dy}{dx}\right|\sigma_{x}.}}\label{eq:sxy}
\end{equation}

\example{Для степенной функции
$y=Ax^{n}$ имеем $\sigma_{y}=nAx^{n-1}\sigma_{x}$, откуда
\[
\frac{\sigma_{y}}{y}=n\frac{\sigma_{x}}{x},\qquad\text{или}\qquad\varepsilon_{y}
=n\varepsilon_{x},
\]
то есть относительная погрешность степенной функции возрастает пропорционально
показателю степени $n$.\par
}%\footnotesize

\example{Для $y=1/x$ имеем $\varepsilon_{1/x}=\varepsilon_{x}$
--- при обращении величины сохраняется её относительная
погрешность.\par
}%\footnotesize

\exercise{Найдите погрешность логарифма $y=\ln x$, если известны~$x$
    и~$\sigma_{x}$.}

\exercise{Найдите погрешность показательной функции $y=a^{x}$,
    если известны~$x$ и~$\sigma_{x}$. Коэффициент $a$ задан точно.}


\subsection{Случай многих переменных}

Пусть величина $u$ вычисляется по измеренным значениям нескольких
различных \emph{независимых} физических величин $x$, $y$, $\ldots$
на основе известного закона $u=f\!\left(x,y,\ldots\right)$. В качестве
наилучшего значения можно по-прежнему взять значение функции $f$
при наилучших значениях измеряемых параметров:
\[
u^{\star}=f\!\left(x^{\star},y^{\star},\ldots\right).
\]

Для нахождения погрешности $\sigma_{u}$ воспользуемся свойством,
известным из математического анализа, --- малые приращения гладких
функции многих переменных складываются линейно, то есть справедлив
\emph{принцип суперпозиции} малых приращений:
\[
\Delta u\approx f'_{x}\cdot\Delta x+f'_{y}\cdot\Delta y+\ldots,
\]
где символом $f'_{x}\equiv\frac{\partial f}{\partial x}$ обозначена
\emph{частная производная} функции $f$ по переменной $x$ ---
то есть обычная производная $f$ по $x$, взятая при условии, что
все остальные аргументы (кроме $x$) считаются постоянными параметрами.
Тогда пользуясь формулой для нахождения дисперсии суммы независимых
величин (\ref{eq:sigma_sum}), получим соотношение, позволяющее вычислять
погрешности косвенных измерений для произвольной функции
$u=f\left(x,y,\ldots\right)$:
\begin{equation}
\boxed{\sigma_{u}^{2}=f_{x}^{\prime2}\,\sigma_{x}^{2}+f_{y}^{\prime2}\,\sigma_{y
}^{2}+\ldots}\label{eq:sigma_general}
\end{equation}
Это и есть искомая общая формула пересчёта погрешностей при косвенных
измерениях.

Отметим, что формулы (\ref{eq:sxy}) и (\ref{eq:sigma_general}) применимы
только если относительные отклонения всех величин малы
($\varepsilon_{x},\varepsilon_{y},\ldots\ll1$),
а измерения проводятся вдали от особых точек функции $f$ (производные
$f_{x}'$, $f_{y}'$ $\ldots$ не должны обращаться в бесконечность).
Также подчеркнём, что все полученные здесь формулы справедливы только
для \emph{независимых} переменных $x$, $y$, $\ldots$

Остановимся на некоторых важных частных случаях формулы
(\ref{eq:sigma_general}).

\example{Для суммы (или разности) $u=\sum\limits _{i=1}^{n}a_{i}x_{i}$ имеем
\begin{equation}
\sigma_{u}^{2}=\sum_{i=1}^{n}a_{i}^{2}\sigma_{x_{i}}^{2}.
\end{equation}}

\example{Найдём погрешность степенной функции:
    $u=x^{\alpha}\cdot y^{\beta}\cdot\ldots$. Тогда нетрудно получить,
что
\[
\frac{\sigma_{u}^{2}}{u^{2}}=\alpha^{2}\frac{\sigma_{x}^{2}}{x^{2}}+\beta^{2}
\frac{\sigma_{y}^{2}}{y^{2}}+\ldots
\]
или через относительные погрешности
\begin{equation}
\varepsilon_{u}^{2}=\alpha^{2}\varepsilon_{x}^{2}+\beta^{2}\varepsilon_{y}^{2}
+\ldots\label{eq:espilon_power}
\end{equation}}

\example{Вычислим погрешность произведения и частного: $u=xy$ или $u=x/y$.
    Тогда в обоих случаях имеем
\begin{equation}
\varepsilon_{u}^{2}=\varepsilon_{x}^{2}+\varepsilon_{y}^{2},
\end{equation}
то есть при умножении или делении относительные погрешности складываются
квадратично.}

\example{Рассмотрим несколько более сложный случай: нахождение угла по его тангенсу
\[
u=\arctg\frac{y}{x}.
\]
В таком случае, пользуясь тем, что $\left(\arctg z\right)'=\frac{1}{1+z^{2}}$,
где $z=y/x$, и используя производную сложной функции, находим
$u_{x}'=u_{z}'z'_{x}=-\frac{y}{x^{2}+y^{2}}$,
$u_{y}'=u'_{z}z'_{y}=\frac{x}{x^{2}+y^{2}}$, и наконец
\[
\sigma_{u}^{2}=\frac{y^{2}\sigma_{x}^{2}+x^{2}\sigma_{y}^{2}}{\left(x^{2}+y^{2}
\right)^{2}}.
\]}

\exercise{Найти погрешность вычисления гипотенузы $z=\sqrt{x^{2}+y^{2}}$
    прямоугольного треугольника по измеренным катетам $x$ и $y$.}

По итогам данного раздела можно дать следующие практические рекомендации.
\begin{itemize}
\item Как правило, нет смысла увеличивать точность измерения какой-то одной
величины, если другие величины, используемые в расчётах, остаются
измеренными относительно грубо --- всё равно итоговая погрешность
скорее всего будет определяться самым неточным измерением. Поэтому
все измерения имеет смысл проводить \emph{примерно с одной и той же
относительной погрешностью}.
\item При этом, как следует из (\ref{eq:espilon_power}), особое внимание
следует уделять измерению величин, возводимых при расчётах в степени
с большими показателями. А при сложных функциональных зависимостях
имеет смысл детально проанализировать структуру формулы
(\ref{eq:sigma_general}):
если вклад от некоторой величины в общую погрешность мал, нет смысла
гнаться за высокой точностью её измерения, и наоборот, точность некоторых
измерений может оказаться критически важной.
\item Следует избегать измерения малых величин как разности двух близких
значений (например, толщины стенки цилиндра как разности внутреннего
и внешнего радиусов): если $u=x-y$, то абсолютная погрешность
$\sigma_{u}=\sqrt{\sigma_{x}^{2}+\sigma_{y}^{2}}$
меняется мало, однако относительная погрешность
$\varepsilon_{u}=\frac{\sigma_{u}}{x-y}$
может оказаться неприемлемо большой, если $x\approx y$.
\end{itemize}


\chapter{Оценка параметров зависимостей}
\disclaimer{
    В этой главе представлена только очень краткая выжимка из теории оценок. Для
должного понимания этой главы математической статистики следует обратить
внимание на дополнительные материалы
% \todo{ссылка}.
}

\section{Оценка параметров}

Процедуру определения физических параметров по измерениями называют оцениванием,
а результат этой процедуры - оценкой. Такая терминология связана с тем, что
никакое физическое значение не может быть измерено с абсолютной точностью.
Всегда есть некоторая погрешность измерения. В простых случаях, оценки могут
быть получены простым усреднением результатов измерений или значений каких-то
функций этих результатов. Но в большинстве случаев измеряется не одна величина,
а зависимость одной величины от другой. В этом случае требуется построить оценку
параметра этой зависимости.

\example{
    Пусть есть резистор, подключенный к источнику постоянного напряжения
(напряжение можно менять), амперметр и вольтметр. Нужно определить сопротивление
$R$ этого резистора. Для того, чтобы получить значение можно провести ряд
измерений с одним или разными напряжениями $U$, посчитать в каждом случае
значение сопротивления как отношение напряжения к току: $R = U / I$, и усреднить
результаты. Такой метод можно использовать, но при этом результат базируется на
предположении о том, что зависимость напряжения от тока является строго
линейной, что, вообще говоря, соблюдается не всегда (к примеру может
существовать дополнительное паразитное сопротивление).

    Более корректным способом является измерение тока для различных значений
напряжения, построения линейной зависимости и оценки ее параметров. В этом
случае точность оценки остается такой же, но при этом возникает возможность
визуальной или аналитической проверки соответствия данных зависимости.
%     \todo[inline]{добавить картику?}
}

Для построения оценки нужны следующие компоненты:
\begin{itemize}
    \item \textbf{Данные}. Непосредственно измерения зависиомтси и погрешности
этих измерений. Экспериментальная погрешность является неотъемлемой частью
набора данных.
    \item \textbf{Модель}. Параметрическое описание зависимости. В примере выше
модель представляет собой прямую с одинм или двумя параметрами.
    \item \textbf{Оценщик}. Математическая процедура, которая по модели и данным
строит значения оценки.
\end{itemize}

Точечная оценка сама по себе не имеет смысла с точки зрения физики, поскольку не
позволяет определить погрешность результата. Поэтому для любого метода
оценивания, применяемом в физике необходимо определить процедуру определения
погрешности или интервала параметров с фиксированным вероятностным содержанием.

\section{Метод Пирсона (минимума \texorpdfstring{$\chi^{2}$}{chi2})}
\label{sec:chi2}

Наиболее распространенным способом оценки параметров является метом Пирсона,
который часто называют методом минимума $\chi^2$. Суть этого метода заключается
в том, что строится сумма вида:
\begin{equation}
    \chi^2 = \sum_i{\frac{(y_i - \mu(x_i,\theta))^2}{\sigma_i^2}},
\end{equation}
которую еще называют суммой $\chi^2$. Здесь $x_i$ и $y_i$ - координаты точек
зависимости, $\theta$ - исследуемый параметр или параметры, а $\mu$ - модель
(функция, описывающая зависимость). Погрешность измерений $\sigma_i$ как правило
указывают только для вертикальной шкалы ($y$). Оси стоит выбирать таким образом,
чтобы ошибка по вертикали была сильно больше, чем по горизонтали. Если ошибки
сопоставимы, то в качестве $\sigma_i$ можно брать среднеквадратичную ошибку по
обеим осям: $\sigma_i^2 = \sigma_{x,i}^2 + \sigma_{y,i}^2$.

Утверждается, что для суммы, определенной таким образом, значение параметра, при
котором $\chi^2$ будет минимальным будет давать корректную оценку этого
параметра (или параметров).

Оценку минимума $\chi^2$ можно получить как частный случай метода максимума
правдоподобия или из наивных соображений: очевидно, что чем ближе данные к
модельной кривой, тем меньше будет сумма.

\note{
    Процедура минимизации $\chi^2$ не имеет общего алгоритма. Для некоторых
частных случаев существуют аналитические решения, в частности, в разделе
\ref{sec:linear} рассмотрен случай линейной модели, но в общем случае следует
использовать соответствующие программы.
}

\subsection{Интервальная оценка}

Важным свойством метода Пирсона является встроенная возможность построения
интервальной оценки (нахождения погрешности). Для этого можно использовать тот
факт, что сама зависимость $\chi^2(\theta)$ в большинстве случае имеет вид
параболы, а величина $\exp{- \chi^2(\theta)/2}$ пропорциональна так называемой
функции правдоподобия, то есть вероятности получить заданный набор данных при
фиксированной модели и параметрах $\theta$. Легко видеть, что функция
правдоподобия в этом случае имеет вид нормального распределения и для
определения вероятностного содержания различных интервалов можно пользоваться
свойствами этого распределения. Для практической оценки достаточно найти такие
значения $\theta$, при которых значение $\chi^2$ будет отличаться от
минимального на 1. Эти значения будут соответствовать краям интервала с
вероятностным содержанием 68\% или $1-\sigma$ интервалу. Отклонение на 2 будет
соответствовать уже 95\% доверительному интервалу.

\note{
    Оценка методом минимума $\chi^2$ может быть использована в подавляющем
большинстве случаев, но все-таки не является универсальной. В случае с особо
точными измерениями рекомендуется обратиться к дополнительной литературе.
}

\subsection{Проверка качества фита}

Дополнительное полезоное свойство суммы $\chi^2$ заключается в том,
что она позволяет оценить, насколько хорошо данные описываются моделью.
В том случае, когда измерения распределены по нормальному закону и
независимы между собой, сумма $\chi^2$ оказывается распределена по
%
% \href{
% https://ru.wikipedia.org/wiki/\%D0\%A0\%D0\%B0\%D1\%81\%D0\%BF\%D1\%80\%D0\%B5\%
% D0\%B4\%D0\%B5\%D0\%BB\%D0\%B5\%D0\%BD\%D0\%B8\%D0\%B5_\%D1\%85\%D0\%B8-\%D0\%BA
% \%D0\%B2\%D0\%B0\%D0\%B4\%D1\%80\%D0\%B0\%D1\%82}
% {
одноименному распределению.
% }
При хорошем соответствии модели и данных величина
$\chi^2 / n $, где $n$ --- так называемое количество степеней
свободы (количество точек минус количество параметров), должна в среднем
быть равна 1. Значения существенно большие (2 и выше) свидетельствуют о
плохом соответствии или заниженных погрешностях. Значения меньше 0.5 как
правило свидетельствуют о завышенных ошибках.

\section{Метод наименьших квадратов}

    В случае, если все ошибки $\sigma_i$ одинаковы, множитель
$\frac{1}{\sigma^2}$ можно вынести за скобку. Для нахождения минимума
постоянный множитель не важен, поэтому мы можем назвать оценкой
наименьших кавдратов такое значение параметра $\hat\theta$, при
котором миниальна сумма квадратов:
\begin{equation}
  Q = \sum{(X_i - \mu_i(\theta))^2}.
\end{equation}
Эта оценка очевидно сохраняет все свойства оценки
минимума $\chi^2$, правда только в том случае, если все ошибки
действительно одинаковы.

Оценка наименьших квадратов удобна в том случае, когда не известны
ошибки отдельных измерений. Она является состоятельной и асимптотически
несмещенной, хотя ее эффективность оптимальна только для ограниченного
количества случаев. В некоторых случаях, ошибку измерения можно оценить
по разбросу данных, используя критерий Пирсона.

\paragraph{Интервальная оценка}

Оценка методом наименьших квадратов очевидно игнорирует информацию о
погрешностях измерений и не позволяет напрямую оценить погрешность
результата без дополнительных предположений. Для получения оценки
погрешностей надо сделать три предположения:

\begin{itemize}
    \item  Данные описываются предложенной моделью;
    \item  Отклонения данных от модельной кривой независимы между собой (носят
  статистический характер);
    \item  Статистические ошибки для всех точек равны между собой.
\end{itemize}

При этих предположениях, можно оценить $\sigma$ для каждой из точек
как средне квадратичное отклонение точек от наилучшей модели (той,
которая получена минимизацией суммы $Q$). После этого задача получения
погрешностей сводится к уже решенной для метода $\chi^2$.

\note{
По очевидным причинам оценка погрешностей, проведенная таким образом, не имеет
смысла для маленького количества измерений (меньше 8-10 точек). Все будет
работать и будет получен какой-то результат, но он будет довольно бессмысленным.
}


\note{
Одна из основных проблем, связанных с определением погрешностей методом
наименьших квадратов заключается в том, что он дает разумные погрешности даже в
том случае, когда данные вообще не соответствуют модели. По этой причине не
рекомендуется использовать его в тех случаях, когда погрешности измеренных
значений известны. Если других инструментов под рукой нет, то результаты работы
метода надо всегда проверять визуально по графику.
}



\section{Многопараметрические оценки}

    Однопараметрические оценки очень просты для понимания и реализации, но
довольно редко встречаются на практике. Даже при оценке параметров
линейно зависимости вида $y = k x + b$ уже существует два параметра:
$k$ - наклон прямой и $b$ - смещение. Все перечисленные выше
математические методы отлично работают и в многомерном случае, но
процесс поиска экстремума функции (максимума в случае метода максимума
правдоподобия и минимума в случае методов семейства наименьших
квадратов) и интерпретация результатов требуют использования специальных
программных пакетов.


\subsection{Доверительные области в многомерном случае}

    Принцип построения доверительной области в многомерном случае точно
такой же, как и для одномерных доверительных интервалов. Требуется найти
такую областью пространства параметров $\Omega$, для которой
вероятностное содержание для оценки параметра $\hat \theta$ (или
самого параметра $\theta$ в засимости от того, какой философии вы
придерживаетесь) будет равно некоторой наперед заданной величине
$\alpha$:
\begin{equation}
P(\theta \in \Omega) = \int_\Omega{L(X | \theta)}d\Omega = \alpha.
\end{equation}

Реализация на практике этого определения сталкивается с тремя
проблемами:

\begin{enumerate}
\item Взятие многомерного интеграла от произвольной функции - не тривиальная
  задача. Даже в случае двух параметров, уже требуется некоторый уровень
  владения вычислительной математикой и компьютерными методами. В случае
  большего числа параметров, как правило надо использовать специально
  разработанные для этого пакеты.
\item Определить центральный интервал для гипер-области гораздо сложнее, чем
  сделать это для одномерного отрезка. Единых правил для выбора такой
  области не существует.
\item Даже если удалось получить доверительную область, описать такой объект
  в общем случае не так просто, так что представление результатов
  составляет определенную сложность.
\end{enumerate}

Для решения этих проблем, пользуются следующим приемом: согласно
центральной предельной теореме, усреднение большого количества одинаково
распределенных величин дает нормально распределенную величину. Это же
верно и в многомерном случае. В большинстве случаев, мы ожидаем, что
функция правдоподобия будет похожа на многомерное нормальное
распределение:
\begin{equation}
    L(\theta) = \frac{1}{(2 \pi)^{n/2}\left|\Sigma\right|^{1/2}} e^{-\frac{1}{2}
(x - \mu)^T \Sigma^{-1} (x - \mu)},
\end{equation}
где n - размерность вектора параметров, $\mu$ - вектор
наиболее вероятных значений, а $\Sigma$ -
%
% \href{
% https://ru.wikipedia.org/wiki/\%D0\%9A\%D0\%BE\%D0\%B2\%D0\%B0\%D1\%80\%D0\%B8\%
% D0\%B0\%D1\%86\%D0\%B8\%D0\%BE\%D0\%BD\%D0\%BD\%D0\%B0\%D1\%8F_\%D0\%BC\%D0\%B0\
%D1\%82\%D1\%80\%D0\%B8\%D1\%86\%D0\%B0}
% {
ковариационная матрица
% }
распределения.

Для многомерного нормального распределения, линии постоянного уровня (то
есть поверхности, на которых значение плотности вероятности одинаковые)
имеют вид гипер-эллипса, определяемого уравнением
$(x - \mu)^T \Sigma^{-1} (x - \mu) = const$. Для любого вероятностного
содержания $\alpha$ можно подобрать эллипс, который будет
удовлетворять условию на вероятностное содержание. Интерес правда
редставляет не эллипс (в случае размерности больше двух, его просто
невозможно отобразить), а ковариацонная матрица. Диагональные элементы
этой матрицы являются дисперсиями соответствующих параметров (с учетом
всех корреляций параметров).


\section{Аналитическая оценка для линейной модели}
\label{sec:linear}

В случае, когда модель имеет линейный вид $y=kx+b$, оценка методом минимума
$\chi^2$ или наименьших квадратов может быть получена аналитически:
\[
\left\{ \left(x_{i},y_{i}\right),i=1\ldots n\right\} .
\]

Расстояние от экспериментальной точки от искомой прямой, измеренное
по вертикали, равно
\[
\Delta y_{i}=y_{i}-\left(kx_{i}+b\right).
\]
Найдём такие коэффициенты $k$ и $b$, чтобы сумма квадратов таких
расстояний для всех точек была минимальной:
\begin{equation}
S\!\left(k,b\right)=\sum\limits _{i=1}^{n}\Delta
y_{i}^{2}\to\mathrm{min}.\label{eq:mnk_S}
\end{equation}
Данный метод построения наилучшей прямой называют \emph{методом наименьших
квадратов} (МНК).

Рассмотрим сперва более простой частный случай. Пусть заведомо известно,
что искомая прямая проходит через ноль, то есть $b=0$ и $y=kx$.
Необходимое условие минимума функции $S\left(k\right)$, как известно,
есть равенство нулю её производной. Дифференцируя сумму (\ref{eq:mnk_S})
по $k$, считая все величины $\left\{ x_{i},\,y_{i}\right\} $ константами,
найдём
\[
\frac{dS}{dk}=-\sum\limits _{i=1}^{n}2x_{i}\left(y_{i}-kx_{i}\right)=0.
\]
Решая относительно $k$, находим
\[
k=\frac{\sum\limits _{i=1}^{n}x_{i}y_{i}}{\sum\limits _{i=1}^{n}x_{i}^{2}}.
\]
Поделив числитель и знаменатель на $n$, этот результат можно записать
более компактно:
\begin{equation}
\boxed{k=\frac{\left\langle xy\right\rangle }{\left\langle x^{2}\right\rangle
}}.\label{eq:MNK0}
\end{equation}
Угловые скобки означают усреднение по всем экспериментальным точкам:
\[
\left\langle \ldots\right\rangle \equiv\frac{1}{n}\sum\limits
_{i=1}^{n}\left(\ldots\right)_{i}
\]

В общем случае при $b\ne0$ функция $S\left(k,b\right)$ должна иметь
минимум как по $k$, так и по $b$. Поэтому имеем систему из двух
уравнений:
\begin{align*}
\frac{\partial S}{\partial k} & =-\sum\limits
_{i=1}^{n}2x_{i}\left(y_{i}-kx_{i}-b\right)=0,\\
\frac{\partial S}{\partial b} & =-\sum\limits
_{i=1}^{n}2\left(y_{i}-kx_{i}-b\right)=0.
\end{align*}
Решая систему, можно получить
\begin{equation}
\boxed{k=\frac{\left\langle xy\right\rangle -\left\langle x\right\rangle
\left\langle y\right\rangle }{\left\langle x^{2}\right\rangle -\left\langle
x\right\rangle ^{2}},\qquad b=\left\langle y\right\rangle -k\left\langle
x\right\rangle }.\label{eq:MNK}
\end{equation}
Эти соотношения и есть решение задачи о построении наилучшей прямой
методом наименьших квадратов.

\note{
    Совсем кратко формулу (\ref{eq:MNK}) можно записать, если ввести обозначение
    \begin{equation}
        D_{xy}\equiv\left\langle xy\right\rangle -\left\langle x\right\rangle
\left\langle y\right\rangle =\left\langle \left(x-\left\langle x\right\rangle
\right)\cdot\left(y-\left\langle y\right\rangle \right)\right\rangle
\label{eq:cov}
    \end{equation}
    (в математической статистике $D_{xy}$ называют \emph{ковариацией};
    при $x\equiv y$ имеем дисперсию $D_{xx}=\left\langle \left(x-\left\langle
x\right\rangle \right)^{2}\right\rangle $).
    Тогда
    \begin{equation}
        k=\frac{D_{xy}}{D_{xx}},\qquad b=\left\langle y\right\rangle
-k\left\langle x\right\rangle .\label{eq:MNK_short}
    \end{equation}
}

\subsection{Погрешность МНК в линейной модели}

Найдём погрешности $\sigma_{k}$ и $\sigma_{b}$ коэффициентов, вычисленных
по формуле (\ref{eq:MNK}) (или (\ref{eq:MNK0})).

Сделаем следующие предположения: погрешность измерений величины $x$
пренебрежимо мала: $\sigma_{x}\approx0$, а погрешность по $y$ одинакова
для всех экспериментальных точек $\sigma_{y}=\mathrm{const}$ и имеет
случайный характер (систематическая погрешность отсутствует).

Пользуясь в этих предположениях формулами для погрешностей косвенных
измерений (см. раздел (\ref{sec:kosv})) можно получить следующие
соотношения (выкладки здесь весьма громоздки, подробности можно найти
в п. (\ref{subsec:MMP})):
\begin{equation}
\sigma_{k}=\sqrt{\frac{1}{n-2}\left(\frac{D_{yy}}{D_{xx}}-k^{2}\right)},\label{
eq:MNK_sigma}
\end{equation}
\begin{equation}
\sigma_{b}=\sigma_{k}\sqrt{\left\langle x^{2}\right\rangle
},\label{eq:MNK_sigma_b}
\end{equation}
где использованы введённые выше сокращённые обозначения (\ref{eq:cov}).
Коэффициент $n-2$ отражает число независимых <<степеней
свободы>>: $n$ экспериментальных точек за вычетом двух
условий связи (\ref{eq:MNK}).

В частном случае $y=kx$ имеем
\begin{equation}
\sigma_{k}=\sqrt{\frac{1}{n-1}\left(\frac{\left\langle y^{2}\right\rangle
}{\left\langle x^{2}\right\rangle }-k^{2}\right)}.\label{eq:MNK_sigma0}
\end{equation}


\paragraph{Условия применимости МНК.}

Формулы (\ref{eq:MNK}) (или (\ref{eq:MNK0})) позволяют провести
прямую по \emph{любому} набору экспериментальных данных, а формулы
(\ref{eq:MNK_sigma}) (или (\ref{eq:MNK_sigma0})) --- вычислить
соответствующую среднеквадратичную ошибку для её коэффициентов. Однако
далеко не всегда результат будет иметь физический смысл. Перечислим
ограничения применимости данного метода.

В первую очередь метод наименьших квадратов --- статистический,
и поэтому он предполагает использование достаточно большого количества
экспериментальных точек (желательно $n>10$).

Поскольку метод предполагает наличие погрешностей только по $y$,
оси следует выбирать так, чтобы погрешность $\sigma_{x}$ откладываемой
по оси абсцисс величины была минимальна.

Кроме того, метод предполагает, что все погрешности в опыте ---
случайны. Соответственно, формулы (\ref{eq:MNK_sigma}) и (\ref{eq:MNK_sigma0})
применимы \emph{только для оценки случайной составляющей} ошибки $k$
и $b$. Если в опыте предполагаются достаточно большие систематические
ошибки, они должны быть оценены \emph{отдельно}. Отметим, что для
оценки систематических ошибок не существует строгих математических
методов, поэтому в таком случае проще и разумнее всего воспользоваться
описанным выше графическим методом.

Наконец, стоит предостеречь от использования МНК <<вслепую>>,
без построения графика. Этот метод неспособен выявить такие <<аномалии>>,
как отклонения от линейной зависимости, немонотонность, случайные
всплески и т.п. Все эти случаи могут быть легко обнаружены при построении
графика и требуют особого рассмотрения.

Резюмируя, можно сформулировать универсальную практическую рекомендацию:
если результаты какого-либо математического метода обработки данных
существенно расходятся с тем, что можно получить <<вручную>>
графически, есть все основания сомневаться в применимости метода в
данной ситуации.


% \chapter{Теория оценок}
\disclaimer{
    В этой главе представлено краткое введение в раздел математической статистики под названием "Теория оценок". Материал в этой главе нужен только тем, кто хочет получить хорошее качественное представление о теории анализа данных. Остальным рекомендуется ознакомиться только с материалом раздела \ref{sec:chi2}
}

\section{Понятие о точечной оценке}

    Математическая статистика имеет огромное количество разнообразнейших
применений, но с точки зрения экспериментальной физики (и как следствие
студентов, изучающих эту науку) наиболее интересным применением является
оценка параметров закономерностей. Пусть есть некоторое явление природы,
которое можно описать при помощи модели $M(\theta)$. Здесь $\theta$
- это некоторый набор параметров модели, которые могут принимать
различные значения. На этом этапе мы не оговариваем, как именно модель
описывает процесс и что мы можем принимать в качестве параметров.
Положим теперь, что существует некоторый выделенный набор параметров
$\theta_0$, который соответствует некоторому ``истинному'' состоянию
природы. Далее мы будем исходить из того предположения, что при попытке
предпринять некоторые измерения, мы будем получать результаты,
соответствующие нашей модели именно с этим набором параметров.

\textbf{Замечание} Тут важно заметить, что мы также негласно
предполагаем, что природа вообще действует согласно нашей модели, но
этот вопрос мы пока оставим за кадром. В какой-то мере мы вернемся к
нему, в главе 5, когда будем обсуждать теорию проверки гипотез.

Предаставим теперь, что мы провели некоторую серию экспериментов
$X = \{X_0, X_1,...X_N\}$, в которых мы тем или иным способом изучаем
состояние природы (будем дальше называть результаты этих экспериментов
экспериментальной выборкой). Нашей задачей в этой главе будет описание
процедуры, при помощи которой можно на основе выборки сделать вывод об
истинном состоянии природы $\theta_0$. Важно понимать, что в общем
случае, результаты измерений являются случайными величинами, поэтому
полученное нами на основании этих данных состояние природы также будет
случайной величиной в противовес истинному состоянию природы
$\theta_0$, которое вообще говоря, истинно случайной величиной не
является. Полученную величину будем называть \emph{точечной оценкой}
состояния природы $\hat{\theta}$ или просто оценкой. Саму процедуру, в
процессе которой получена оценка, будем называть оцениванием.

\textbf{Пример} Положим, что знания студента в области физики являются
состоянием природы (а точнее данного конкретного студента). Очевидно,
что досконально проверить этот факт не представляется возможным, поэтому
для измерения этой величины мы проводим эксперимент - экзамен. То, что
по результатам экзамена оказывается в ведомости является оценкой не
только с точки зрения деканата, но и с точки зрения математической
статистики.

В дальнейшем будем считать, что состояния природы описываются
действительным числом или набором действительных чисел. Сама по себе
теория этого не требует, но в противном случае довольно сложно
сравнивать состояния между собой (требуется определять понятие близости
в произвольном пространстве). В этом случае наша процедура оценивания:
\begin{equation}
    \hat{\theta} = f(X)
\end{equation} является действительной функцией на пространстве векторов
$X$, состоящих из случайных переменных. Такие функции еще называют
статистиками. Очевидно, что далеко не людая такая функция будет давать
тот результат, которого мы хотим. Поэтому вводятся дополнительные
обязательные свойства оценок.

\section{Свойства точечных оценок}

\subsection{Состоятельность}

Естественное пожелание к оценщику, заключается в том, что качество
оценки должно зависеть от объема выборки, числа измерений $n$
случайных переменных $X$: чем больше выборка, тем качественней оценка
$\hat{\theta}$. Иными словами, мы хотим, чтобы с ростом объема выборки
значение оценки приближалось к истинному значению параметра. При
использовании сходимости по вероятности оценку $\hat{\theta}$
определяют как состоятельную, если при любых $\varepsilon > 0$ и
$\eta > 0$, найдется такое $N$, что вероятность $P \left(
\left| \hat{\theta} - \theta \right| > \varepsilon \right) \textless{} \eta $ при всех $n > N$. Или другими
словами, всегда можно выбрать такое количество измерений, для которого
вроятность отклонения оценки от истинного значения не превышает наперед
заданное число.

\note{
Нужно заметить, что на практике оценки являются
состоятельными только когда при построении оценки не учитывается
систематическая ошибка. В противном случае, может наблюдаться сходимость
по вероятности не к нулю, а к некоторой фиксированной константе.
}

\subsection{Несмещенность}

Рассмотрим набор измерений, каждое из которых состоит из $n$
наблюдений $X$, характеризуемый функцией плотности вероятности
$P(X | \theta) = P(\hat\theta | \theta)$ при фиксированном $n$ и
определим смещение как отклонение среднего по этому набору
$\hat{\theta_n}$ от истинного \begin{equation}
   b = E[\hat{\theta_n}] - \theta
\end{equation} Оценка называется несмещенной, если $b = 0$.

Заметим, что смещение не зависит от измеренных величин, но зависит от
размера образца, формы оценщика и от истинных (в общем случае
неизвестных) свойств ФПВ $f(x)$, включая истинное значение параметра.
Если смещение исчезает в пределе $n \to \infty$, говорят об
асимптотически несмещенной оценке. Заметим, что из состоятельности
оценки не следует несмещенность. Это означает, что даже если
$\hat{\theta}$ сходится к истинной величине $\theta$ в единичном
эксперименте с большим числом измерений, нельзя утверждать, что среднее
$\hat{\theta}$ по бесконечному числу повторений эксперимента с
конечным числом измерений $n$ будет сходится к истинному $\theta$.
Несмещенные оценки пригодны для комбинирования результатов разных
экспериментов. В большинстве практических случаев смещение должно быть
мало по сравнению со статистической ошибкой и им пренебрегают.

\subsection{Несмещённая оценка среднего}

\todo[author = Nozik]{Надо этому столько времени уделять?}

Рассмотрим один конкретный пример смещенных оценок. Формула (\ref{eq:sigma}) $s_{x}^{2}=\frac{1}{n}\sum\Delta x_{i}^{2}$
вычисляется по конечному числу измерений, а потому даёт лишь приближённое
значение (\emph{оценку}) для величины дисперсии. Кроме того, при этом
мы вынуждены использовать \emph{выборочное} среднее $\overline{x}=\frac{1}{n}\sum x_{i}$,
что вносит дополнительную ошибку в вычисление $\sigma$. Оказывается,
что при малых $n$ эта ошибка может давать сильно заниженные результаты.
Математическая статистика рекомендует использовать слегка модифицированную
формулу --- так называемую называемой \emph{несмещённую}
оценку среднеквадратичного отклонения:
\begin{equation}
\boxed{s_{x}^{2}=\frac{1}{n-1}\sum\limits _{i=1}^{n}\left(x_{i}-\overline{x}\right)^{2}},\label{eq:sigma_straight}
\end{equation}
то есть сумму квадратов отклонений нужно делить не на полное число
слагаемых $n$, а уменьшенное на единицу.

Погрешность, вычисленную по формуле (\ref{eq:sigma_straight}), называют
также \emph{стандартным отклонением} от среднего. Эту формулу имеет
смысл применять при $n<10$. При $n\ge10$ результаты вычислений по
формулам (\ref{eq:sigma}) и (\ref{eq:sigma_straight}) отличаются
уже не более, чем на 5\%.

Коэффициент $n-1$ можно интерпретировать следующим образом. Отклонения
от выборочного среднего $\Delta x_{i}=x_{i}-\overline{x}$ подчиняются
очевидному соотношению $\sum\limits _{i=1}^{n}\Delta x_{i}=0$, поэтому
лишь $n-1$ из них являются независимыми. Таким образом, величину
$n-1$ можно назвать \emph{числом степеней свободы} для отклонений
измеряемой величины.

\example{
    Общее доказательство
    того, что оценка (\ref{eq:sigma_straight}) <<лучше>>,
    чем (\ref{eq:sigma}) (т.\,е. является <<несмещенной>>),
    довольно громоздко. В качестве относительно простого примера рассмотрим
    случай $n=2$. Пусть физическая величина $x$ имеет нормальное распределение
    с нулевым средним $\bar{x}=0$ (это не ограничивает общности, поскольку
    всегда можно сделать замену переменных $x-x_{0}\to x$) и дисперсией
    $s^{2}$. По выборке из двух измерений $\left\{ x_{1},\thinspace x_{2}\right\} $
    находим оценки для среднего
    \[
    \overline{x}=\frac{x_{1}+x_{2}}{2},
    \]
    (заметим, что вообще говоря $\left\langle x\right\rangle \ne0$) и
    среднеквадратичного отклонения 
    \[
    s_{x}^{2}=\frac{1}{2}\left[\left(x_{1}-\frac{x_{1}+x_{2}}{2}\right)^{2}+\left(x_{2}-\frac{x_{1}+x_{2}}{2}\right)^{2}\right]=\frac{1}{4}x_{1}^{2}-\frac{1}{2}x_{1}x_{2}+\frac{1}{4}x_{2}^{2}.
    \]
    
    Проведём усреднение полученных выражений по большому
    числу опытов, в каждом из которых проводится по $n=2$ измерений.
    Обозначим такое усреднение угловыми скобками. Тогда
    \[
    \left\langle \overline{x}\right\rangle =\frac{\left\langle x_{1}\right\rangle +\left\langle x_{2}\right\rangle }{2}=\left\langle x\right\rangle =0.
    \]
    Ввиду независимости измерений имеем $\left\langle x_{1}\cdot x_{2}\right\rangle =0$,
    так что
    \[
    \left\langle s_{x}^{2}\right\rangle =\frac{1}{4}\left\langle x_{1}^{2}\right\rangle +\frac{1}{4}\left\langle x_{2}^{2}\right\rangle =\frac{1}{2}\sigma^{2}\ne\sigma^{2}.
    \]
    Таким образом, оценка для среднего стремится к правильному пределу
    $\left\langle x\right\rangle =0$, однако оценка дисперсии по формуле
    (\ref{eq:sigma}) стремится к <<смещённому>> значению $\frac{1}{2}\sigma^{2}$, вдвое отличающемуся от правильного.
    Видно, что если бы мы воспользовались формулой (\ref{eq:sigma_straight}),
    то результат получился бы 
    \emph{несмещенный}.
    
    Чтобы разобраться в причине <<смещения>>,
    посмотрим, что получится, если вместо выборочного среднего $\overline{x}$
    использовать предельное $\left\langle x\right\rangle =x_{0}=0$:
    \[
    s_{x}^{2}=\frac{1}{2}\left(x_{1}^{2}+x_{2}^{2}\right)\qquad\to\qquad\left\langle s_{x}^{2}\right\rangle =\frac{1}{2}\left\langle x_{1}^{2}\right\rangle +\frac{1}{2}\left\langle x_{2}^{2}\right\rangle =\sigma^{2}.
    \]
    Видно, что оценка получается <<правильной>>,
    т.е. несмещенной. Таким образом, ошибка в оценке $\sigma$ возникает
    из-за использования выборочного среднего $\overline{x}$ вместо <<истинного>>
    $\left\langle x\right\rangle =\lim\limits_{n\to\infty}\overline{x}$.
}

\subsection{Эффективность}

Для сравнения разных методов оценки, очень важным свойством является
эффективность. Говоря простым языком, эффективность - это величина,
обратная разбросу значений $\hat{\theta}$ при применении к разным
наборам данных. Для того, чтобы хорошо разобраться в этом свойстве, надо
вспомнить, что оценка, как случайная величина, распределена с плотностью
$P(\hat\theta | \theta)$. Вид этого распределения может быть не
известен полностью, но знать его свойства -- низшие моменты --
необходимо. Среднее по нему суть смещение, а дисперсия
$\sigma_{\hat\theta}^2 = \int{ (\hat\theta - \theta} ) P(\hat\theta | \theta) d\hat\theta$
суть мера ошибки в определении оценки. Выбирая между различными
методами, мы, естественно, хотим, чтобы ошибка параметра была
минимальной из всех доступных нам способов его определения для
фиксированного эксперимента. Разные методы обладают разной
эффективностью и в общем случае при конечной статистике дисперсия
распределения оценки никогда не будет равна нулю. Разумеется, встает
вопрос о том, можно ли построить оценку с максимальной возможной
эффективностью.

\subsection{Граница Рао-Крамера и информация}
\disclaimer{Материал этого раздела предназначен для продвинутого изучения статистики}

\paragraph{Утверждение}

Пусть есть несмещенная оценка $\hat\theta$ параметра $\theta$, тогда
всегда выполняется неравенство:

\begin{equation}
  D(\hat\theta) \geq \frac{1}{I(\theta)},
\end{equation} где $I(\theta)$ - информация Фишера:

\begin{equation}
  I(\theta) = E_X \left[\left( \frac{\partial \ln L(X,\theta)}{\partial \theta} \right)^2 \right],
\end{equation} где в свою очередь $L(X, \theta)$ - функция
правдоподобия. $L(X,\theta) = \prod{P(X_i,\theta)}$ - вероятность
получить набор данных при фиксированном значении $\theta$.

\paragraph{Доказательство в одномерном случае}

Введем функцию

\begin{equation}
  W = \frac{\partial \ln L(X,\theta)} {\partial \theta}.
\end{equation}

Найдем математическое ожидание этой функции:

\begin{equation}
  E(W) = \int{L(X, \theta) W(X, \theta) dX} = \int{ L \frac{1}{L} \frac{\partial L}{\partial \theta} dX} = \frac{\partial}{\partial \theta} \int{L dX} = 0.
\end{equation}

Теперь рассмотрим ковариацию $E(\hat\theta W)$:

\begin{equation}
  E(\hat\theta W) = \int{\hat\theta \frac{1}{L} \frac{\partial L} {\partial {\theta}} L dX} = \frac{\partial}{\partial \theta} \int{\hat\theta L dX} =  \frac{\partial}{\partial \theta} E(\hat\theta).
\end{equation}

Для несмещенных оценок последнее выражение упрощается до вида
$E(\hat\theta W) = {\partial \theta}/{\partial \theta} = 1$ Согласно
неравенствую
\href{https://ru.wikipedia.org/wiki/\%D0\%9D\%D0\%B5\%D1\%80\%D0\%B0\%D0\%B2\%D0\%B5\%D0\%BD\%D1\%81\%D1\%82\%D0\%B2\%D0\%BE_\%D0\%9A\%D0\%BE\%D1\%88\%D0\%B8_\%E2\%80\%94_\%D0\%91\%D1\%83\%D0\%BD\%D1\%8F\%D0\%BA\%D0\%BE\%D0\%B2\%D1\%81\%D0\%BA\%D0\%BE\%D0\%B3\%D0\%BE}{Коши
--- Буняковского}, получаем:
$\sqrt{D(\hat\theta) D(W)} \geq \left| E(\hat\theta W) \right| = 1$.
Отсюда легко получаем желаемое утверждение.

\paragraph{Следствие}

Максимальная эффективность достигается в том случае, если величины
$\hat\theta$ и $W$ являются скоррелированными. Оценка,
максимизирующая функцию $L(X,\theta)$ является в общем случае
состоятельной, несмещенной, кроме того совпадает с оценкой вида
$W(\hat\theta, X) = 0$. То есть является максимально эффективной.

\section{Интервальные оценки}

    На практике применение точечных оценок сильно затруднено тем, что не
известно, на сколько каждая такая оценка точна. Действительно, мы можем
спокойно утверждать, что слон весит один килограмм если разброс нашей
оценки составляет больше массы слона. Для того, чтобы решить эту
проблему есть два пути. Первый путь - это на ряду с точечной оценки
указывать меру эффективности этой оценки или ее
разброс$\sigma_{\hat\theta}$. Но тут любой внимательный слушатель
заметит, что для определения эффективности, вообще говоря, надо знать
истинное значение пареметра $\theta$, которого мы, разумеется не
знаем. Следовательно приходится использовать не эффективность, а оценку
этой эффективности, которая сама по себе является случайной величиной.
Кроме того, часто случается, что распределение оценки является не
симметричным и описать его одним числом не удается.

Более корректным способом является построение интервальной оценки
(доверительного интервала). Формально определение интервальной оценки
будет отличаться в зависимости от того, какое определение вероятности вы
будете использовать.

\textbf{Частотная интерпретация}: интервальной оценкой параметра или
группы параметров $\theta$ с уровнем достоверности $\alpha$
называется такая область на пространстве параметров (в одномерном случае
- интервал), которая при многократном повторении эксперимента с
вероятностью (частотой) $\alpha$ перекрывает истинное значение
$\theta$.

\textbf{Субъективная интерпретация}: доверительным интервалом для
параметров $\theta$ будем называть такую область в пространстве
параметров, в которой интегральная апостериорная вероятность нахождения
истинного значения параметра равна $\alpha$.

Для точного описания результата проведения анализа как правило в
качестве результата приводят как точечную оценку, так и интервальную
оценку с некоторым уровнем достоверности (в английском варианте
Confidence Level или C. L.). В некоторых случаях приводят несколько
интервальных оценок с разным уровнем достоверности. В случае, когда речь
идет об определении верхней или нижней границы какого-то параметра,
точечная оценка как правило не имеет смысла и в качестве результата
дается только итнервальная оценка.

\note{ 
Точечная оценка также не имеет смысла в случае, когда
распределение оценки, скажем имеет вид однородного распределения на
отрезке. В этом случае все параметры на этом отрезке равновероятны и не
понятно, какой из них называть результатом.
}

\note{
Вполне очевидно, что для одних и тех же данных с
использованием одного и того же метода оценивания можно построить
бесконечное множество интервальных оценок с фиксированным уровнем
достоверности. Действительно, мы можем двигать интервал в разные стороны
таким образом, чтобы его вероятностное содержание не менялось. Обычно,
если не оговорено иначе, используются так называемые центральные
доверительные интервалы, в которых вероятностные содержание за границами
интервалов с обеих сторон равны.
}

\section{Методы построения оценок}

\subsection{Метод максимума правдоподобия}

\paragraph{Определение}

Введем функцию правдоподобия следующим образом:
$L(X,\theta) = \prod{P(X_i,\theta)}$, где $P(X_i, \theta)$ -
вероятность получить случайным образом компоненту данных с номером $i$
при истинном значении параметра $\theta$. Будем называеть оценкой
максимума правдоподобия такую $\hat\theta$, для которой
$L(\hat\theta, X)$ максимально для заданного набора данных. Такая
оценка в общем случае является состоятельной и несмещенной. Она имеет
очень хорошо понятный физический смысл: $\hat\theta$ - это такое
значение параметра, для которого вероятность получить набор данных
максимальна. Кроме того, как упоминалось в параграфе \textbf{3.1.2},
такая оценка кроме всего прочего будет еще и наиболее эффективной из
всех возможных (достаточной статистикой). 

\paragraph{Ограничения Оценка}
является эффективной только если область измереия $X$ не зависит от
$\theta$. 

\paragraph{Интервальная оценка} Для построения интервальной
оценки воспользуемся субъективной вероятностью и теоремой Байеса:
\begin{equation}
  P(\theta | X) = \frac{L(X|\theta) \pi(\theta)}{\int{LdX}}
\end{equation} Здесь $L$ - функция правдоподобия, а $\pi$ -
априорная вероятность для параметра $\theta$. Если нет никакой
дополнительной информации оп параметре, то мы можем положить
$\pi = 1$. Мы получаем, что распределение значения реального параметра
повторяет форму функции правдоподобия. Вероятность того, что параметр
$\theta$ лежит в диапазоне от $a$ до $b$ составляет
\begin{equation}
  P(a < \theta < b) = \int_b^a{L(X | \theta)}.
\end{equation} 

Интервальная оценка в асимптотическом случае
Согласно центральной предельной теореме, при достаточно большом
количестве данных $X$ (и при разумных предположениях о форме
распределения для этих данных), функция правдоподобия будет иметь вид
нормального распределения. В этом случае центральный доверительный
интервал можно вычислить, пользуясь аналитической формулой. Центральный
доверительный интервал с уровнем значимости 68\% будет соответствовать
диапазону между значениями $\theta$, такими, что значение функции
правдоподобия в них отличается от максимума на 0.5.

\paragraph{Логарифмическое правдоподобие}
В реальных задачах часто работают не с
самой функцией правдоподобия, а с ее логарифмом
$\ln L(X|\theta) = \sum{\ln P(X_i | \theta)}$. Это связано с тем, что
при численных манипуляциях проще работать со сложением, чем с
умножением. Очевидно, что максимум правдоподобия будет одновременно и
максимумом его логарифма и обратно. Для построения интервальных оценок
логарифм прадоподобия не годится.


\subsection{Метод минимума \texorpdfstring{$\chi^{2}$}{chi2}}
\label{sec:chi2}

    Разберем частный, но очень часто встречающийся случай, когда
распределение измеренной величины - нормальное (Гауссово):
\begin{equation}
  P(X_i | \theta) = \frac{1}{\sqrt {2 \pi} \sigma_i} e^{- \frac{(X_i - \mu_i(\theta))^2}{2 \sigma_i^2}}
\end{equation} Здесь $\mu_i(\theta)$ - модельное значение. Опуская
нормировочный множитель, не существенный для нахождения максимума можно
записать логарифм функции правдоподобия в виде: \begin{equation}
  \ln L(X_i | \theta) = - \sum{\frac{(X_i - \mu_i(\theta))^2}{2 \sigma_i^2}}
\end{equation} Определим величину \begin{equation}
    \chi^2 = - 2 \ln L(X_i | \theta) =  \sum{\frac{(X_i - \mu_i(\theta))^2}{\sigma_i^2}}.
\end{equation} Назовем оценкой минимума $\chi^2$ такое значение
параметра, при котором эта величина минимально. Очевидно, что эта оценка
эквивалента максимуму правдоподобия для нормально расспределенных
измерений. Оценка минимума $\chi^2$ будет состоятельной и несмещенной
и для других распределений измерений, но только для таких измерений она
будет обладать максимальной эффективностью.

\subsubsection{Интервальная оценка}

Также как и в методе максимума правдоподобия, центральный интервал можно
получить из аналитической форму функции правдоподобия (для нормально
распределенных ошибок, форма правдоподобия будет нормальной всегда, а не
только в асимптотике). Из-за множителя 2, мы получаем, что центральный
интервал в 68\% будет соответствовать отклонению $\chi^2$ на 1.
Отклонение на 2 будет соответствовать уже 95\% доверительному интервалу.

\subsubsection{Проверка качества фита}

Дополнительное полезоное свойство суммы $\chi^2$ заключается в том,
что она позволяет оценить, насколько хорошо данные описываются моделью.
В том случае, когда измерения распределены по нормальному закону и
независимы между собой, сумма $\chi^2$ оказывается распределена по
\href{https://ru.wikipedia.org/wiki/\%D0\%A0\%D0\%B0\%D1\%81\%D0\%BF\%D1\%80\%D0\%B5\%D0\%B4\%D0\%B5\%D0\%BB\%D0\%B5\%D0\%BD\%D0\%B8\%D0\%B5_\%D1\%85\%D0\%B8-\%D0\%BA\%D0\%B2\%D0\%B0\%D0\%B4\%D1\%80\%D0\%B0\%D1\%82}{одноименному
распределению}. При хорошем соответствии модели и данных величина
$\chi\^{}2 / n $, где $n$ - так называемое количество степеней
свободы (количество точек минус количество параметров), должна в среднем
быть равна 1. Значения существенно большие (2 и выше) свидетельствуют о
плохом соответствии или заниженных погрешностях. Значения меньше 0.5 как
правило свидетельствуют о завышенных ошибках.


\subsection{Метод наименьших квадратов}

    В случа, если все ошибки $\sigma_i$ одинаковы, множитель
$\frac{1}{\sigma^2}$ можно вынести за скобку. Для нахождения минимума
постоянный множитель не важен, поэтому мы можем назвать оценкой
наименьших кавдратов такое значение параметра $\hat\theta$, при
котором миниальна сумма квадратов: \begin{equation}
  Q = \sum{(X_i - \mu_i(\theta))^2}.
\end{equation} Эта оценка очевидно сохраняет все свойства оценки
минимума $\chi^2$, правда только в том случае, если все ошибки
действительно одинаковы.

Оценка наименьших квадратов удобна в том случае, когда не известны
ошибки отдельных измерений. Она является состоятельной и асимптотически
несмещенной, хотя ее эффективность оптимальна только для ограниченного
количества случаев. В некоторых случаях, ошибку измерения можно оценить
по разбросу данных, используя критерий Пирсона.

\paragraph{Интервальная оценка}

Оценка методом наименьших квадратов очевидно игнорирует информацию о
погрешностях измерений и не позволяет напрямую оценить погрешность
результата без дополнительных предположений. Для получения оценки
погрешностей надо сделать три предположения:

\begin{itemize}
    \item  Данные описываются предложенной моделью;
    \item  Отклонения данных от модельной кривой независимы между собой (носят
  статистический характер);
    \item  Статистические ошибки для всех точек равны между собой.
\end{itemize}

При этих предположениях, можно оценить $\sigma$ для каждой из точек
как средне квадратичное отклонение точек от наилучшей модели (той,
которая получена минимизацией суммы $Q$). После этого задача получения
погрешностей сводится к уже решенной для метода $\chi^2$.

\note{
По очевидным причинам оценка погрешностей, проведенная таким образом, не имеет смысла для маленького количества измерений (меньше 8-10 точек). Все будет работать и будет получен какой-то результат, но он будет довольно бессмысленным.
}


\note{
Одна из основных проблем, связанных с определением погрешностей методом наименьших квадратов заключается в том, что он дает разумные погрешности даже в том случае, когда данные вообще не соответствуют модели. По этой причине не рекомендуется использовать его в тех случаях, когда погрешности измеренных значений известны. Если других инструментов под рукой нет, то результаты работы метода надо всегда проверять визуально по графику.
}

\subsection{Другие методы}

    Существует довольно большое количество других методов определения
параметров зависимости. Как правило они специально заточены под
конкретную задачу. В качестве примера можно привести
\href{https://ru.wikipedia.org/wiki/\%D0\%9C\%D0\%B5\%D1\%82\%D0\%BE\%D0\%B4_\%D0\%BC\%D0\%BE\%D0\%BC\%D0\%B5\%D0\%BD\%D1\%82\%D0\%BE\%D0\%B2}{метод
моментов}. Методо дает состоятельную и несмещенную, но довольно слабо
эффективную оценку, но при этом позволяет очень быстро делать оценки для
больших объемов данных. Еще один пример -
\href{https://arxiv.org/abs/physics/0604127}{метод квази-оптимальных
весов Ф. В. Ткачева}, напротив обладает высокой эффективностью и
стабильностью, но несколько сложнее в реализации.

\section{Многопараметрические оценки}

    Однопараметрические оценки очень просты для понимания и реализации, но
довольно редко встречаются на практике. Даже при оценке параметров
линейно зависимости вида $y = k x + b$ уже существует два параметра:
$k$ - наклон прямой и $b$ - смещение. Все перечисленные выше
математические методы отлично работают и в многомерном случае, но
процесс поиска экстремума функции (максимума в случае метода максимума
правдоподобия и минимума в случае методов семейства наименьших
квадратов) и интерпретация результатов требуют использования специальных
программных пакетов.

    
\subsection{Доверительные области в многомерном случае}

    Принцип построения доверительной области в многомерном случае точно
такой же, как и для одномерных доверительных интервалов. Требуется найти
такую областью пространства параметров $\Omega$, для которой
вероятностное содержание для оценки параметра $\hat \theta$ (или
самого параметра $\theta$ в засимости от того, какой философии вы
придерживаетесь) будет равно некоторой наперед заданной величине
$\alpha$: \begin{equation}
  P(\theta \in \Omega) = \int_\Omega{L(X | \theta)}d\Omega = \alpha.
\end{equation}

Реализация на практике этого определения сталкивается с тремя
проблемами:

\begin{enumerate}
\item Взятие многомерного интеграла от произвольной функции - не тривиальная
  задача. Даже в случае двух параметров, уже требуется некоторый уровень
  владения вычислительной математикой и компьютерными методами. В случае
  большего числа параметров, как правило надо использовать специально
  разработанные для этого пакеты.
\item Определить центральный интервал для гипер-области гораздо сложнее, чем
  сделать это для одномерного отрезка. Единых правил для выбора такой
  области не существует.
\item Даже если удалось получить доверительную область, описать такой объект
  в общем случае не так просто, так что представление результатов
  составляет определенную сложность.
\end{enumerate}

Для решения этих проблем, пользуются следующим приемом: согласно
центральной предельной теореме, усреднение большого количества одинаково
распределенных величин дает нормально распределенную величину. Это же
верно и в многомерном случае. В большинстве случаев, мы ожидаем, что
функция правдоподобия будет похожа на многомерное нормальное
распределение: \begin{equation}
    L(\theta) = \frac{1}{(2 \pi)^{n/2}\left|\Sigma\right|^{1/2}} e^{-\frac{1}{2} (x - \mu)^T \Sigma^{-1} (x - \mu)}
\end{equation} где n - размерность вектора параметров, $\mu$ - вектор
наиболее вероятных значений, а $\Sigma$ -
\href{https://ru.wikipedia.org/wiki/\%D0\%9A\%D0\%BE\%D0\%B2\%D0\%B0\%D1\%80\%D0\%B8\%D0\%B0\%D1\%86\%D0\%B8\%D0\%BE\%D0\%BD\%D0\%BD\%D0\%B0\%D1\%8F_\%D0\%BC\%D0\%B0\%D1\%82\%D1\%80\%D0\%B8\%D1\%86\%D0\%B0}{ковариационная
матрица} распределения.

Для многомерного нормального распределения, линии постоянного уровня (то
есть поверхности, на которых значение плотности вероятности одинаковые)
имеют вид гипер-эллипса, определяемого уравнением
$(x - \mu)^T \Sigma^{-1} (x - \mu) = const$. Для любого вероятностного
содержания $\alpha$ можно подобрать эллипс, который будет
удовлетворять условию на вероятностное содержание. Интерес правда
редставляет не эллипс (в случае размерности больше двух, его просто
невозможно отобразить), а ковариацонная матрица. Диагональные элементы
этой матрицы являются дисперсиями соответствующих параметров (с учетом
всех корреляций параметров).

    
\section{Аналитическая оценка для линейной модели}

Рассмотрим математически более строгий метод построения наилучшей
прямой $y=kx+b$ по набору экспериментальных точек 
\[
\left\{ \left(x_{i},y_{i}\right),i=1\ldots n\right\} .
\]

Расстояние от экспериментальной точки от искомой прямой, измеренное
по вертикали, равно
\[
\Delta y_{i}=y_{i}-\left(kx_{i}+b\right).
\]
Найдём такие коэффициенты $k$ и $b$, чтобы сумма квадратов таких
расстояний для всех точек была минимальной:
\begin{equation}
S\!\left(k,b\right)=\sum\limits _{i=1}^{n}\Delta y_{i}^{2}\to\mathrm{min}.\label{eq:mnk_S}
\end{equation}
Данный метод построения наилучшей прямой называют \emph{методом наименьших
квадратов} (МНК).

Рассмотрим сперва более простой частный случай. Пусть заведомо известно,
что искомая прямая проходит через ноль, то есть $b=0$ и $y=kx$.
Необходимое условие минимума функции $S\left(k\right)$, как известно,
есть равенство нулю её производной. Дифференцируя сумму (\ref{eq:mnk_S})
по $k$, считая все величины $\left\{ x_{i},\,y_{i}\right\} $ константами,
найдём 
\[
\frac{dS}{dk}=-\sum\limits _{i=1}^{n}2x_{i}\left(y_{i}-kx_{i}\right)=0.
\]
Решая относительно $k$, находим 
\[
k=\frac{\sum\limits _{i=1}^{n}x_{i}y_{i}}{\sum\limits _{i=1}^{n}x_{i}^{2}}.
\]
Поделив числитель и знаменатель на $n$, этот результат можно записать
более компактно:
\begin{equation}
\boxed{k=\frac{\left\langle xy\right\rangle }{\left\langle x^{2}\right\rangle }}.\label{eq:MNK0}
\end{equation}
Угловые скобки означают усреднение по всем экспериментальным точкам:
\[
\left\langle \ldots\right\rangle \equiv\frac{1}{n}\sum\limits _{i=1}^{n}\left(\ldots\right)_{i}
\]

В общем случае при $b\ne0$ функция $S\left(k,b\right)$ должна иметь
минимум как по $k$, так и по $b$. Поэтому имеем систему из двух
уравнений:
\begin{align*}
\frac{\partial S}{\partial k} & =-\sum\limits _{i=1}^{n}2x_{i}\left(y_{i}-kx_{i}-b\right)=0,\\
\frac{\partial S}{\partial b} & =-\sum\limits _{i=1}^{n}2\left(y_{i}-kx_{i}-b\right)=0.
\end{align*}
Решая систему, можно получить
\begin{equation}
\boxed{k=\frac{\left\langle xy\right\rangle -\left\langle x\right\rangle \left\langle y\right\rangle }{\left\langle x^{2}\right\rangle -\left\langle x\right\rangle ^{2}},\qquad b=\left\langle y\right\rangle -k\left\langle x\right\rangle }.\label{eq:MNK}
\end{equation}
Эти соотношения и есть решение задачи о построении наилучшей прямой
методом наименьших квадратов.

{\footnotesize{}Совсем кратко формулу (\ref{eq:MNK}) можно записать,
если ввести обозначение
\begin{equation}
D_{xy}\equiv\left\langle xy\right\rangle -\left\langle x\right\rangle \left\langle y\right\rangle =\left\langle \left(x-\left\langle x\right\rangle \right)\cdot\left(y-\left\langle y\right\rangle \right)\right\rangle \label{eq:cov}
\end{equation}
(в математической статистике $D_{xy}$ называют }\emph{\footnotesize{}ковариацией}{\footnotesize{};
при $x\equiv y$ имеем дисперсию $D_{xx}=\left\langle \left(x-\left\langle x\right\rangle \right)^{2}\right\rangle $).
Тогда
\begin{equation}
k=\frac{D_{xy}}{D_{xx}},\qquad b=\left\langle y\right\rangle -k\left\langle x\right\rangle .\label{eq:MNK_short}
\end{equation}
}{\footnotesize\par}

\paragraph{Погрешность МНК.}

Найдём погрешности $\sigma_{k}$ и $\sigma_{b}$ коэффициентов, вычисленных
по формуле (\ref{eq:MNK}) (или (\ref{eq:MNK0})).

Сделаем следующие предположения: погрешность измерений величины $x$
пренебрежимо мала: $\sigma_{x}\approx0$, а погрешность по $y$ одинакова
для всех экспериментальных точек $\sigma_{y}=\mathrm{const}$ и имеет
случайный характер (систематическая погрешность отсутствует).

Пользуясь в этих предположениях формулами для погрешностей косвенных
измерений (см. раздел (\ref{sec:kosv})) можно получить следующие
соотношения (выкладки здесь весьма громоздки, подробности можно найти
в п. (\ref{subsec:MMP})):
\begin{equation}
\sigma_{k}=\sqrt{\frac{1}{n-2}\left(\frac{D_{yy}}{D_{xx}}-k^{2}\right)},\label{eq:MNK_sigma}
\end{equation}
\begin{equation}
\sigma_{b}=\sigma_{k}\sqrt{\left\langle x^{2}\right\rangle },\label{eq:MNK_sigma_b}
\end{equation}
где использованы введённые выше сокращённые обозначения (\ref{eq:cov}).
Коэффициент $n-2$ отражает число независимых <<степеней
свободы>>: $n$ экспериментальных точек за вычетом двух
условий связи (\ref{eq:MNK}).

В частном случае $y=kx$ имеем
\begin{equation}
\sigma_{k}=\sqrt{\frac{1}{n-1}\left(\frac{\left\langle y^{2}\right\rangle }{\left\langle x^{2}\right\rangle }-k^{2}\right)}.\label{eq:MNK_sigma0}
\end{equation}


\paragraph{Условия применимости МНК.}

Формулы (\ref{eq:MNK}) (или (\ref{eq:MNK0})) позволяют провести
прямую по \emph{любому} набору экспериментальных данных, а формулы
(\ref{eq:MNK_sigma}) (или (\ref{eq:MNK_sigma0})) --- вычислить
соответствующую среднеквадратичную ошибку для её коэффициентов. Однако
далеко не всегда результат будет иметь физический смысл. Перечислим
ограничения применимости данного метода.

В первую очередь метод наименьших квадратов --- статистический,
и поэтому он предполагает использование достаточно большого количества
экспериментальных точек (желательно $n>10$).

Поскольку метод предполагает наличие погрешностей только по $y$,
оси следует выбирать так, чтобы погрешность $\sigma_{x}$ откладываемой
по оси абсцисс величины была минимальна.

Кроме того, метод предполагает, что все погрешности в опыте ---
случайны. Соответственно, формулы (\ref{eq:MNK_sigma}) и (\ref{eq:MNK_sigma0})
применимы \emph{только для оценки случайной составляющей} ошибки $k$
и $b$. Если в опыте предполагаются достаточно большие систематические
ошибки, они должны быть оценены \emph{отдельно}. Отметим, что для
оценки систематических ошибок не существует строгих математических
методов, поэтому в таком случае проще и разумнее всего воспользоваться
описанным выше графическим методом.

Наконец, стоит предостеречь от использования МНК <<вслепую>>,
без построения графика. Этот метод неспособен выявить такие <<аномалии>>,
как отклонения от линейной зависимости, немонотонность, случайные
всплески и т.п. Все эти случаи могут быть легко обнаружены при построении
графика и требуют особого рассмотрения.

Резюмируя, можно сформулировать универсальную практическую рекомендацию:
если результаты какого-либо математического метода обработки данных
существенно расходятся с тем, что можно получить <<вручную>>
графически, есть все основания сомневаться в применимости метода в
данной ситуации.

\chapter{Рекомендации по выполнению и представлению результатов работы}

\section{Проведение измерений}

Ключевым элементом проведения лабораторной работы является ведение \emph{лабораторного
журнала}. Журнал является главным источником информации о проведенном
эксперименте.

\subsection{Правила ведения лабораторного журнала}
\label{sec:journal}

\begin{itemize}
    \small
    \item Лабораторный журнал оформляется \emph{от руки}.
    Для оформления лучше использовать большую тетрадь \emph{формата A4}
    с несъемными листами. Это правило
связано с тем, что никакой электронный журнал не обладает такой же
информативностью и гибкостью в оформлении. Рукописные
журналы используются на всех крупных современных физических экспериментах.

    \item В журнале необходимо фиксировать всю информацию о проводимом
эксперименте: название работы, дату и время проведения эксперимента, типы
использованных приборов, схему установки, а также любые другие показатели,
которые могут быть связаны с проведением работы и обработкой результатов.

    \note{Недопустимым считается отсутствие какой-то информации в журнале
поскольку она \textquote{есть в лабнике}. Информация в описании может быть
устаревшей или не соответствовать конкретной установке. Допускается
использование элементов описания, нарисованных на компьютере, напечатанных и
вклеенных в журнал.}

    \item Лабораторный журнал должен содержать \emph{максимально полную}
информацию о процессе проведения эксперимента, а не только результаты измерений.
Обязательно должны быть указаны все проводимые экспериментатором действия.
% (ссылка на пункт программы в лабнике, если эта программа не переписана в журнал,
% не допускается).
По возможности должны присутствовать временные метки всех
действий (например, чтобы потом можно было сверить журнал с журналами других
студентов, работающих в это время, или с другой информацией).

    \example{Результаты измерений могу зависеть от окружающей температуры 
    и влажности (особенно в работах по термодинамике). Поэтому если в момент измерений, кто-то открывает дверь или окно в лаборатории, может случиться синхронный скачок
измеряемых значений на всех установках. Этот скачок может быть не заметен на
стадии измерений и обнаружен только при обработке. Если хотя бы в одном журнале
есть запись о том, что была открыта дверь, а во всех остальных есть временные
метки, то можно при обработке учесть изменение условий.}

    \item Не допускается исключение из журнала \textquote{неправильных} (или показавшихся
неправильными) измерений. Если по какой-то причине сделано заключение о том, что
измерение проведено в неправильных условиях, результаты должны быть сохранены, а
в журнале сделана пометка о том, почему это измерение считается ненадежным. 
История знает много примеров, когда на первый взгляд \textquote{ошибочные} измерения приводили к открытиям.

    \item Рекомендуется дублировать в журнале показания приборов даже если они
записываются автоматически электронным способом. Это позволяет избежать многих
ошибок.

    \item При записи результатов измерений не допускается использование карандаша, корректора или черновиков.
\end{itemize}

\subsection{Подготовка к работе}

Перед выполнением лабораторной работы \emph{необходимо}
\begin{itemize}
    \small
    \item ознакомиться с описанием работы и теоретическим введением по
соответствующей теме: получить таким образом представление об
изучаемых явлениях, порядках измеряемых величин и связывающих их закономерностях,
а также о методе измерения, используемых приборах и последовательности
действий при проведении измерений;

% \note{Необходимо изучить не только описание работы, но и теоретическое введение
%     к соответствующей главе \emph{полностью}.}

    \item продумать предложенный в описании план действий, оценить необходимое
    количество измерений. Количество измерений студент должен оценивать
    самостоятельно исходя из а)~требуемой точности измерений и б)~планируемого времени выполнения работы;
% Допускается подготовка таблиц для измерений заранее. В этом случае таблицу надо
% делать с запасом, поскольку часто по ходу эксперимента выясняется, что нужно
% проделать или переделать часть измерения.

    \item желательно заранее (в крайнем случае, на начальном этапе работы)
    представлять диапазон изменения измеряемых величин и выбрать для них
    соответствующие единицы измерения;

    \item предварительно оценить достижимую точность
    измерений, проанализировать возможные источники погрешностей и их
    влияние на погрешность конечного результата.
%     \example{При измерениях величин, имеющих степенную зависимость
%     от непосредственно измеряемых, относительная погрешность величин,
%     входящих с б\'{о}льшими показателями степени, должна быть меньше,
%     то есть их следует измерять точнее. По возможности следует избегать
%     методов, при которых приходится вычислять разность двух близких
%     по значениям величин.
\end{itemize}

Для подготовки к выполнению работы \emph{рекомендуется} наличие в журнале следующих
элементов:
\begin{itemize}
    \small
    \item название (не только номер!) и цели работы; схема установки и описание использованных приборов.

%    \item 
%    Следует иметь в виду, что \emph{реальная} схема конкретной установки
%    может отличаться от той, что изображена в описании;

    \item основные теоретические положения и расчётные формулы для данной работы.
    Не следует переписывать (или перепечатывать) всё, что изложено в описании
    работы --- нужно выделить ключевые моменты, необходимые для
    проведения работы и интерпретации результатов.

    \item план работы с оценкой количества измерений и времени, необходимого на
выполнение каждого пункта.
% План не обязан (и не должен) один в один повторять
% то, что написано в описании.
В процессе работы план может меняться, о чем должна
быть сделана соответствующая пометка в журнале (с указанием причин).
\end{itemize}


\subsection{Начало работы}

В начале работы необходимо тщательно ознакомиться с экспериментальной
установкой, проверить работоспособность приборов. Сведения о приборах
%(в первую очередь класс точности, максимальное значение на шкале, по которой
%производятся измерения, и цену деления) 
и условиях эксперимента необходимо
зафиксировать в лабораторном журнале. Рекомендуется переписать полные наименования
приборов --- в этом случае недостающую информацию о них можно всегда найти в интернете.

% Нужно разобраться, как они регулируются, включаются и выключаются.

% Всегда очень важно аккуратное и бережное обращение с приборами. Не
% следует вскрывать чувствительные приборы и менять настройку.

% , так как они потребуются при получении окончательных результатов.

\note{При сборке электрических схем источники питания подключаются
к схеме в последнюю очередь. Регулировочные ручки напряжения или тока должны
исходно находиться в \emph{нулевом} положении.}

Прежде чем приступить к основным измерениям, необходимо проверить
работу установки. Первые измерения должны быть контрольными, чтобы
убедиться, что все работает нормально, диапазон и точность измерений
выбраны правильно. Если разброс повторных измерений не превышает
инструментальную погрешность, то многократных измерений не требуется.

Замеченные неполадки в работе приборов и установок надо зафиксировать в журнале
и сообщить об этом преподавателю.

\subsection{Выбор количества измерений}

%Выбор количества измерений является сложной задачей, не имеющей
%единого алгоритма принятия решений. 
%Тем не менее, 
Каждый экспериментатор
(в том числе, студент) должен самостоятельно определять, какое количество
измерений является достаточным, базируясь на соображениях точности результатов,
времени измерений и здравого смысла.

% Проблему выбора можно разделить
% на две ситуации:
% \begin{itemize}
%     \item Измерение фиксированной величины
%     \item Измерение зависимости
% \end{itemize}

\paragraph{Измерение фиксированной величины.}
При измерении некоторой отдельной величины количество необходимых измерений
зависит от наблюдаемого разброса результатов. Для первичной оценки этого разброса
рекомендуется проделать измерения как минимум 3--4 раза (если позволяет время).
Разброс полученных значений приблизительно соответствует
статистической ошибке отдельного измерения. Если разброс существенно превышает
точность измерительных приборов, то имеет смысл провести более длительную серию (8--10) измерений, и после этого вычислить среднеквадратичное отклонение отдельного измерения от среднего.

Если остальные измерения в серии проводятся аналогичным образом,
то разумно ожидать, что разброс остальных измерений будет таким же,
и повторять длинную серию  для всех измерений не нужно.
% Разумеется, это правило не является абсолютным. Во многих
% случаях разброс измерений будет разным при разных настройках измерительной
% аппаратуры.

\example{Допустим, требуется с помощью секундомера измерять периоды
    колебания маятника с точностью $\varepsilon=0,1\%$. Предположим,
    что ошибка измерения связана только с временем реакции экспериментатора.
    Эта ошибка, очевидно, не зависит от длительности измерения и её можно
    измерить непосредственно: для этого можно 8--10 раз измерить
    время некоторого целого числа колебаний и по результатам вычислить
    среднеквадратичную погрешность времени реакции $\sigma_{t}^{\text{реакц}}$
    (как правило, $\sigma_{t}^{\text{реакц}}\sim 0,2\;\text{с}$).

    По заданной абсолютной величине погрешности и требуемой точности
    $\varepsilon$ находим необходимое полное время измерений:
    $t = \sigma_{t}^{реакц} / \varepsilon \sim 200\;\text{с}$. Тогда все последующие измерения
    можно не повторять многократно, а проводить 1--2 раза в течение
    рассчитанного времени $t$.

    Эти рассуждения не учитывают возможное отставание или опережение часов
    при больших $t$ --- предполагается, что часы
    откалиброваны с достаточной точностью (их \textquote{уход} за время $t$
    не превышает времени реакции).}



\paragraph{Измерение зависимостей.}
При измерениях функциональной зависимости в первую очередь следует позаботиться
о том, насколько хорошо будут вычисляться параметры этой зависимости.
Число параметров не может быть больше, чем число экспериментальных точек
(невозможно строить прямую по одной точке!). Но даже в случае, если число точек равно числу параметров, эксперимент нельзя считать удовлетворительным, поскольку нет возможности проверить, является ли модель правильной и не было ли одно из
измерений ошибочным (нельзя проводить прямую и по двум точкам!). Универсального правила по выбору количества точек нет, но для
определения параметров прямой рекомендуется иметь не менее 8--10 точек. 

Экспериментальные \textquote{точки} должны по возможности равномерно заполнять весь диапазон измерений. При этом в~областях резкого изменения функции точки должны лежать по возможности чаще.
В условиях ограниченного времени не стоит многократно проводить измерения в одной и той же точке, пытаясь уменьшить её погрешность --- лучше измерить больше разных точек, поскольку при последующей аппроксимации накопленная информация по разным точкам все равно будет просуммирована.

Важно отметить, что часто параметры установки \textquote{плывут} (\textquote{дрейфуют})
во время проведения эксперимента, поэтому рекомендуется при измерениях
% вместо того, чтобы делать несколько измерений с одними настройками подряд,
делать проходы в одну и в другую сторону по всему диапазону значений.

\subsection{Измерения}

Результаты измерений и сопутствующих вычислений должны быть представлены
в \emph{таблицах}.
Таблицы должны иметь подписи с кратким описанием
их содержания и, возможно, с пояснениями по структуре расположения
данных. Заглавные столбцы (или строки) должны быть подписаны, в них
должны быть указаны буквенные обозначения величин (введенные в тексте
ранее) и их размерность.

При записи результатов измерений фиксируются \emph{непосредственные
показания прибора} --- без какого либо пересчёта единиц измерения,
округления и т.п. В частности, если прибор имеет шкалу, записывается
\emph{число делений} отклонения стрелки, и отдельно --- цена деления
(в отдельном столбце или перед таблицей).
Пересчёт в физические единицы с учётом цены деления производится позже
при обработке. Это позволяет минимизировать ошибки при снятии показаний.

Полезно строить предварительные графики (прямо в экспериментальном журнале)
зависимостей измеряемых величин по мере получения результатов.
При этом сразу выделяются области резких изменений,
в которых измерения должны проводиться подробнее (больше точек),
чем на участках плавного изменения. Если изучаемая закономерность,
например линейная, выполняется только на некотором участке,
то область измерений должна быть выбрана шире этого участка, чтобы можно было
установить границы работы закономерности.

% Если в начале работы выясняется, что разброс результатов измерений
% очень большой, то иногда лучше поискать и устранить причину этого,
% чем выполнять большое количество измерений для получения необходимой
% точности результата.

\subsection{Расчёты, анализ и представление результатов}

Полученные первичные результаты используются для расчёта значений искомых 
величин, параметров зависимостей и их погрешностей. Результаты
обработки данных и их анализ оформляются в виде отдельного \emph{отчёта} о работе
(см. п.~\ref{sec:report}).

Отчёт не должен быть столь же подробен как журнал с точки зрения деталей проведения эксперимента. Также в отчёте можно опустить прямые результаты измерений --- достаточно дать ссылки на соответствующие страницы журнала.

Для измеряемых величин окончательные результаты должны быть представлены
в виде среднего значения, погрешности и количества проведённых измерений.

Для окончательной оценки качества результатов необходимо
сравнить их с данными, приводимыми в справочниках.

\note{Совпадение или несовпадение измеренного значения со справочным
    не может считаться критерием правильности проведения работы.
    Во-первых, значения действительно могут отличаться. Материалы,
    используемые в лабораторных работах, не всегда являются чистыми и
    соответствуют справочнику. Во-вторых, могут быть объективные причины,
    по которым результаты разошлись. Поиск и объяснения этих причин является
    более важным, чем точное совпадение значений!}

\section{Анализ инструментальных погрешностей}

Перед выполнением любого эксперимента необходимо предварительно проанализировать
возможные погрешности используемых приборов. Они могут
иметь как систематический, так и случайный характер. Можно говорить
о единой оценке \emph{инструментальной погрешности} прибора
$\sigma_{\text{инстр}}$, которая учитывает обе составляющие.

\paragraph{Погрешность шкалы.}
При работе с приборами \emph{со шкалой} (линейка, штангенциркуль, стрелочные
приборы и т.д.) один из источников погрешности~--- необходимость
выбора некоторого значения (интерполяции) между метками шкалы. Эта
погрешность, которую как правило оценивают в \emph{половину цены деления},
называется \emph{погрешностью отсчёта} по шкале. Аналогичная погрешность
есть и у приборов с цифровым дисплеем --- это погрешность
округления цифры последнего разряда. Данная погрешность может быть
как случайной, так и систематической: в частности, если показания
прибора стабильны (стрелка не дрожит и при повторных измерения стрелка
попадает в то же самое место шкалы), ошибка отсчёта будет систематической;
если стрелка дрожит (или \textquote{плавает} последняя цифра разряда),
ошибка будет случайной.

\note{Стоит по возможности избегать измерений в начале
шкалы: если измеряемая величина лишь немногим превосходит цену деления (или
единицу последнего разряда дисплея), относительная ошибка измерения резко
возрастает.}

\paragraph{Паспортная погрешность.}
Любой прибор имеет погрешность изготовления, калибровки, а также внутренние
источники ошибок (например, шумы). Как правило, максимальные значения
этих погрешностей определяются производителем и описаны в паспорте
прибора. Погрешности могут зависеть от условий эксплуатации (температура,
влажность и т.д.), что также должно отражаться в паспорте.

\example{Согласно паспорту
    вольтметра В7--34, его относительная погрешность при работе на пределе
    измерений 1~В, оценивается по формуле
    \[
\varepsilon_{x}=\left[0{,}015+0{,}002\left(\frac{1\;\text{В}}{U_{x}}
-1\right)\right]\cdot\bigr[1+0{,}01\cdot|t-20|\bigl],
    \]
    где $U_{x}$~{[}В{]} --- значение измеряемой величины,
    $t$ {[}$^{\circ}\mathrm{C}${]}~--- комнатная температура.
    Если измерения проводятся при температуре $24\;^{\circ}\mathrm{C}$
    и прибор показывает напряжение $U_{x}=500\;\text{мВ}$, то относительная
    погрешность равна $\varepsilon\approx1{,}7\%$, а абсолютная $\delta
U\approx\pm8\;\text{мВ}.$}

Для стрелочных приборов традиционно используется понятие \emph{класса
точности}. Предельная инструментальная погрешность равна произведению
класса точности (в процентах) на показание прибора при максимальном
отклонении стрелки. В цифровых приборах погрешность, как правило,
зависит от диапазона измерения, поэтому понятие класса точности для
них не применяется.

\example{Стрелочный вольтметр
    имеет диапазон измерения от 0 до 5~В и цену деления $10$~мВ, а
    его класс точности равен $0{,}5$. Следовательно, погрешность измерения,
    гарантируемая производителем, составляет $5\;\text{В}\cdot0{,}5\%=25$~мВ.
    Хотя цена деления меньше, в качестве погрешности следует взять именно
    $\pm25$ мВ. Не стоит рассчитывать на высокую точность при измерениях
    напряжения менее 1~В, поскольку относительная ошибка составит более 2,5\%.}

\paragraph{Сложение погрешностей.}
Наконец, при считывании показаний стрелка прибора или цифры на циферблате
могут \textquote{дрожать} (\emph{флуктуировать}) вблизи некоторого значения.
Это может быть связано как с разного рода шумами и помехами внутри прибора,
так и с колебаниями самой измеряемой величины. Если записывается некоторое
среднее значение показаний, то амплитуда флуктуаций должна быть учтена как
дополнительная случайная погрешность.

Не стоит также забывать, что в процессе эксперимента почти наверняка
возникнут дополнительные погрешности, связанные с конкретной постановкой
опыта и методикой измерений. Для нахождения результирующей погрешности
измерения необходимо сложить все \emph{независимые} источники ошибок
среднеквадратичным образом:
\[\sigma_{\text{полн}}=\sqrt{\sigma_{\text{инстр}}^{2}+
\sigma_{\text{отсч}}^{2}+\sigma_{\text{случ}}^{2}+\ldots}.\]

%\note{Отметим, что цену деления шкалы
%    или разрядность дисплея \emph{добросовестный} производитель
%    выбирает таким образом, чтобы погрешность отсчёта и погрешность самого
%    прибора были согласованы. В таком случае погрешность отсчёта по шкале
%    отдельно учитывать не нужно --- она уже учтена производителем
%    при расчёте инструментальной погрешности.}

\section{Отчёт о работе}\label{sec:report}

Лабораторная работа студента --- миниатюрное научное исследование.
Настоящие требования основаны на общепринятых стандартах научных публикаций,
упрощенных для студентов младших курсов.

Отчёт о проделанной лабораторной работе должен представлять собой
целостный документ, позволяющий читателю получить максимально полную
информацию о проделанной работе и полученных результатах ---
без каких-либо дополнительных пояснений со стороны студента.

Материал в отчёте должен излагаться последовательно, а сам отчёт должен
быть структурирован по разделам. Отчёт, как правило, содержит разделы:
1)~аннотация, 2)~теоретические сведения, 3)~методика измерений,
4)~используемое оборудование, 5)~результаты измерений и обработка данных,
6)~обсуждение результатов, 7)~заключение. Структура и названия разделов
могут незначительно варьироваться в зависимости от конкретного содержания
работы.

% Comment[Nozik] тоже вредно
% Начальные разделы отчёта должны быть подготовлены \emph{до проведения
% эксперимента} (при подготовке к работе). Непосредственно ход эксперимента
% должен фиксироваться в отдельном лабораторном журнале студента. 
Записи лабораторного журнала прикрепляются к отчёту в качестве приложения.
Допускается ведение лабораторного журнала и оформление отчётов в одной
рабочей тетради (формата А4).

% Результаты измерений и сопутствующих вычислений должны быть представлены
% в \emph{таблицах}. Таблицы должны иметь подписи с кратким описанием
% их содержания и, возможно, с пояснениями по структуре расположения
% данных. Заглавные столбцы (или строки) должны быть подписаны, в них
% должны быть указаны буквенные обозначения величин (введенные в тексте
% ранее) и их размерность.

Размерность измеренных величин --- как в таблицах, так и
на графиках --- должна быть подобрана так, чтобы данные
были удобны для чтения и \emph{не содержали избыточное количество
нулей}.

Помимо таблиц и графиков в тексте отчёта также должны быть представлены
промежуточные результаты обработки данных (с соответствующими погрешностями),
указаны используемые методы обработки данных и приведены соответствующие
формулы. Окончательные и наиболее важные промежуточные результаты
должны быть записаны с указанием погрешности (как абсолютной, так
и относительной) и округлены согласно принятым в физике правилам округления
(см. ниже).

\subsection{Требования к содержанию разделов}
\begin{description}\small
\item [{Аннотация:}] краткое (1--2 абзаца) описание работы: её
цели, используемые методы и приборы, ожидаемые результаты.
\item [{Теоретические~сведения:}] краткий обзор основных понятий и теоретических
законов, используемых или проверяемых в работе; упрощения и предположения,
используемые при анализе и интерпретации результатов эксперимента;
основные расчётные формулы.
\item [{Методика~измерений:}] схема и описание экспериментальной установки;
краткое описание основных методик проведения эксперимента, получения
и обработки экспериментальных данных.
\item [{Используемое~оборудование:}] перечень измерительных приборов,
используемых в работе; инструментальные погрешности приборов и предварительный
анализ их влияния на результаты опыта.
\item [{Результаты~измерений~и~обработка~данных:}] результаты проведенных
измерений в форме таблиц и графиков; промежуточные и окончательные
расчёты, в том числе расчёт погрешностей полученных результатов.
\item [{Обсуждение~результатов:}] анализ точности проведённых измерений
и достоверности результатов; обсуждение применимости использованных
теоретических предположений; сравнение результатов с табличными (справочными)
данными или результатами других экспериментов; обсуждение возможных
причин ошибок и способов их устранения.
\item [{Заключение~(или~выводы):}] краткое резюме по результатам эксперимента:
что удалось или не удалось измерить, были ли достигнуты поставлены
цели, выводы по результатам работы и т.п.
\end{description}

\subsection{Правила округления}\label{subsec:round}

Запись числовых значений, полученных в результате измерений, отличается
от стандартной записи чисел, принятой в арифметике или в бухгалтерской
отчётности. При десятичной записи результата важно следить за тем,
какие цифры соответствуют реально измеренным в эксперименте, а какие
возникли исключительно в результате математических операций и находятся
за пределами точности опыта.

Все цифры, начиная с первой ненулевой, называют \emph{значащими}.
Для корректной записи результата необходимо следить, чтобы количество
значащих цифр было согласовано с погрешностью измерения. Перечислим
правила, которыми необходимо руководствоваться при записи результатов:
\small
\begin{itemize}
\item последняя цифра записи результата измерения должна соответствовать
тому же разряду, что и последняя цифра в погрешности:
\end{itemize}
\noindent%
\begin{tabular}{llll}
    \color{red}неправильно:  &
    \color{red}$1{,}245\pm0{,}05$  &
    \color{red}$5{,}2\pm0{,}36$  &
    \color{red}$1{,}24\pm0{,}012$\\
правильно:  & $1{,}25\pm0{,}05$  & $5{,}2\pm0{,}4$  & $1{,}240\pm0{,}012$
\end{tabular}
\begin{itemize}
\item величина погрешности имеет характер сугубо статистической \emph{оценки}
и практически \emph{не может быть определена с точностью лучше} 20\%.
Поэтому погрешность нужно округлять до \emph{одной--двух
значащих цифр}. Как правило, если последняя цифра в погрешности единица
или двойка, в погрешности оставляют две значащие цифры, в остальных
случаях --- одну:
\end{itemize}
\noindent%
\begin{tabular}{llll}
\color{red}неправильно:  &
\color{red}$5{,}27\pm0{,}86$  &
\color{red}$1{,}236\pm0{,}137$  &
\color{red}$1\pm0{,}239$\\
правильно:  & $5{,}3\pm0{,}9$ & $1{,}24\pm0{,}14$ & $1{,}0\pm0{,}2$ или\\
 &  &  & $1{,}00\pm0{,}24$
\end{tabular}\par
\medskip
Величину $\pm0{,}14$ не следует округлять до $\pm0{,}1$, так как
при этом значение изменяется на 40\%.
\begin{itemize}
\item Ноль на конце десятичного числа является значащей цифрой. Запись
$l=1{,}4$~м
не эквивалентна $l=1{,}40$ м, т.\,к. последняя подразумевает в 10
раз большую точность измерения. Например, не эквивалентны записи $m=1$~т
и $m=1000$~кг, так как в первом случае одна значащая цифра, а во
втором четыре.
\item При необходимости нужно пользоваться\emph{ научной} (или
\emph{экспоненциальной})
формой записи числа, подбирая наиболее удобные единицы измерения.
Например, если длина объекта определена с точностью $\pm5$~см и
составляет $l=123\pm5$~см, то волне допустимы также записи:
$l=1{,}23\pm0{,}05$~м,
или $l=\left(12{,}3\pm0{,}5\right)\cdot10^{-1}\;\text{м}$, или
$l=\left(1{,}23\pm0{,}05\right)\cdot10^{3}\text{ мм}$,
и т.п. Не вполне корректно было бы написать $l=1230\pm50$~мм, поскольку
такая запись подразумевает превышение точности как в измеренной величине,
так и в оценке погрешности.
\item Если погрешность физической величины не указана, то по умолчанию
подразумевается,
что она измерена с точностью до изменения последней значащей цифры
на единицу. Например, запись $l=1{,}23$~м эквивалентна $l=1{,}23\pm0{,}01$
м или $l=123\pm1$~см, но не эквивалентна $l=1230$~мм.
\end{itemize}

\note{Все рассуждения в данном разделе относятся к представлению
    результатов в \emph{отчёте}. При проведении измерений и ведении лабораторного журнала не следует проводить никаких округлений, а напротив записывать всю доступную информацию.}

%Для избежания ненужных ошибок округления в них следует сохранять хотя бы одну \textquote{лишнюю} значащую цифру.
%Также следует пользоваться только инженерными/научными калькуляторами, не имеющими ограничений по разрядности, и специализированными программами
%для обработки экспериментальных и статистических данных.

\section{Построение графиков}
\label{sec:graph}

\normalsize

Пусть между двумя величинами $x$ и $y$ предполагается некоторая
функциональная зависимость. Измеряя пары значений ($x_{i},\,y_{i}$),
получим набор из $n$ результатов --- экспериментальных \textquote{точек}
\[
\left\{ (x_{1},\,y_{1}),\,(x_{2},\,y_{2}),\,\ldots\,,(x_{n},\,y_{n})\right\},
\]
которые изобразим на графике. Каждое измерение $\left(x_{i},y_{i}\right)$
имеет свою погрешность (случайную и/или систематическую)
$\delta x_i$ и $\delta y_i$. На графиках погрешности принято изображать в виде \textquote{крестов} размером $\pm\delta x$ по горизонтали и 
$\pm\delta y$ по вертикали.

Рассмотрим простейший случай, когда зависимость предполагается линейной:
$y=kx+b$. Из-за случайных погрешностей при $n>2$
будет невозможно провести прямую, проходящую через все экспериментальные точки. 
Можно, тем не менее, попробовать провести \textquote{наилучшую}
прямую, проходящую максимально близко ко всем точкам. В математической
статистике такую процедуру называют также \emph{линейной регрессией}.

\begin{figure}[th]
\begin{minipage}[t]{0.5\columnwidth}%
\includegraphics[width=1\linewidth]{images/graph1.pdf}%
\end{minipage}%
\begin{minipage}[t]{0.5\columnwidth}%
\includegraphics[width=1\linewidth]{images/graph2.pdf}%
\end{minipage}
\caption{Графический метод проведения прямой и оценки погрешностей}
\end{figure}

Самый простой и грубый метод --- провести наилучшую прямую
\textquote{от руки}. Этот метод, конечно, нестрогий,
но весьма наглядный. На практике к нему приходится часто прибегать
для грубой и быстрой оценки промежуточных результатов. Для этого нужно
приложить прозрачную линейку к графику так, чтобы по возможности кресты
всех экспериментальных точек находились максимально близко к проводимой
линии, а по обе стороны от неё оказалось примерно одинаковое количество
точек.

% \todo[author=ppv, inline]{Добавить отсылку к месту, где разбираются
% математические методы
% построения прямой}

Построив таким образом \textquote{наилучшую} прямую,
можно найти её параметры: угловой коэффициент $k$ и вертикальное
смещение $b$. Этим же способом можно грубо оценить ошибку определения
$k$ и $b$. Смещая линейку вертикально в пределах крестов погрешностей,
оценим погрешность $\delta b$. Аналогично, изменяя наклон линейки
относительно условного \textquote{центра масс}
экспериментального графика, получим оценку для погрешности углового
коэффициента $\delta k$. Если известно, что погрешности экспериментальных
точек $\left(\delta x,\,\delta y\right)$ имеют преимущественно случайный
характер, результат стоит разделить на корень из числа точек:
$\sigma_{k}\approx\delta k/\sqrt{n}$,
$\sigma_{b}\approx\delta b/\sqrt{n}$ (для систематических погрешностей
так делать не стоит).

Эта же процедура позволяет проверить, является ли измеренная зависимость
в самом деле линейной: прямая должна пересекать большую часть (хотя
бы 2/3) крестов погрешностей. В противном случае можно предполагать
существенное отклонение экспериментальной зависимости от линейной
теоретической. Отметим, что если кресты погрешностей на графике не
отмечены, такой анализ провести затруднительно.

Существуют и аналитические методы подбора параметров (см. гл.~\ref{ch:estimate}),
минимизирующие отклонения экспериментальных точек от некоторой теоретической зависимости
(например, \emph{метод наименьших квадратов}). Студентам первого курса
рекомендуется осваивать их постепенно, по мере накопления опыта экспериментальной
работы.

\begin{figure}
    \centering
    \includegraphics[width=0.6\linewidth]{errorbars.pdf}
    \caption{Влияние величины погрешности на интерпретацию результата}
    \label{fig:graph-method}
\end{figure}

\example{На рис. \ref{fig:graph-method} изображены одни и те же
    экспериментальные точки при разных погрешностях измерений,
график \ref{fig:graph-method}а, несомненно,
указывает на нерегулярный ход изучаемой зависимости
(кривая линия). Те же данные при больших погрешностях опыта
(рис. \ref{fig:graph-method}б) успешно описываются прямой линией.
Без указания крестов погрешностей разделить эти два случая было
бы невозможно.}

\paragraph{Нелинейные зависимости.}

Если теория предсказывает \emph{нелинейную} функциональную зависимость
между величинами, часто можно сделать \emph{замену переменных} так,
чтобы результирующий график получался линейным.

Заметим, что аналитические методы позволяют подбирать параметры и для нелинейных
зависимостей. Хотя готовых формул для общего случая не существует, задача
легко решается численно --- и в большинстве современных программ обработки
данных это сделать не сложнее, чем построить наилучшую прямую.
Построение прямой является наиболее наглядным и позволяет проверить
\textquote{разумность} полученных результатов, сверив их с построением
\textquote{от руки}.

%Однако в учебной лаборатории такой подход использовать не рекомендуется.

\example{Высота и время падения
груза без начальной скорости в поле тяжести связаны соотношением
$y=\frac{1}{2}gt^{2}+y_{0}$.
Для того, чтобы получить линейную зависимость, можно построить график
в координатах $\left(y,t^{2}\right)$. По угловому коэффициенту наилучшей
прямой можно в таком случае вычислить ускорение свободного падения:
$k=\frac{1}{2}g$.}

\example{В термодинамике и химии часто встречается зависимость
    вида $y=Ce^{-a/x}$. Чтобы определить коэффициенты $C$ и $a$,
    можно построить график в координатах $(u,v)$, где $u=\ln y$ и
    $v=\frac{1}{x}$. В таком случае, как нетрудно видеть, $u=-av+\ln C$.}

%Допускается использования программных пакетов для обработки линейных и нелинейных зависимостей при условии, что студент понимает все детали 
%применяемой процедуры обработки.

\subsection{Рекомендации по оформлению графиков}

Основная цель использования графиков --- \emph{наглядность}
отображения результатов. В связи с этим к ним
предъявляются следующие требования:
\begin{itemize}
    \small
    \item график обязательно должен иметь подпись (заглавие) с кратким описанием его
содержания (графики --- это первое, на что обращает внимание читатель, еще до прочтения текста отчёта!);
\item подписи, данные и линии не должны быть нагромождены друг на друга
так, что препятствовало бы их чтению;
    \item оси на графике должны быть \emph{подписаны}: указаны буквенное
обозначение
величины и её единицы измерения; если величина безразмерна, указывается
\textquote{отн. ед.} (относительные единицы);
    \item на осях должны быть отмечены \emph{масштаб} и \emph{положение нуля};
масштаб обозначается несколькими отметками с подписанными значениями
и дополнительными малыми отметками без подписей; масштаб должен быть
удобным для чтения (использованы \textquote{круглые}
числа, делящиеся на 10, 5 или 2);
    \item масштаб осей и начало отсчёта должны быть выбраны так, чтобы
экспериментальные
данные занимали всю площадь листа, отведённую под график;
    \item если график строится не \textquote{от нуля}, это
следует подчеркнуть отдельно, например \textquote{разрывом}
оси;
    \item при необходимости сравнения данных из разных серий измерений, их
следует
размещать на одном графике, обозначая их разными символами или цветами;
    \item график с несколькими сериями данных должен быть снабжен \textquote{легендой},
в которой указано соответствие серий данных и их обозначений; экспериментальные
\textquote{точки} должны изображаться \emph{символами
конечных размеров} (позволяющими отличить их от случайных \textquote{пятен});
    \item точки не должны быть без необходимости соединены линиями; также не
нужно подписывать положение каждой точки графика (при необходимости
можно указать положение 1-2 \emph{особых} точек, если это не загромождает
график);
    \item все экспериментальные точки должны быть снабжены \emph{крестами
погрешностей},
размер которых соответствует инструментальной погрешности измерения
соответствующей величины (либо вычисленной по результатам косвенных
измерений); кресты погрешностей можно не отмечать, только если погрешности
малы (настолько, что они не будут видны на графике) или не известны;
    \item если теория предполагает некоторую (например, линейную) функциональную
зависимость, на график должна быть тонкой линией нанесена соответствующая
теоретическая кривая; расчёт параметров этой кривой (например, коэффициентов
МНК для линейной зависимости) должен проводиться отдельно в тексте
отчёта --- с указанием используемых методов и формул; результаты
таких расчётов и их погрешности указываются в легенде графика или
в подписи к нему;
    \item оптимальный размер графика --- от четверти до половины страницы
(при условии, что отчёт оформляется на страницах формата А4).
\end{itemize}

На рис.~\ref{fig:incorrect} приведён пример того, как \emph{не надо}
строить графики. В нём собраны наиболее типичные ошибки, совершаемые
студентами. Предлагаем читателю выявить их самостоятельно. Для сравнения
на рис.~\ref{fig:correct} изображён график для \emph{тех же данных},
выполненный с соблюдением изложенных выше указаний.
\begin{figure}[ht]
\begin{centering}
\includegraphics[width=9cm]{images/bad.png}
\par\end{centering}
\caption{\label{fig:incorrect}Пример неправильно построенного графика}
%     (график построен с использованием электронных таблиц LibreOffice Calc)
\end{figure}
\begin{figure}[ht!]
\begin{centering}
\includegraphics[width=8.5cm]{images/good.pdf}
\par\end{centering}
\caption{\label{fig:correct}Пример корректного графического представления
    данных}
% (график построен с использованием специализированного приложения
% для анализа и визуализации данных QtiPlot)
\end{figure}


\section{Некоторые типичные ошибки обработки данных}

\paragraph{Нахождение углового коэффициента по среднему от частного.}

Студент измеряет сопротивление резистора по зависимости $U\!\left(I\right)$.
Получив некоторое количество экспериментальных точек
$\left(I_{i},\,U_{i}\right)$,
и пользуясь законом Ома $R=U/I$, он вычисляет сопротивление для каждого
измерения $R_{i}=U_{i}/I_{i}$, а затем определяет сопротивление резистора
как среднее значение $R_{0}=\left\langle R_{i}\right\rangle =\frac{1}{n}\sum
R_{i}$.
Что не так с этим методом (результат-то получается вполне \textquote{разумным})?

\begin{longnote}
    Во-первых, применять процедуру усреднения можно только
    при повторении \emph{одного и того же} измерения. В данном случае значения
$R_{i}$
    относятся к \emph{разным} измерениям, так как параметры системы каждый раз
изменялись.
    Во-вторых, не была проверена линейность зависимости $U\left(I\right)$, то
есть
    справедливость закона Ома (существуют и нелинейные элементы,
    для которых он не выполняется). В-третьих, даже если зависимость
    можно считать линейной, может оказаться так, что она не проходит через
    ноль (например, из-за сдвига нуля у вольтметра или амперметра) ---
    тогда формула $R_{i}=U_{i}/I_{i}$ не годится.
    И наконец, даже если выполнена линейность и зависимость проходит через
    ноль, вычисление таким способом чревато большими погрешностями. Нетрудно
    видеть, что среднее значение $\left\langle U_{i}/I_{i}\right\rangle $
    по сути есть среднее тангенсов углов наклона линий, проведённых из
    начала координат в экспериментальную точку. Как известно,
    функция $\tg x$ при $x>\pi/4$ очень резко возрастает (и стремится
    к бесконечности при $x=\pi/2$). В таком случае даже небольшое \textquote{шевеление}
    экспериментальной точки, особенно если она находится достаточно близко
    к оси ординат, может привести к резкому увеличению вклада этой точки
    в итоговый результат.

    Таким образом, \textquote{разумный} результат студента --- плод удачного стечения
многих
    обстоятельств. Правильный --- обоснованный и надёжный --- алгоритм
    нахождения сопротивления: построить график $U\left(I\right)$, убедиться
    в его линейности, и построить наилучшую прямую. 
    Угловой коэффициент этой прямой и будет наилучшей оценкой 
    для сопротивления резистора.
    \todo[inline]{Сделать рисунок}
\end{longnote}


\paragraph{Недооценка систематической погрешности.}
Студент измеряет сопротивление резистора, действуя по правильному
алгоритму, описанному выше. При измерениях используются вольтметр
и амперметр с классом точности $0{,}5$. Получив большое число экспериментальных
точек и построив наилучшую прямую методом наименьших квадратов (формула
(\ref{eq:MNK})), студент находит сопротивление (например, $R=5{,}555$~Ом)
и его погрешность по формуле (\ref{eq:MNK_sigma_k}), которая оказывается
равна $\sigma_{R}=0{,}003\;\text{Ом}.$ Окончательный результат измерения
записывается как $R=5{,}555\pm0{,}003\;\text{Ом}$.

Выходит так, что с помощью приборов, относительная погрешность которых
составляет $0{,}5\%$, получен на порядок более точный результат
$\varepsilon\approx0{,}05\%$.
Возможно ли такое?

\begin{longnote}
Ситуация эта вполне реальна и встречается в учебной лаборатории довольно часто.
Дело в
том, что метод наименьших квадратов позволяет оценить \emph{только
случайную ошибку} --- и она в самом деле может оказаться довольно мала. Однако
учебные приборы далеки от совершенства и их ошибка имеет в основном
\emph{систематический} характер. Поэтому в данном случае при записи конечного
результата
необходимо учесть систематическую ошибку, относительная величина которой по
составляет
не менее $0{,}5$\% (согласно классу точности прибора) и значительно превосходит
случайную. Результат эксперимента стоило бы записать как
$R=5{,}55\pm0{,}03\;\text{Ом}$.

Всё же, может статься и так, что инструментальные
ошибки наших вольтметра и амперметра имеют случайный характер, и мы
в самом деле добились кратного повышения точности за счёт многократных
повторений измерений (сколько нужно измерений, чтобы увеличить точность
на порядок?). Это нетрудно проверить, если заменить вольтметр или
амперметр на аналогичный и повторить опыты. Таким образом мы превратим
систематическую ошибку \emph{одного} прибора в случайную ошибку
\emph{множества}
приборов. Если отклонение нового результата от исходного значительно
превысит величину $\pm0{,}003$~Ом, гипотеза о случайности инструментальной
ошибки будет опровергнута, то есть погрешность отдельного прибора
действительно имеет в основном систематический характер.\par
\end{longnote}

\paragraph{Аппроксимация полиномом.}
Студент получает набор экспериментальных данных $\left\{ x_{i},y_{i}\right\} $,
которые по теории должны ложиться на прямую. Нанеся точки на график,
студент видит, что на прямую они ложатся не очень хорошо (см.
рис.~\ref{fig:approx}а).
Для того, чтобы точки лучше ложились на график, студент решает использовать
функцию аппроксимации полиномом (например, квадратичным), имеющуюся
в наличии во всех электронных таблицах и программах обработки данных.
Можно ли так делать?

\begin{figure}[h]
\begin{minipage}[t]{0.49\columnwidth}%
\begin{center}
\includegraphics[width=1\linewidth]{images/x2.pdf}
\par\end{center}
\begin{center}
а)
\par\end{center}%
\end{minipage}%
\begin{minipage}[t]{0.49\columnwidth}%
\begin{center}
\includegraphics[width=1\linewidth]{images/x2b.pdf}
\par\end{center}
\begin{center}
б)
\par\end{center}%
\end{minipage}
\caption{\label{fig:approx}Примеры аппроксимации данных: а)~сплошная линия
--- линейная аппроксимация, пунктир --- квадратичная
аппроксимация; б)~сплошная линия --- линейная аппроксимация
по нескольким начальным точкам, пунктир --- аппроксимация
полиномом высокой степени.}
\end{figure}

\begin{longnote}
    Ответ на вопрос зависит от того, какую цель преследует
    студент. Если цель --- добиться того, чтобы \textquote{точки лучше ложились на график}, то всё сделано правильно.
    Если говорить \textquote{по-научному} ---
    это попытка решить задачу \emph{интерполяции}
    экспериментальных данных: по ограниченному набору $\left\{
x_{i},\,y_{i}\right\} $
    получить аналитическую функцию $y=f\!\left(x\right)$, позволяющую
    рассчитывать значения $y$ при произвольном $x$. В учебном практикуме
    это может быть, например, задача построения калибровочного графика.
    В таком случае можно лишь дать практический совет --- стараться
    использовать для интерполяции полином \emph{как можно
    меньшей степени} (дело в том, что при слишком большой
    степени функция почти наверняка станет сильно немонотонной, а это
    вовсе не то, что хочется иметь в качестве результата интерполяции,
    см. рис.~\ref{fig:approx}б).

    Однако ни в одной лабораторной работе курса общей
    физики задача интерполяции не является основной целью работы! Как
    правило, цель --- проверить теоретические закономерности
    и измерить физические характеристики системы. Если нет теории,
предсказывающей
    и объясняющей квадратичную зависимость, проделанная студентом процедура
    \emph{бесполезна}, поскольку вычисленным коэффициентам полинома
невозможно 
    приписать физический смысл.

    Правильно было бы в такой ситуации выявить, на каком
    участке зависимости линейный закон \emph{выполняется},
    оценить его границы, и по нему построить наилучшую прямую (сплошная
    линия на рис.~\ref{fig:approx}б). Для участка, отклоняющегося от
    предсказываемой линейной зависимости, следует теоретически проанализировать
    причины отклонения и по возможности предложить уточнение теории. Возможно,
    стоит ожидать не квадратичное, а кубическое отклонение? Различить
    их на ограниченном наборе данных с большими погрешностями невозможно!
    Имея достаточное количество точек, предложенную теорию можно проверить
    и лишь после этого аппроксимации более сложной функцией можно придать
    физический смысл.
\end{longnote}


\chapter{Приложение}

\small

\section{Корреляции}
\label{sec:correlation}

Напомним, если величины $x$ и $y$ независимы, то среднее значение (математическое ожидание) произведения отклонений $\Delta x = x-\limaverage{x}$ и $\Delta y = y - \limaverage{y}$
равно нулю:
\begin{equation*}
\limaverage{\Delta x\cdot\Delta y}=\limaverage{\Delta x}\cdot\limaverage{\Delta
y}=0.
\end{equation*}
Если же $x$ и $y$ не являются полностью независимыми, среднее значение произведения
их отклонений может быть использовано как количественная мера их зависимости.
Наиболее употребительной мерой зависимости двух случайных величин
является \emph{коэффициент линейной корреляции}:
\begin{equation}
r_{xy}=\frac{\average{\Delta x\cdot\Delta
y}}{\sigma_{x}\cdot\sigma_{y}}.\label{eq:pearson}
\end{equation}
Нетрудно проверить (с помощью неравенства Коши\textendash Буняковского),
что $-1\le r\le1$. В частности, для полностью независимых величин
коэффициент корреляции равен нулю, $r=0$, а для линейно зависимых
$y=kx+b$ нетрудно получить $r=1$ при $k>0$ и $r=-1$ при $k<0$.
Примеры промежуточных случаев представлены на рис. TODO.

\note{Угловой коэффициент прямой в задаче линейной регрессии \eqref{eq:MNK} выражается
через коэффициент корреляции как
\begin{equation*}
k = r_{xy}\frac{\sigma_y}{\sigma_x}.
\end{equation*}}

Если коэффициент $r_{xy}$ близок к единице, говорят, что величины
\emph{коррелируют} между собой (от \emph{англ.} correlate ---
находиться в связи).

\todo[inline,author=ppv]{А не объяснить ли заодно, что чудный $R^2$, который выдаёт эксель,
это для прямой и есть $r^2$?}

\paragraph{Отсутствие корреляции $\protect\not\Rightarrow$ независимость.}

Отметим, что (\ref{eq:indep}) --- необходимое,
но не достаточное условие независимости величин. На рис. TODO приведён
пример очевидно зависимых $x$ и $y$, для которых $r\approx0$.

\paragraph{Корреляция $\protect\not\Rightarrow$ причинность.}

Ещё одна типичная ошибка --- исходя из большого
коэффициента корреляции ($r\to1$) между двумя величинами сделать
вывод о функциональной (причинной) связи между $x$ и $y$. Рассмотрим
конкретный пример. Между током и напряжением на некотором резисторе
имеет место линейная зависимость $U=IR$, и коэффициент корреляции
$r_{UI}$ действительно равен единице. Однако \emph{обратное
в общем случае неверно}. Например, ток в резисторе коррелирует
с его температурой $T$, $r_{IT}\to1$ (больше ток --- больше
тепловыделение по закону Джоуля\textendash Ленца), однако ясно, что
нагрев резистора извне не приведёт к повышению тока в нём (скорее
наоборот, так как сопротивление металлов с температурой растёт). Ошибка
отождествления корреляции и причинности особенно характерна при исследовании
сложных многофакторных систем, например, в медицине, социологии и
т.п.

\section{Свойства точечных оценок} \label{sec:point}
Если измеряется одна физическая величина $x$, то можно поставить задачу
по конечному набору данных $\mathbf{x}=\{x_i\}$ ($i=1\ldots n$) оценить параметры
случайного распределения, которому подчиняется $x$. В частности,
найти среднее значение (математическое ожидание)~$\limaverage{x}$ и
дисперсию~$\sigma^2$.

Если результатом оценки параметра является просто число~---
без указания интервала, в котором может лежать истинное значение,~---
такую оценку называют \emph{точечной}.
Пример точечных оценок дают формулы для выборочного среднего
\eqref{eq:average}:
\begin{equation}
\limaverage{x} \approx \average{x} = \frac{1}{n}\sum_i x_i\qquad
\label{eq:estimate_x_avg}
\end{equation}
и выборочной дисперсии \eqref{eq:sigma}:
\begin{equation}
\sigma^2 \approx s^2_n = \frac{1}{n}\sum_i (x_i - \average{x})^2.
\label{eq:estimate_x_sigma}
\end{equation}

% Точечная оценка сама по себе не имеет смысла с точки зрения физики, поскольку не
% позволяет определить погрешность результата. Поэтому для любого метода
% оценивания, применяемом в физике необходимо определить процедуру определения
% погрешности или интервала параметров с фиксированным вероятностным содержанием.

Оценка параметров должна давать правильное значение
хотя бы в пределе большого числа измерений.
Если при $n\to \infty$ оценка стремится к истинному значению параметра,
\[
\lim_{n\to \infty} \hat{\theta}(\mathbf{x}) \to \limaverage{\theta},
\]
то её называют \emph{состоятельной}.
Можно показать (см. \cite{idie}), что если у распределения,
которому подчиняется случайная величина,
существуют конечные средние и дисперсия, то оценки
\eqref{eq:estimate_x_avg}, \eqref{eq:estimate_x_sigma} являются состоятельными.

\paragraph{Несмещенные оценки.}
Рассмотрим случай малого числа измерений ($n\gtrsim 1$).
Тогда даже если оценка состоятельна, она может давать довольно большую ошибку.
При фиксированном $n$ функцию оценки $\hat{\theta}(x_1,\ldots,x_n)$
можно рассматривать как случайную величину с некоторым распределением,
отличающимся от распределения измеряемой величины.
Естественно потребовать, чтобы среднее (математическое ожидание) этого
распределения совпадало с истинным значением искомого параметра:
\[
\limaverage{\strut\hat{\theta}(\mathbf{x})} = \limaverage{\theta}.
\]
В таком случае оценку называют \emph{несмещённой}.

Нетрудно показать, что выборочное среднее \eqref{eq:estimate_x_avg}
является несмещённой оценкой. А вот оценка $s_n^2$ из
\eqref{eq:estimate_x_sigma} таким свойством не обладает. Математическое
ожидание для величины $s_n^2$ при фиксированном $n$ оказывается равно
$\limaverage{s_n^2} = \frac{n-1}{n} \sigma^2$ (предлагаем в качестве упражнения
проверить данное утверждение самостоятельно). Именно поэтому при малых $n$
для оценки дисперсии рекомендуется использовать формулу
\eqref{eq:sigma_straight}:
\[
\sigma^2 \approx s^2_{n-1} = \frac{1}{n-1}\sum_i (x_i - \average{x})^2.
\]

% \example{Рассмотрим формулу \eqref{eq:estimate_x_sigma} для среднеквадратичного
%     отклонения при $n=2$. Выборочное среднее равно
%     $\average{x}=\frac{x_{1}+x_{2}}{2}$, выборочная дисперсия:
%     \[
%     s_{x}^{2}=\frac{1}{2}\left[\left(x_{1}-\frac{x_{1}+x_{2}}{2}\right)^{2}+\left(x_{2}-\frac{x_{1}+x_{2}}{2}\right)^{2}\right]=\frac{1}{4}x_{1}^{2}-\frac{1}{2}x_{1}x_{2}+\frac{1}{4}x_{2}^{2}.
%     \]
%     Рассмотрим выражение для $s_x^2$ как случайную функцию двух случайных переменных
%     $x_1$ и $x_2$, и найдём её математическое ожидание
%     (то есть усредним $s_x^2$ по большому числу опытов, в каждом из которых
%     проведено по по $n=2$ измерений).
%     Учитывая, что $x_1$ и $x_2$~--- независимы
%     ($\limaverage{x_1x_2}=\limaverage{x}_1\cdot \limaverage{x}_2$), получим
%     \[
%     \limaverage{s_{x}^{2}}=\frac12 \limaverage{x^2}~--- \frac12 \limaverage{x}^2.
%     \]
%     Сравнивая с известной формулой
%     $\sigma^2 = \limaverage{x^2}~--- \limaverage{x}^2$,
%     видим, что среднее значение оценки отличается от истинного в 2 раза.
% }


\paragraph{Эффективность оценки.}
Для сравнения разных методов оценки очень важным свойством является
их \emph{эффективность}. На качественном уровне эффективность~--- величина,
обратная разбросу значений $\hat{\theta}(\mathbf{x})$ при применении к разным
наборам данных $\mathbf{x}$. Как обсуждалось выше, оценка $\hat{\theta}(\mathbf{x})$
есть случайная величина, подчиняющаяся некоторому, в общем случае неизвестному,
распределению. Среднее $\overline{\hat{\theta}(\mathbf{x})}$ по этому распределению
определяет смещение оценки.
А его дисперсия $\sigma^2\left(\hat\theta\right)$~--- как раз мера ошибки
в определении параметра. Выбирая между различными методами (минимума хи-квадрат, максимального правдоподобия, наименьших квадратов и т. д.), мы, естественно,
хотим, чтобы ошибка была минимальной. Разные статистические методы обладают
разной эффективностью и в общем случае при конечном $n$ величина
$\sigma^2\left(\hat\theta\right)$ никогда не будет равна нулю.

Теорема, устанавливающая максимальное значение эффективности оценки,
рассмотрена в п.~\ref{sec:rao}.
% Разумеется, встает
% вопрос о том, можно ли построить оценку с максимальной возможной
% эффективностью.


\section{Максимальная эффективность оценки (граница Рао--Крамера)}\label{sec:rao}

Максимальная эффективность оценки ограничена теоремой Рао--Крамера.
\paragraph{Утверждение.}
Пусть оценка $\hat\theta$ параметра $\theta$ является несмещённой, тогда
всегда выполняется неравенство:
\begin{equation}
  \sigma^2(\hat\theta) \geq \frac{1}{I(\theta)},
\end{equation}
где
\begin{equation}
    I(\theta) =
    \limaverage{\left(\frac{\partial \ln L}{\partial \theta}\right)^2}.
\end{equation}
Здесь $L(\mathbf{y},\,\theta)$~--- введённая в п.~\ref{sec:chi2} функция правдоподобия
(вероятность получить набор результатов $\mathbf{y}$ при заданном параметре
$\theta$). Функцию $I(\theta)$ также называют \emph{информацией Фишера}.

\paragraph{Доказательство в одномерном случае.}
Обозначим
\[
U \equiv \frac{\partial \ln L} {\partial \theta} =
\frac{1}{L}\frac{\partial L}{\partial \theta}
\]
и найдём математическое ожидание этой функции:
\[
\limaverage{U} =
\int U\cdot L  d\mathbf{y} =
\int \frac{\partial L}{\partial \theta} d\mathbf{y} =
\frac{\partial}{\partial \theta} \int{L d\mathbf{y}} = 0.
\]
Теперь рассмотрим ковариацию параметра $\theta$ и функции
$U$:
\begin{equation}
\limaverage{\hat\theta\cdot U} 
= \frac{\partial}{\partial \theta} \int{\hat\theta L d\mathbf{y}} =
\frac{\partial \limaverage{\hat\theta}}{\partial \theta}.
\end{equation}
Для несмещенных оценок математическое ожидание оценки параметра равно
самому значению параметра: $\limaverage{\hat{\theta}} = \theta$,
поэтому последнее выражение есть просто единица.
Согласно неравенству Коши--Буняковского имеем
\[
\sigma^2 (\hat\theta) \cdot \sigma^2 (U) \geq
\left| \limaverage{\hat\theta \cdot U} \right| = 1,
\]
откуда и следует сделанное утверждение.

\paragraph{Следствие.}
Максимальная эффективность достигается в том случае, если величины
$\hat\theta$ и $U$ \emph{коррелируют} друг с другом.
Оценка, максимизирующая функцию $L(\mathbf{y},\theta)$
(метод максимального правдоподобия), является \emph{состоятельной},
\emph{несмещенной}, кроме того совпадает с оценкой вида
$U(\mathbf{y}, \hat\theta) = 0$, а значит является \emph{максимально эффективной}.

% \section{Принцип максимального правдоподобия и наилучшая оценка среднего}
%
% Пусть при измерениях одно и той же величины два студента
% независимым образом получили результаты
% \[
% x_{1}\pm\sigma_{1}\qquad\text{и}\qquad x_{2}\pm\sigma_{2}.
% \]
% Можно ли как-то объединить их ответы и таким образом улучшить оценку
% измеряемой величины?
%
% Первое, что может прийти на ум --- найти среднее
% арифметическое $\average{x}=\frac{x_{1}+x_{2}}{2}.$ Однако нетрудно
% понять, что если, скажем, измерение 2 сильно хуже, чем 1 ($\sigma_{2}\gg\sigma_{1}$),
% то разумнее было бы значение $x_{2}$ вообще отбросить и использовать
% $x_{1}$ как \textquote{наилучшую} оценку.
%
% Предположим, что измеряемая величина имеет нормальное распределение
% с некоторым средним $x_{0}$. Вероятность получить значение $x_{1}$
% при первом измерении согласно (\ref{eq:normal}) пропорциональна величине
% \[
% P_{1}\propto\frac{1}{\sigma_{1}}e^{-\left(x_{1}-x_{0}\right)/2\sigma_{1}^{2}},
% \]
% вероятность получить значение $x_{2}$ при втором измерении:
% \[
% P_{2}\propto\frac{1}{\sigma_{2}}e^{-\left(x_{2}-x_{0}\right)^{2}/2\sigma_{2}^{2}}.
% \]
% Вероятность получить пару значений $\left\{ x_{1},\,x_{2}\right\} $
% пропорциональна произведению $P_{1}P_{2}$:
% \[
% P\left(x_{1},x_{2}\right)=P_{1}P_{2}\propto\frac{1}{\sigma_{1}\sigma_{2}}e^{-\left(x_{1}-x_{0}\right)^{2}/2\sigma_{1}^{2}-\left(x_{2}-x_{0}\right)^{2}/2\sigma_{2}^{2}}.
% \]
%
% Рассмотрим выражение, оказавшееся в показателе экспоненты:
% \[
% \chi^{2}=\left(\frac{x_{1}-x_{0}}{\sigma_{1}}\right)^{2}+\left(\frac{x_{2}-x_{0}}{\sigma_{2}}\right)^{2}.
% \]
% Назовём \emph{наилучшей} такую оценку параметра
% $x_{0}$, при котором полученные в опытах результаты имеют \emph{максимальную
% вероятность} ($P\to\mathrm{max}$). Такой подход, называемый
% \emph{принципом максимального правдоподобия}, используется
% во многих задачах статистики.
%
% Из полученного выше видно, что $P\to\mathrm{max}$, если
% величина $\chi^{2}$ (\emph{хи-квадрат}) будет иметь
% \emph{минимум}. Дифференцируя по $x_{0}$ и приравнивая
% результат к нулю, запишем
% \[
% \frac{d\chi^{2}}{dx_{0}}=-2\frac{x_{1}-x_{0}}{\sigma_{1}^{2}}-2\frac{x_{2}-x_{0}}{\sigma_{2}^{2}}=0,
% \]
% откуда найдём
% \begin{equation}
% x_{0}=\frac{w_{1}x_{1}+w_{2}x_{2}}{w_{1}+w_{2}},\qquad\text{где }w_{1,2}=\frac{1}{\sigma_{1,2}^{2}}.
% \end{equation}
%
% Таким образом, для вычисления \emph{наилучшей} (максимально правдоподобной) оценки
% среднего нужно вычислить \emph{взвешенное среднее} с весами, \emph{обратно
% пропорциональными квадратам соответствующих погрешностей}.
%
% Результат непосредственно обобщается на произвольное число измерений:
% \begin{equation}
% x_{\text{наил}}=\frac{\sum\limits _{i}w_{i}x_{i}}{\sum\limits _{i}w_{i}},\qquad w_{i}=\sigma_{i}^{-2}.
% \end{equation}

% \section{Метод максимального правдоподобия для построения наилучшей
% прямой\label{subsec:MMP}}
%
% При описании метода наименьших квадратов мы не обосновали,
% почему и в каком смысле именно этот метод является \textquote{наилучшей}
% оценкой для коэффициентов линейной регрессии $y=kx+b$. Кроме того,
% мы получили формулы только для частного случая, когда погрешности
% всех экспериментальных точек одинаковы: $\sigma_{y}=\mathrm{const}$.
%
% Рассмотрим более общий случай. Пусть по-прежнему погрешности
% по оси абсцисс малы, $\sigma_{x}\to0$, а погрешности по оси $y$
% различны для каждой точки и равны $\sigma_{y_{i}}$. Пусть теория
% предсказывает \emph{линейную}\footnote{Отметим, что метод легко обобщается и на нелинейные зависимости общего вида $y=f\left(x;a,b,\ldots\right)$. Хотя формулы
% получаются существенно более громоздкими, при вычислении на компьютере
% оперировать с ними не сложнее, чем с линейной регрессией. В учебном
% практикуме мы рекомендуем всегда делать замену переменных, сводящую
% теоретическую зависимость к линейной, поскольку проведение прямой
% наиболее наглядно и может быть в грубом приближении проделано просто
% \textquote{по линейке}.}
% зависимость $y=kx+b$.
%
% Отклонение точки от теоретической зависимости обозначим как
% \[
% \Delta y_{i}=y_{i}-\left(kx_{i}+b\right).
% \]
%
% Воспользуемся \emph{принципом максимального правдоподобия}
% и построим такую прямую, чтобы вероятность обнаружить наблюдаемые
% в опыте отклонения $\left\{ \Delta y_{i}\right\} $ от неё была максимальна.
%
% Обозначим вероятность отклонения на величину $\Delta y_{i}$
% при известном $\sigma_{y_{i}}$ как $P\!\left(\Delta y_{i};\sigma_{y_{i}}\right)$.
% Предположим, что ошибки измерения для всех экспериментальных точек
% можно считать \emph{случайными} и \emph{независимыми}.
% В таком случае вероятность отклонения для всех $n$ точек равна произведению
% вероятностей, так что метод максимального правдоподобия сводится к
% поиску максимума выражения
% \begin{equation}
% \prod\limits _{i=1}^{n}P\!\left(\Delta y_{i};\sigma_{y_{i}}\right)\to\mathrm{max}.\label{eq:MMP_general}
% \end{equation}
% Максимизация производится по параметрам аппроксимирующей функции (на
% нашем случае это $k$ и $b$).
%
% Рассмотрим частный случай, когда погрешности имеют \emph{нормальное}
% (гауссово) распределение (\ref{eq:normal}) (напомним, что нормальное
% распределение применимо, если отклонения возникают из-за большого
% числа независимых факторов, что на практике реализуется довольно часто).
% Тогда, поскольку гауссова функция распределения пропорциональна величине
% $\propto e^{-\Delta y^{2}/2\sigma^{2}}$, выражение (\ref{eq:MMP_general})
% достигает максимума, если минимальна сумма
% \begin{equation}
% \boxed{\chi^{2}=\sum_{i=1}^{n}\frac{\Delta y_{i}^{2}}{\sigma_{y_{i}}^{2}}\to\mathrm{min}}.\label{eq:chi2}
% \end{equation}
% Здесь мы ввели стандартное обозначение для такой суммы ---
% $\chi^{2}$ (\emph{хи-квадрат)}.
%
% Таким образом, задача построения наилучшей прямой сводится
% к минимизации суммы квадратов отклонений, нормированных на соответствующие
% дисперсии $\sigma_{y_{i}}^{2}$. Если все погрешности одинаковы, $\sigma_{y_{i}}=\mathrm{const}$,
% мы приходим к методу наименьших квадратов.
%
% Получим выражения для наилучших коэффициентов $k$ и $b$.
% Заметим, что сумма (\ref{eq:chi2}) является \emph{взвешенной}
% суммой квадратов отклонений с весами
% \begin{equation}
% w_{i}=\frac{1}{\sigma_{y_{i}}^{2}}.
% \end{equation}
%
% Можно определить \emph{взвешенное среднее} от
% некоторого набора значений $\left\{ x_{i}\right\}$ как
% \[
% \left\langle x\right\rangle ^{\prime}=\frac{1}{W}\sum_{i}w_{i}x_{i},
% \]
% где $W=\sum\limits _{i}w_{i}$ --- нормировочная константа.
% Далее в этом разделе штрих будем для краткости опускать.
%
% Потребуем, согласно (\ref{eq:chi2}), чтобы была минимальна
% сумма
% \[
% \sum\limits _{i=1}^{n}w_{i}\Delta y_{i}^{2}\to\mathrm{min}.
% \]
% Повторяя процедуру, использованную при выводе (\ref{eq:MNK}), можно
% получить совершенно аналогичные формулы для оптимальных коэффициентов:
% \begin{equation}
% \boxed{k=\frac{\left\langle xy\right\rangle -\left\langle x\right\rangle \left\langle y\right\rangle }{\left\langle x^{2}\right\rangle -\left\langle x\right\rangle ^{2}},\qquad b=\left\langle y\right\rangle -k\left\langle x\right\rangle },\label{eq:MMP}
% \end{equation}
% с тем отличием, что под угловыми скобками $\left\langle \ldots\right\rangle $
% теперь надо понимать усреднение с весами $w_{i}=1/\sigma_{y_{i}}^{2}$.
%
% Найденные формулы позволяют вычислить коэффициенты линейной
% регрессии, \emph{если} известны величины $\sigma_{y_{i}}$.
% Значения $\sigma_{y_{i}}$ могут быть получены либо из некоторой теории,
% либо измерены непосредственно (многократным повторением измерений
% при каждом $x_{i}$), либо оценены из каких-то дополнительных соображений
% (например, как инструментальная погрешность).

\section{Погрешности коэффициентов построения прямой}\label{sec:sigma_kb}

Проведём подробный вывод для погрешностей коэффициентов наилучшей
прямой $\sigma_{k}$ и $\sigma_{b}$. Воспользуемся общей формулой
(\ref{eq:sigma_general}) для погрешности косвенных измерений. Считая,
что величины $x_{i}$ известны точно, запишем для погрешности углового
коэффициента
\[
\sigma_{k}^{2}=\sum\limits _{i}\left(\frac{\partial k}{\partial y_{i}}\right)^{2}\sigma_{y_{i}}^{2}.
\]
Продифференцируем (\ref{eq:MMP}) по $y_{i}$:
\[
\frac{\partial k}{\partial y_{i}}=\frac{1}{D_{xx}}\frac{\partial}{\partial y_{i}}\left(\frac{1}{W}\sum w_{i}x_{i}y_{i}-\left\langle x\right\rangle \frac{1}{W}\sum w_{i}y_{i}\right)=\frac{w_{i}\left(x_{i}-\left\langle x\right\rangle \right)}{WD_{xx}},
\]
где $D_{xx}=\average{x^{2}} -\average{x}^{2}$, $W=\sum_i \sigma_{y_i}^{-2}$.
Под угловыми скобками здесь понимается выборочное среднее с весами $w_i=1/\sigma_{y_i}^2$.
Тогда
\[
\sigma_{k}^{2}=\frac{1}{W^{2}D_{xx}^{2}}\sum\limits _{i}w_{i}^{2}\left(x_{i}-\left\langle x\right\rangle \right)^{2}\sigma_{y_{i}}^{2}.
\]
Учитывая, что $w_{i}\sigma_{y_{i}}^{2}=1$, получим
\begin{equation}
\sigma_{k}^{2}=\frac{1}{W D_{xx}}.\label{eq:MMP_sigma_k}
\end{equation}

Аналогично, для погрешности свободного члена имеем
\[
\sigma_{b}^{2}=\sum_{i}\left(\frac{\partial b}{\partial y_{i}}\right)^{2}\sigma_{y_{i}}^{2},
\]
где 
\[
\frac{\partial b}{\partial y_{i}}=\frac{w_{i}}{W}+\frac{\partial k}{\partial y_{i}}\left\langle x\right\rangle =\frac{w_{i}}{W}\left(1-\frac{x_{i}-\left\langle x\right\rangle }{\left\langle x^{2}\right\rangle -\left\langle x\right\rangle ^{2}}\left\langle x\right\rangle \right)=\frac{w_{i}}{W}\frac{\left\langle x^{2}\right\rangle -x_{i}\left\langle x\right\rangle }{D_{xx}}.
\]
Отсюда, пользуясь (\ref{eq:MMP_sigma_k}), приходим к формуле (\ref{eq:MNK_sigma_b}):
\begin{equation}
\sigma_{b}^{2}=\sigma_{k}^{2}
\frac{\left\langle \left(\left\langle x^{2}\right\rangle -x\left\langle x\right\rangle \right)^{2}\right\rangle }%
{D_{xx}}=
\sigma_{k}^{2}\left\langle x^{2}\right\rangle .
\end{equation}

\paragraph{Случай $\sigma_{y}=\mathrm{const}$.}
В частном случае метода наименьших квадратов (п. \ref{sec:MNK}),
формула (\ref{eq:MMP_sigma_k}) упрощается:
\begin{equation}
\sigma_{k}^{2}=\frac{\sigma_{y}^{2}}{nD_{xx}},\qquad\sigma_{b}^{2}=\sigma_{k}^{2}
\average{x^{2}}.\label{eq:MMP_sigma_k_simple}
\end{equation}
Здесь величина $\sigma_{y}$ может быть оценена непосредственно из
экспериментальных данных:
\begin{equation}
\sigma_{y}\approx\sqrt{\frac{1}{n-2}\sum_{i}\Delta y_{i}^{2}},\label{eq:MMP_sigma_y}
\end{equation}
где $n-2$ --- число \textquote{степеней свободы}
для приращений $\Delta y_{i} = y_i - (kx_i+b)$ ($n$ точек за вычетом двух связей
(\ref{eq:MMP})).

Формул (\ref{eq:MMP_sigma_k_simple}) и (\ref{eq:MMP_sigma_y}),
вообще говоря, достаточно для вычисления погрешности величины $k$
по известным экспериментальным точкам. Однако часто их объединяют
в одно упрощённое выражение. Для этого преобразуем (\ref{eq:MMP_sigma_y})
следующим образом: учитывая, что $\left\langle y\right\rangle =k\left\langle x\right\rangle +b$,
запишем
\[
\Delta y_{i}=y_{i}-kx_{i}-b=\left(y_{i}-\left\langle y\right\rangle \right)-k\left(x_{i}-\left\langle x\right\rangle \right).
\]
Возведём в квадрат, усредним и воспользуемся выражением для $k$ в
форме (\ref{eq:MNK_short}):
\[
\left\langle \Delta y^{2}\right\rangle =D_{yy}+k^{2}D_{xx}-2kD_{xy}=D_{yy}-k^{2}D_{xx}.
\]
Таким образом,
\[
\sigma_{y}=\sqrt{\frac{n}{n-2}\left(D_{yy}-k^{2}D_{xx}\right)},
\]
и с помощью (\ref{eq:MMP_sigma_k_simple}) получаем формулы (\ref{eq:MNK_sigma_k}),
(\ref{eq:MNK_sigma_b}):
\[
\boxed{\sigma_{k}=\sqrt{\frac{1}{n-2}\left(\frac{D_{yy}}{D_{xx}}-k^{2}\right)},\qquad\sigma_{b}=\sigma_{k}\sqrt{\left\langle x^{2}\right\rangle }}.
\]

\section{Многопараметрические оценки}
\label{sec:multiparam}

Однопараметрические оценки просты для понимания и реализации, но относительно
редко встречаются на практике. Даже при оценке параметров линейной зависимости
$y = k x + b$ требуется уже два параметра: наклон $k$ смещение $b$.
Все рассмотренные выше методы нахождения оптимальных параметров работают и
в многомерном случае, но поиск экстремума функций
(например, максимума функции правдоподобия или минимума суммы квадратов)
и интерпретация результатов требуют, как правило, использования численных методов.

\subsection{Двумерный случай}
Остановимся подробнее на построении прямой. Пусть некоторым методом получены
точечные оценки для наилучших значений $\hat{k}$ и $\hat{b}$.
Однако самих значений мало --- нас интересует область, в которой могут
оказаться параметры $k$, $b$ с некоторой доверительной вероятностью
(например, $P=0,68$) --- двумерная доверительная область.

Предположим для простоты, что оценки параметров имеют нормальное или близкое
к нему распределение (это разумное предположение, если результаты получены
из большого числа независимых измерений).

Если бы $k$ и $b$ были независимы, достаточно было бы найти среднеквадратичные
отклонения $\sigma_k$ и $\sigma_b$, как это сделано в п. \ref{sec:sigma_kb}:
тогда искомая доверительная область значений параметров на плоскости $(k,b)$
представляла бы собой эллипс, оси которого параллельны координатным
(см. рис.~\ref{fig:kb}а).

Однако, если взглянуть, к примеру, на рис. \ref{fig:graph-method}б, иллюстрирующий
графический метод построения прямой, можно убедиться, что при варьировании
наклона $k$ обязательно меняется и смещение $b$. То есть параметры
$(k,\,b)$ вообще говоря \emph{скореллированы}. Количественно отклонения параметров
будут характеризоваться ковариационной матрицей:
\[
D = \left(\begin{matrix}
    D_{kk} & D_{kb} \\
    D_{bk} & D_{bb}
\end{matrix}\right),
\]
где $D_{kk}=\sigma_k^2$, $D_{bb}=\sigma_b^2$ --- дисперсии искомых параметров,
а 
\begin{equation*}
    D_{kb}=D_{bk} = \average{(k-\average{k})\cdot(b-\average{b})} = \rho_{kb}\sigma_k\sigma_b.  
\end{equation*}

Коэффициент $\rho_{kb}$ называют коэффициентом корреляции и он служит показателем \textquote{связанности} параметров. Для полностью независимых параметров он равен нулю, а в случае, если параметры нельзя отличить друг от друга --- единице.

По известной теореме линейной алгебры, симметричную матрицу можно
привести к диагональному виду поворотом координатных осей. Поэтому доверительная
область в таком случае будет представлять собой \emph{наклонный} эллипс
(см. рис.~\ref{fig:kb}б), а наклон его осей будет определяться
коэффициентом корреляции $r_{kb}$.
% Уравнение эллипса:
% \[
% \frac{k^2}{D_{kk}} - 2r_{kq}^2 k b + \frac{b^2}{D_{bb}}
% \]

\begin{figure}[h]
    \centering
    \input{images/kb.pdf_tex}
    \caption{Доверительная область значений коэффициентов прямой а)~$k$ и~$b$ независимы,
б)~$k$ и~$b$ скоррелированы.}
    \label{fig:kb}
\end{figure}


\subsection{Многомерный случай}

Принцип построения доверительной области в многомерном случае точно
такой же, как и для одномерных доверительных интервалов. Требуется найти
такую областью пространства параметров $\Omega$, для которой
вероятностное содержание для оценки параметра $\hat \theta$
% (или
% самого параметра $\theta$ в зависимости от того, какой философии вы
% придерживаетесь)
будет равно некоторой наперед заданной
величине $\alpha$:
\begin{equation}
    P(\theta \in \Omega) = \int\limits_\Omega{L(\mathbf{x} | \theta)}d\Omega = \alpha.
\end{equation}

Построение многомерной доверительной области на практике сталкивается с тремя
проблемами:
\begin{enumerate}
\item Взятие многомерного интеграла от произвольной функции~--- не тривиальная
  задача. Даже в случае двух параметров требуется владение
  методами вычислительной математики. Соответствующие методы реализованы
  в специализированных программных пакетах.
\item Определение центрального интервала для многомерной гиперобласти
является неоднозначной задачей.
\item Даже если удалось получить доверительную область, описать многомерный
объект в общем случае непросто, так что представление результатов
  составляет определенную сложность.
\end{enumerate}

Для решения этих проблем пользуются следующим приемом: согласно
центральной предельной теореме, усреднение большого количества одинаково
распределенных величин дает нормально распределенную величину. Это же
верно и в многомерном случае. В большинстве случаев, мы ожидаем, что
функция правдоподобия будет похожа на многомерное нормальное
распределение:
\begin{equation}
    L(\theta) = \frac{1}{(2 \pi)^{n/2}\left|D\right|^{1/2}} e^{-\frac{1}{2}
        (\mathbf{x} - \limaverage{\mathbf{x}})^T D^{-1} (\mathbf{x} - \limaverage{\mathbf{x}})},
\end{equation}
где $n$~--- размерность вектора параметров, $\limaverage{\mathbf{x}}$~--- вектор
наиболее вероятных значений, а $D$~--- ковариационная матрица распределения.

Для многомерного нормального распределения, линии постоянного уровня (то
есть поверхности, на которых значение плотности вероятности одинаковые)
имеют вид гиперэллипса, определяемого уравнением
$(\mathbf{x} - \limaverage{\mathbf{x}})^T D^{-1} (\mathbf{x} - \limaverage{\mathbf{x}}) = \mathrm{const}$.
Для любого вероятностного содержания $\alpha$ можно подобрать эллипс, который будет
удовлетворять условию на вероятностное содержание. Интерес, правда,
представляет не сам эллипс (в случае размерности больше двух, его просто
невозможно наглядно изобразить), а \emph{ковариацонная матрица}. Диагональные элементы
этой матрицы являются дисперсиями соответствующих параметров (с учетом
всех корреляций параметров).

\subsection{Использование пакета \texttt{scipy} для построение оценки}

Существует огромное количество программных пакетов для построения численной оценки параметров. Наиболее доступным и широко используемым является пакет \texttt{scipy} на языке Python. Приведем здесь только пример вызова процедуры оптимизации.

Пусть есть экспериментальные данные, представленные в виде трех колонок: $x$, $y$ и $err$. Требуется построить наилучшую прямую, описывающую эти данные.
Код для этого будет выглядеть следующим образом:
\begin{minted}[frame=lines, fontsize=\footnotesize]{python}
    from scipy.optimize import curve_fit

    function = lambda x, a, b: a*x + b

    popt, pcov = curve_fit(function, xdata = x, ydata = y, sigma = err)
\end{minted}

После выполнения этого кода, переменная \texttt{popt} содержит массив из двух значений, соответствующих оценке \texttt{a} и \texttt{b}, а переменная \texttt{pcov} содержит ковариационную матрицу для полученных параметров. 

Погрешности параметров можно получить как корни из диагональных элементов ковариационной матрицы:
\begin{minted}[frame=lines, fontsize=\footnotesize]{python}
    import numpy as np
    
    sigma_a = np.sqrt(pcov[0,0])
    sigma_b = np.sqrt(pcov[1,1])
\end{minted}

\note{Следует отметить, что существует огромное количество способов оценки оптимальных значений параметров и ковариационной матрицы. Поэтому при использовании того или иного инструмента, всегда следует сверяться с документацией и выяснять, что именно он делает. Также следует всегда проверять результаты обработки из качественных, \textquote{наивных} соображений.}



% \section{Проверка гипотез}
%
% Предположим, что теория предсказывает некоторую зависимость
% \[
% y=f\!\left(x;a,b,\ldots\right),
% \]
% а в эксперименте получен набор значений $\left\{ x_{i},\,y_{i}\right\} $.
% Метод максимального правдоподобия позволяет получить параметры
% $\left\{ a,b,\ldots\right\} $
% функции $f$, \textquote{наилучшим} образом приближающие
% экспериментальные значения. Причём, как бы плохо экспериментальные
% точки не ложились на теоретическую кривую, ответ будет получен в любом
% случае. Как проверить, действительно ли измеряемые величины можно
% считать связанными зависимостью $y=f\!\left(x\right)$?
%
% Задача может быть решена, если, как обычно, сделать ряд упрощающих
% предположений\footnote{Отметим, что в общем случае такая проверка не осуществима:
% в частности, если погрешности экспериментальных точек велики или не
% известны, то через них с равным \textquote{успехом}
% можно провести почти \emph{любую} функцию!}.
% Пусть опять все ошибки измерения \emph{независимы},
% распределены \emph{нормально} и нам известны их
% среднеквадратичные значения $\sigma_{y_{i}}$
% (или хотя бы их грубые оценки).
%
% Рассмотрим определённую выше сумму \emph{хи-квадрат} (\ref{eq:chi2})
% как функцию $n$ переменных $\left\{ y_{i}\right\}$:
% \begin{equation}
% \chi^{2}\!\left(y_{1},\,y_{2},\,\ldots,\,y_{n}\right)=\sum\limits _{i=1}^{n}\left(\frac{y_{i}-f\!\left(x_{i}\right)}{\sigma_{y_{i}}}\right)^{2}.\label{eq:chi2-1}
% \end{equation}
% Пусть функция $f\!\left(x\right)$ содержит $p$ \textquote{подгоночных}
% параметров (например, $p=2$ для линейной зависимости $f\!\left(x\right)=kx+b$).
% Найдём их наилучшие значения по методу максимального правдоподобия
% ($\chi^{2}\to\mathrm{min}$), и зафиксируем их. После этого $\chi^{2}$
% можно рассматривать как функцию
% \[
% m=n-p
% \]
% независимых переменных. Величину $m$ назовём \emph{числом степеней свободы} задачи.
%
% Попробуем сперва качественно ответить на вопрос, какое значение
% величины $\chi^{2}$ можно ожидать, если зависимость $y=f\!\left(x\right)$
% справедлива? Ясно, что если распределение ошибок нормальное, можно
% ожидать отклонений порядка среднеквадратичного: $\Delta y_{i}=y_{i}-f\!\left(x_{i}\right)\sim\sigma_{y_{i}}$.
% Поэтому значение суммы (\ref{eq:chi2-1}) должно оказаться порядка
% числа входящих в неё независимых слагаемых: $\chi_{m}^{2}\sim m$.
% В теории вероятностей доказывается (см., например, \cite{hudson}), что ожидаемое среднее значение (математическое ожидание) $\chi^{2}$ в точности равно числу степеней свободы: $\average{\chi_{m}^{2}}=m$.
%
% Теперь можно сформулировать качественный критерий проверки
% гипотезы о наличии некоторой функциональной зависимости (его называют
% \emph{критерий хи-квадрат}:
% \begin{itemize}
% \item  если $\chi^{2}\sim m$, согласие эксперимента с теорией \emph{удовлетворительное}
% (гипотеза не опровергнута);
% \item если $\chi^{2}\gg m$ --- \emph{согласия нет},
% то есть гипотеза о зависимости $y=f\!\left(x\right)$ скорее всего не верна.
% \end{itemize}
% Заметим, что если вдруг $\chi^{2}\ll m$, то совпадение \emph{слишком}
% хорошее, и скорее всего имеет место завышенная оценка для случайных
% погрешностей измерения $\sigma_{y_{i}}$.
%
% Для того, чтобы дать строгий \emph{количественный}
% критерий, с какой долей вероятности гипотезу $y=f\!\left(x\right)$
% можно считать подтверждённой или опровергнутой, нужно детально исследовать
% вероятностный закон, которому подчиняется функция $\chi^{2}$. В теории
% вероятностей он называется \emph{распределение хи-квадрат}
% (с $m$ степенями свободы). В элементарных функциях распределение
% хи-квадрат не выражается, но может быть легко найдено численно: функция
% встроена во все основные статистические пакеты, либо может быть вычислена
% по таблицам. Как правило, определяется вероятность
% $P\left(\chi^{2}>\chi_{0}^{2}\right)$
% того, что хи-квадрат имеет значение больше некоторого $\chi_{0}^{2}$,
% вычисленного из эксперимента. Если эта вероятность достаточно мала
% (например, $P<5\%$; конкретная величина доверительной вероятности
% всегда остаётся на усмотрение исследователя), соответствующую гипотезу
% следует признать несостоятельной.
%
% {\footnotesize
% \textbf{Пример.} Для данных на рис.~\ref{fig:correct}
% при $\sigma_{y}=0{,}2$ см величина хи-квадрат равна $\chi^{2}\approx4{,}7$.
% За вычетом двух параметров линейной аппроксимации имеем $m=6-2=4$
% степеней свободы. По графику TODO определяем, что вероятность того,
% что согласие окажется хуже, чем на TODO (т.е. разброс точек относительно
% \emph{той же} прямой будет больше),
% составляет $P\sim30\%$. Оснований для отказа от <<гипотезы
% о линейной зависимости>> нет.
%
% Если бы погрешность каждой точки была равна $\sigma_{y}=0{,}14$
% см, то мы получили бы $\chi^{2}=9{,}7$. Это соответствует $P<5\%$,
% так что считать зависимость линейной, по-видимому, было бы нельзя
% (либо не верна оценка для погрешности $\sigma_{y}$).\par
% }%\footnotesize
%
% Напоследок еще раз подчеркнём, что критерий хи-квадрат, во-первых,
% статистический и не может дать однозначного ответа --- только
% вероятностную оценку, а во-вторых, он работает корректно при условии,
% что ошибки разных точек независимы и каждая имеет нормальное распределение.
% Эти предположения, вообще говоря, выполняются далеко не всегда и,
% по-хорошему, требуют отдельной проверки.


\begin{thebibliography}{1}

\bibitem{taylor} \textit{Тейлор Дж}. Введение в теорию ошибок.

\bibitem{squires} \textit{Сквайрс Дж.} Практическая физика.

\bibitem{zaidel} \textit{Зайдель А.Н.} Погрешности измерений физических величин.

\bibitem{hudson} \textit{Худсон Д.} Статистика для физиков.

\bibitem{idie} \textit{Идье В.}, \textit{Драйард Д.}, \textit{Джеймс Ф}., \textit{Рус М.}, \textit{Садуле Б.} Статистические методы в экспериментальной физике.

\end{thebibliography}

\end{document}
